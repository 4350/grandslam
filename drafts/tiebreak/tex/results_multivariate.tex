%!TEX root = ../main.tex

\section{Multivariate dependence in univariate residuals} % (fold)
%\label
% explain the logic in what we are doing, that there might still be dependence in what looks like white noise residuals
After the univariate modeling of returns in ARMA-GARCH models, we find that the ARMA-GARCH residuals look like white noise, with no serial correlation or ARCH effects. However, when considered jointly, we find that there is substantial dependence remaining in the residual series -- as will be shown in this section, correlation is non-zero and time-varying, and there is tail dependence between the residual series. 

Therefore, we do not use returns directly in the multivariate analysis, as significant parts of the dependence can be removed by univariate models alone. Instead, we use the residuals from the ARMA-GARCH models. The intuition behind this is that we do not want to confuse multivariate dependency in returns with what are in fact just predictable effects that can be understood on the univariate level alone.

Our philosophy, which is in line with the copula methodology, is to try to sanitize the data as far as possible on the univariate level. Then, any dependence left in the residuals can be attributed to multivariate dependence.

\subsection{Threshold correlations}
%\label
% method
Threshold (or exceedance) correlations have previously been used to highlight the asymmetric dependence structure of i.a. country equity indices~\autocite{LonginSolnik2001}, portfolios by industry, size, value and momentum~\autocite{AngChen2002} and factor strategies~\autocite{ChristoffersenLanglois2013}. The following analysis is still new as it adds the factors investment (CMA) and profitability (RMW). We follow~\textcite{ChristoffersenLanglois2013} definition of threshold correlation
\begin{align}
    ThCorr(r_i, r_j) = 
    \begin{cases} 
        Corr\Big(r_i, r_j \,|\, r_i < F_i^{-1}(p), r_j < F_j^{-1}(p)\Big)  & \text{for } p < 0.5 \\
        Corr\Big(r_i, r_j \,|\, r_i \geq F_i^{-1}(p), r_j \geq F_j^{-1}(p)\Big)  & \text{for } p \geq 0.5
    \end{cases}
\end{align}
where $F_i^{-1}(p)$ the empirical quantile of $r_i$ at percentile $p$. 

Graphically, threshold correlation are understood as the standard correlation estimate in more and more distant parts of the first and third quadrant as $p$ approaches either zero or unity. This subsetting of data is illustrated in \autoref{fig:illustrate_threshold}. 

On the left hand plot, we see the scatter of returns on Mkt-RF and HML, and how $p$, found on the x axis of the right hand plot, determines the subset of data that is included in the correlation calculation. Note that the further away from the median we get, the more uncertain the point estimate of the correlation coefficient becomes, as fewer observations are available. For the 49\% and 51\% points, estimates are much more precise as essentially half the data set is available. We also note that the standard correlation, given by the dashed line in the right hand plot, is clearly negative, while threshold correlations in the first and third quadrants are significantly more positive, which shows the usefulness of this analysis; not taking threshold correlations into account provides a vaguer picture of the dependence structure.

Although threshold correlations are nothing but linear correlation coefficients for a subset of the data, they do provide an intuitive measure of tail dependence -- how correlated are returns in bad times? And does this differ from the correlation in good times? Ceteris paribus, assets pairs with weaker or negative threshold correlation as $p \rightarrow 0$ are better diversified, as they have less tendency to coincide in extreme negative realizations. [ADD SOMETHING ABOUT Between two normal distributed variables you expect /\-shape (for positively correlated variables).]

% plots [possibly add diff series/residuals]

% talk
As previously noted, we use residuals from the ARMA-GARCH models in the threshold correlation analysis instead of returns directly. The plots for the standardized residuals are presented in~\autoref{fig:threshold_corr1, threshold_corr2}. The patterns of returns and residuals are highly similar, with returns exhibiting more extreme patterns (see \autoref{app:threshold_returns1, app:threshold_return2} for a comparison).

First, we note that the threshold correlation is significantly different from the unconditional correlation coefficient denoted by the dashed line for most factor pairs (with the exception of Mkt-RF - RMW, and CMA - HML). This means that the multivariate dependence of factor strategies has tail dependence that is not well captured by the standard correlation measure. Conditional on being in the first or third quadrant of returns, assuming a correlation coefficient of the unconditional correlation is misleading. A concrete example is provided by the Mkt-RF and HML pair. In a period when both residuals are observed below their 25\% percentile, the correlation is most certainly not -0.30, but actually positive.

In our multivariate analysis, we would like a model of joint returns to capture this behavior. Risk management is inherently a concern about what happens in the extremes, and the multivariate model should capture some of the tail dependence we observe in threshold correlations.

Second, we note that there is asymmetry around the median for some factor pairs, including the Mom - CMA, RMW - HML, RMW - CMA, and to a lesser extent Mkt-RF - RMW asset pairs. For example, in the Mom-CMA asset pair, the threshold correlation jumps up for the first percentile below the median, indicating that the correlation is higher when both realize below the median than when both realize above the median. This type of asymmetric property, where downside (below the median) correlation is higher than upside correlation is unwanted, as it reflects a poorer diversification in bad times. A good model of joint returns in factor strategies should therefore have flexibility to generate this type of asymmetry.

Third, we note that, although estimated with substantial uncertainty, the threshold correlation do not seem to be constant as $p$ approaches either zero or one. For example, the Mkt-RF - HML asset pair seems to have a downward pattern, where correlation is the most positive in the lowest percentiles of residuals and the most negative in the highest percentile of residuals. In fact, this pattern is unwanted from a diversification perspective, as diversification benefits decrease when the series are realizing in the lowest percentiles.

Fourth, while all other asset pairs exhibit unconditional correlations of around zero or negative, the CMA - HML pair exhibits high and positive correlation of more than 0.60. Furthermore, the threshold correlation estimate is much closer to the unconditional correlation. While all other factor strategy pairs appear to be very good diversifiers, CMA and HML factors look quite similar. This further begs the question whether they should both be included in factor strategy portfolios.

\subsection{Rolling correlation}
%\label
% method
We compute rolling correlation estimates in order to investigate whether the factor strategies exhibit constant correlation over time. 
\begin{align}
    RCorr(r_{i, t}, r_{j, t})_t^{w} = \frac{\sum^{t}_{t-w+1}(r_{i, t} - \bar{r}_i)(r_{j,t} - \bar{r}_j)}{\sqrt{\sum^{t}_{t-w+1} (r_{i,t} - \bar{r}_i)^2} \sqrt{\sum^{t}_{t-w+1} (r_{j,t} - \bar{r}_j)^2}}
\end{align}
where $r_i$, $r_j$ are the $N \cdot (N-1) / 2$ different pairs of the factor strategies' ARMA-GARCH residuals or log returns and $w$ is the rolling window of data. We use one year of data with $w = 52$ on weekly data. As previously, we use residuals from the ARMA-GARCH models in the rolling correlation analysis instead of returns directly and present results in~\autoref{fig:rolling_correlations}.\footnote{A comparison of rolling correlations on returns and residuals is available in~\autoref{app:rolling_correlations}.}
% plots

% talk
First, we note that for most factor pairs, the rolling 52-week correlations are highly time-varying. While the unconditional correlation is often around zero, the rolling 52-week correlation ranges between approx. -0.75 and 0.75 for the HML - Mkt-RF factor pair. The time-varying correlations lead us to believe that a suitable model of joint returns will incorporate dynamic correlations. As long as there is some persistence in the series, a dynamic model should have merits in modeling returns jointly, as the best guess of tomorrow's correlation then is not simply the unconditional correlation.

Second, by visual inspection, the correlations of a few asset pairs could be suspected to be non-stationary, either due to overall trending or break points -- for example, the HML - Mkt-RF and CMA - HML asset pairs might have a break point around year 2000, as correlations are higher after this period; and the HML - CMA asset pair might have a break point around the same time, as correlations move outside a relatively close range of 40-90\% to substantially lower levels during 2000-2010. If trends or breakpoints were shown to be present, they could be incorporated in the model of joint returns. We believe that especially the potential upward trend in HML - Mkt-RF and CMA - Mkt-RF is highly interesting, and could mean that diversification benefits of these pairs is becoming weaker, but do not explore this further. In our model, we consider correlations to be stationary.

Third, the HML and CMA factors again stand out as different to other asset pairs. The unconditional correlation is much closer to the rolling estimates than for other factor pairs. However, while the correlation is quite high and constant for much of the early sample period, the lower correlations towards the end give reasons to think that they might in fact provide strong diversification benefits during certain times.

\subsection{Conclusions from analysis of multivariate dependence in univariate residuals}
Univariate residuals appear to be white noise series with no serial correlation or ARCH effects. However, there exists important dependence between residuals of different strategies. First, threshold correlations show that there is tail dependence -- in times when factor pairs simultaneously realize in their best and worst percentiles, correlations are significantly different from the unconditional mean. In fact, threshold correlations are substantially higher than the unconditional mean, which indicates that diversification benefits are smaller than expected when factors simultaneously have low returns. Second, rolling correlations show that correlations between series are highly time-varying and seem to exhibit persistence. This should mean that a dynamic model of correlations will improve on the description of joint returns.

Both analyses also show that the HML - CMA asset pair is quite different from the other pairs, exhibiting a much higher and more stable dependence. Differently put, they look quite similar as opposed to any other factor pair, and the merit of including both in factor portfolios seems more unclear.

For modeling of multivariate dependence, we expect that good candidate models will incorporate both tail dependence and dynamic correlations.

\section{Modeling multivariate dependence: Copula} % (fold)
%\label


\subsection{Multivariate model selection}
%\label
% why copula?
% explain logic and intuition: this is a parsimonious way, and solves the two issues with asymmetry and time-variation to some extent
% other models BEKK, etc have issues with dimensionality or can't capture the features of threshold and rolling

\subsection{Copula}
%\label
% method of model
% estimation of model

\subsection{Copula results with constant correlations}
%\label
% diagnostics of fit
% theory
A key purpose of the copula is to better capture tail dependence between factor series. Tail dependence was demonstrated in the threshold correlation analysis, and we may now compare the empirical threshold correlations to simulated threshold correlations from the constant copula model. 

By simulating a 250,000 weeks of shocks in the copula, and then transforming these shocks into standardized residuals for each of the factors, we can test copula's ability to generate the tail dependence in the data. The results are given in~\autoref{fig:threshold_copula1, threshold_copula2}.

% plot

% talk
In each plot, we present the empirical threshold correlation with its associated 95\% confidence bound, along with three copula models: the standard normal copula, the Student's \textit{t} copula and the skewed Student's \textit{t}.

First, we note that for most factors, the normal copula is the farthest away from generating threshold correlations that correspond to the empirical distribution around the median. More specifically, it seems to underestimate the threshold correlation, i.e. not generate sufficient tail dependence. The Student's \textit{t} and skewed Student's \textit{t} copulae better capture the threshold correlations, as the fatter tails of the Student's \textit{t} distribution allows for tail dependence. For example, note how the normal copula generates negative threshold correlations for both the Mom - HML and RMW - HML asset pairs, while the Student's \textit{t} based copulae are much closer to the higher values in the data. On the other hand, the Student's \textit{t} based copulae sometimes seem to overshoot the empirical threshold correlation, as in the Mkt-RF - RMW asset pair.

Second, we find that although the skewed Student's \textit{t} does generate some asymmetry around the mean, which can be seen most clearly for the Mom - RMW and RMW - HML asset pairs, the asymmetry is far too weak to be an accurate description of the data. 

In conclusion, threshold correlation comparison between empirical and simulated data shows the limitations of our copula approach. Although the copulae are flexible and can express tail dependence, which is a clear improvement to no tail dependence at all, it does not overlap very well with the data. For example, the Student's \textit{t} copula only has one degree of freedom parameter that controls the fatness of tails, and the skewed Student's \textit{t} copula only has one skewness parameter for each series. This imposes limits on how strongly the model can express fat tails or asymmetries between factors A and B and simultaneously express other fat tails or asymmetries (or lack thereof) between factors A and C. For a collection of six factors with heterogenous dependence, this is even harder. This is a clear limitation of the quite parsimonious copula approach, and will be discussed further in the concluding section. Although imperfect, the copula modeling of tail dependence could constitute a significant improvement to alternatives, especially in the field of risk management, where understanding of tail events is paramount.

\subsection{Copula results}
% TABLES NEED TO BE MODIFIED IN THE FOLLOWING WAYS
% 1) Change {tabular} to {tabularx}{\textwidth} and make leftmost column an X column
%     and change top and bottom \hline to \toprule \bottomrule
%
% paste the following at start but before & \multicolumn
%
% \begin{tabularx}{\textwidth}{@{\extracolsep{5pt}} X D{.}{.}{-3} D{.}{.}{-3} D{.}{.}{-3} } 
% \\[-1.8ex] \midrule
% \\[-1.8ex] 
%
% paste the following at end after R2 row but before Note row
% \bottomrule \\[-1.8ex] 
%
% 2) Change the variable names to greeks
% 3) Change specification names if needed
% 4) Change R2 to LLH and add similar lines for Ljung-Box and ARCH-LM
% 5) Add label and caption
% 6) Paste this to get table heading description
% 7) Copy table heading tabularx footnote size text
%
% \begin{tabularx}{\textwidth}{X}
% \\[-1.8ex]\toprule
%\\[-1.8ex] 
% text goes here
% \end{tabularx}
%
% 6) Copy the whole table, only change caption, label, factor/spec labels and (1)-(3) to (4)-(6)
% Table created by stargazer v.5.2 by Marek Hlavac, Harvard University. E-mail: hlavac at fas.harvard.edu
% Date and time: ons, okt 12, 2016 - 12:37:02
% Requires LaTeX packages: dcolumn 
\begin{table}[!htbp] \centering 
  \caption{Copula results: Constant specifications} 
  \label{tab:copula_estimates_constant} 
\begin{tabularx}{\textwidth}{X}
  \\[-1.8ex]\toprule
  \\[-1.8ex] 
  \footnotesize Parameter estimates from constant copula models based on uniform residuals from ARMA-GARCH models. Stationary bootstrap standard errors in parentheses, following \textcite{PolitisRomano1994}. Copula parameters: $\nu$ is the degree of freedom, $\gamma$ is the vector of skewness parameters, $\alpha, \beta$ are the shock loading and autoregressive loading of the \textit{c}DCC process. All data 1963-07-05 - 2016-07-01. 
\end{tabularx}
\begin{tabularx}{\textwidth}{@{\extracolsep{5pt}} X D{.}{.}{-3} D{.}{.}{-3} D{.}{.}{-3} } 
  \\[-1.8ex]\midrule
  \\[-1.8ex] 
   & \multicolumn{3}{c}{Constant copula models} \\ 
  \cline{2-4} 
  \\[-1.8ex] & \multicolumn{1}{c}{(1)} & \multicolumn{1}{c}{(2)} & \multicolumn{1}{c}{(3)}\\ 
  \\[-1.8ex] & \multicolumn{1}{c}{Gaussian} & \multicolumn{1}{c}{Student-\textit{t}} & \multicolumn{1}{c}{Skewed Student-\textit{t}}\\ 
  \hline \\[-1.8ex] 
 $\nu$ &  & 6.625 & 6.671 \\ 
  &  & () & () \\ 
  & & & \\ 
 $\gamma_{Mkt.RF}$ &  &  & -0.057 \\ 
  &  &  & () \\ 
  & & & \\ 
 $\gamma_{SMB}$ &  &  & -0.103 \\ 
  &  &  & () \\ 
  & & & \\ 
 $\gamma_{Mom}$ &  &  & -0.202 \\ 
  &  &  & () \\ 
  & & & \\ 
 $\gamma_{HML}$ &  &  & 0.103 \\ 
  &  &  & () \\ 
  & & & \\ 
 $\gamma_{CMA}$ &  &  & 0.076 \\ 
  &  &  & () \\ 
  & & & \\ 
 $\gamma_{RMW}$ &  &  & 0.021 \\ 
  &  &  & () \\ 
  & & & \\ 
\hline \\[-1.8ex] 
Observations & \multicolumn{1}{c}{2,766} & \multicolumn{1}{c}{2,766} & \multicolumn{1}{c}{2,766} \\ 
LLH & \multicolumn{1}{c}{1,169} & \multicolumn{1}{c}{1,556} & \multicolumn{1}{c}{1,573} \\ 
No. parameters & \multicolumn{1}{c}{15} & \multicolumn{1}{c}{16} & \multicolumn{1}{c}{22} \\ 
BIC & \multicolumn{1}{c}{-2,220} & \multicolumn{1}{c}{-2,985} & \multicolumn{1}{c}{-2,971} \\ 
Correlation $(Q)$ persistence $(\alpha+\beta)$ & \multicolumn{1}{c}{N/A} & \multicolumn{1}{c}{N/A} & \multicolumn{1}{c}{N/A} \\ 
\bottomrule \\[-1.8ex] 
\textit{Note:}  & \multicolumn{3}{c}{$^{*}$p$<$0.1; $^{**}$p$<$0.05; $^{***}$p$<$0.01} \\ 
\end{tabularx} 
\end{table} 


\subsection{Copula results with dynamic correlations}
%\label
% still haven't captured dynamic, lets do this now
% diagnostics of fit
% why we prefer symmetric t
% correlation patterns over time in-sample graph looks very much like the data rolling
% Table created by stargazer v.5.2 by Marek Hlavac, Harvard University. E-mail: hlavac at fas.harvard.edu
% Date and time: ons, okt 12, 2016 - 12:37:02
% Requires LaTeX packages: dcolumn 
\begin{table}[!htbp] \centering 
  \caption{Copula results: \textit{c}DCC specifications} 
  \label{tab:copula_estimates_dynamic} 
\begin{tabularx}{\textwidth}{X}
\\[-1.8ex]\toprule
\\[-1.8ex] 
\footnotesize Parameter estimates from dynamic copula models based on uniform residuals from ARMA-GARCH models. Stationary bootstrap standard errors in parentheses, following \textcite{PolitisRomano1994}. Copula parameters: $\nu$ is the degree of freedom, $\gamma$ is the vector of skewness parameters, $\alpha, \beta$ are the shock loading and autoregressive loading of the \textit{c}DCC process. All data 1963-07-05 - 2016-07-01. 
\end{tabularx}
\begin{tabularx}{\textwidth}{@{\extracolsep{5pt}} X D{.}{.}{-3} D{.}{.}{-3} D{.}{.}{-3} } 
\\[-1.8ex]\midrule
\\[-1.8ex] 
 & \multicolumn{3}{c}{Dynamic copula models} \\ 
\cline{2-4} 
\\[-1.8ex] & \multicolumn{1}{c}{(4)} & \multicolumn{1}{c}{(5)} & \multicolumn{1}{c}{(6)}\\ 
\\[-1.8ex] & \multicolumn{1}{c}{Gaussian} & \multicolumn{1}{c}{Student-\textit{t}} & \multicolumn{1}{c}{Skewed Student-\textit{t}}\\ 
\hline \\[-1.8ex] 
 $\nu$ &  & 11.936 & 11.881^{***} \\ 
  &  & () & (1.064) \\ 
  & & & \\ 
 $\gamma_{Mkt.RF}$ &  &  & -0.078 \\ 
  &  &  & (0.054) \\ 
  & & & \\ 
 $\gamma_{SMB}$ &  &  & -0.175^{**} \\ 
  &  &  & (0.077) \\ 
  & & & \\ 
 $\gamma_{Mom}$ &  &  & -0.145^{*} \\ 
  &  &  & (0.073) \\ 
  & & & \\ 
 $\gamma_{HML}$ &  &  & 0.083 \\ 
  &  &  & (0.058) \\ 
  & & & \\   
 $\gamma_{CMA}$ &  &  & 0.001 \\ 
  &  &  & (0.063) \\ 
  & & & \\ 
 $\gamma_{RMW}$ &  &  & 0.095 \\ 
  &  &  & (0.061) \\ 
  & & & \\ 
 $\alpha$ & 0.065 & 0.068 & 0.068^{***} \\ 
  & () & () & (0.007) \\ 
  & & & \\ 
 $\beta$ & 0.915 & 0.913 & 0.913^{***} \\ 
  & () & () & (0.011) \\ 
  & & & \\ 
\hline \\[-1.8ex] 
Observations & \multicolumn{1}{c}{2,766} & \multicolumn{1}{c}{2,766} & \multicolumn{1}{c}{2,766} \\ 
LLH & \multicolumn{1}{c}{2,791} & \multicolumn{1}{c}{2,978} & \multicolumn{1}{c}{2,989} \\ 
No. parameters & \multicolumn{1}{c}{17} & \multicolumn{1}{c}{18} & \multicolumn{1}{c}{24} \\ 
BIC & \multicolumn{1}{c}{-5,447} & \multicolumn{1}{c}{-5,813} & \multicolumn{1}{c}{-5,788} \\ 
Correlation $(Q)$ persistence $(\alpha+\beta)$ & \multicolumn{1}{c}{0.981} & \multicolumn{1}{c}{0.981} & \multicolumn{1}{c}{0.981} \\ 
\bottomrule \\[-1.8ex] 
\textit{Note:}  & \multicolumn{3}{c}{$^{*}$p$<$0.1; $^{**}$p$<$0.05; $^{***}$p$<$0.01} \\ 
\end{tabularx} 
\end{table} 
% potentially subsection exogenous regressors
%

\section{Putting the model to work}
%\label
% now, lets see what this model can say about a few things

\subsection{Mean-variance investing}
%\label
% fama french regressions, what actually happens to MV weights, do they go to zero?
% explain method for out-of-sample distributions, simulation based
% show that weights don't go to zero, not for sample and not for advanced super copula model
% what is the out-of-sample investment performance of relying on the best copula model?
% it looks pretty good but not drastically better than relying on sample average - OOS is hard

\subsection{Conditional diversification benefits}
%\label
% what are the theoretical diversificaiton benefits of including hml? is HML or CMA riskier if one can only choose either or? CDB analysis
