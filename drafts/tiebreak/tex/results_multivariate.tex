%!TEX root = ../main.tex

\section{Multivariate dependence in univariate residuals} % (fold)
%\label
% explain the logic in what we are doing, that there might still be dependence in what looks like white noise residuals

\subsection{Threshold correlations}
%\label
% method, plots, [diff between series/residuals], talk

\subsection{Rolling correlation}
%\label
% method, plots, talk

\section{Modeling multivariate dependence: Copula} % (fold)
%\label


\subsection{Multivariate model selection}
%\label
% why copula?
% explain logic and intuition: this is a parsimonious way, and solves the two issues with asymmetry and time-variation to some extent
% other models BEKK, etc have issues with dimensionality or can't capture the features of threshold and rolling

\subsection{Copula}
%\label
% method of model
% estimation of model

\subsection{Copula results with constant correlations}
%\label
% diagnostics of fit
% include threshold correlations, look better for skewed t, gaussian bad
\subsection{Copula results}
% TABLES NEED TO BE MODIFIED IN THE FOLLOWING WAYS
% 1) Change {tabular} to {tabularx}{\textwidth} and make leftmost column an X column
%     and change top and bottom \hline to \toprule \bottomrule
%
% paste the following at start but before & \multicolumn
%
% \begin{tabularx}{\textwidth}{@{\extracolsep{5pt}} X D{.}{.}{-3} D{.}{.}{-3} D{.}{.}{-3} } 
% \\[-1.8ex] \midrule
% \\[-1.8ex] 
%
% paste the following at end after R2 row but before Note row
% \bottomrule \\[-1.8ex] 
%
% 2) Change the variable names to greeks
% 3) Change specification names if needed
% 4) Change R2 to LLH and add similar lines for Ljung-Box and ARCH-LM
% 5) Add label and caption
% 6) Paste this to get table heading description
% 7) Copy table heading tabularx footnote size text
%
% \begin{tabularx}{\textwidth}{X}
% \\[-1.8ex]\toprule
%\\[-1.8ex] 
% text goes here
% \end{tabularx}
%
% 6) Copy the whole table, only change caption, label, factor/spec labels and (1)-(3) to (4)-(6)
% Table created by stargazer v.5.2 by Marek Hlavac, Harvard University. E-mail: hlavac at fas.harvard.edu
% Date and time: ons, okt 12, 2016 - 12:37:02
% Requires LaTeX packages: dcolumn 
\begin{table}[!htbp] \centering 
  \caption{Copula results: Constant specifications} 
  \label{tab:copula_estimates_constant} 
\begin{tabularx}{\textwidth}{X}
  \\[-1.8ex]\toprule
  \\[-1.8ex] 
  \footnotesize Parameter estimates from constant copula models based on uniform residuals from ARMA-GARCH models. Stationary bootstrap standard errors in parentheses, following \textcite{PolitisRomano1994}. Copula parameters: $\nu$ is the degree of freedom, $\gamma$ is the vector of skewness parameters, $\alpha, \beta$ are the shock loading and autoregressive loading of the \textit{c}DCC process. All data 1963-07-05 - 2016-07-01. 
\end{tabularx}
\begin{tabularx}{\textwidth}{@{\extracolsep{5pt}} X D{.}{.}{-3} D{.}{.}{-3} D{.}{.}{-3} } 
  \\[-1.8ex]\midrule
  \\[-1.8ex] 
   & \multicolumn{3}{c}{Constant copula models} \\ 
  \cline{2-4} 
  \\[-1.8ex] & \multicolumn{1}{c}{(1)} & \multicolumn{1}{c}{(2)} & \multicolumn{1}{c}{(3)}\\ 
  \\[-1.8ex] & \multicolumn{1}{c}{Gaussian} & \multicolumn{1}{c}{Student-\textit{t}} & \multicolumn{1}{c}{Skewed Student-\textit{t}}\\ 
  \hline \\[-1.8ex] 
 $\nu$ &  & 6.625 & 6.671 \\ 
  &  & () & () \\ 
  & & & \\ 
 $\gamma_{Mkt.RF}$ &  &  & -0.057 \\ 
  &  &  & () \\ 
  & & & \\ 
 $\gamma_{SMB}$ &  &  & -0.103 \\ 
  &  &  & () \\ 
  & & & \\ 
 $\gamma_{Mom}$ &  &  & -0.202 \\ 
  &  &  & () \\ 
  & & & \\ 
 $\gamma_{HML}$ &  &  & 0.103 \\ 
  &  &  & () \\ 
  & & & \\ 
 $\gamma_{CMA}$ &  &  & 0.076 \\ 
  &  &  & () \\ 
  & & & \\ 
 $\gamma_{RMW}$ &  &  & 0.021 \\ 
  &  &  & () \\ 
  & & & \\ 
\hline \\[-1.8ex] 
Observations & \multicolumn{1}{c}{2,766} & \multicolumn{1}{c}{2,766} & \multicolumn{1}{c}{2,766} \\ 
LLH & \multicolumn{1}{c}{1,169} & \multicolumn{1}{c}{1,556} & \multicolumn{1}{c}{1,573} \\ 
No. parameters & \multicolumn{1}{c}{15} & \multicolumn{1}{c}{16} & \multicolumn{1}{c}{22} \\ 
BIC & \multicolumn{1}{c}{-2,220} & \multicolumn{1}{c}{-2,985} & \multicolumn{1}{c}{-2,971} \\ 
Correlation $(Q)$ persistence $(\alpha+\beta)$ & \multicolumn{1}{c}{N/A} & \multicolumn{1}{c}{N/A} & \multicolumn{1}{c}{N/A} \\ 
\bottomrule \\[-1.8ex] 
\textit{Note:}  & \multicolumn{3}{c}{$^{*}$p$<$0.1; $^{**}$p$<$0.05; $^{***}$p$<$0.01} \\ 
\end{tabularx} 
\end{table} 


\subsection{Copula results with dynamic correlations}
%\label
% still haven't captured dynamic, lets do this now
% diagnostics of fit
% why we prefer symmetric t
% correlation patterns over time in-sample graph looks very much like the data rolling
% Table created by stargazer v.5.2 by Marek Hlavac, Harvard University. E-mail: hlavac at fas.harvard.edu
% Date and time: ons, okt 12, 2016 - 12:37:02
% Requires LaTeX packages: dcolumn 
\begin{table}[!htbp] \centering 
  \caption{Copula results: \textit{c}DCC specifications} 
  \label{tab:copula_estimates_dynamic} 
\begin{tabularx}{\textwidth}{X}
\\[-1.8ex]\toprule
\\[-1.8ex] 
\footnotesize Parameter estimates from dynamic copula models based on uniform residuals from ARMA-GARCH models. Stationary bootstrap standard errors in parentheses, following \textcite{PolitisRomano1994}. Copula parameters: $\nu$ is the degree of freedom, $\gamma$ is the vector of skewness parameters, $\alpha, \beta$ are the shock loading and autoregressive loading of the \textit{c}DCC process. All data 1963-07-05 - 2016-07-01. 
\end{tabularx}
\begin{tabularx}{\textwidth}{@{\extracolsep{5pt}} X D{.}{.}{-3} D{.}{.}{-3} D{.}{.}{-3} } 
\\[-1.8ex]\midrule
\\[-1.8ex] 
 & \multicolumn{3}{c}{Dynamic copula models} \\ 
\cline{2-4} 
\\[-1.8ex] & \multicolumn{1}{c}{(4)} & \multicolumn{1}{c}{(5)} & \multicolumn{1}{c}{(6)}\\ 
\\[-1.8ex] & \multicolumn{1}{c}{Gaussian} & \multicolumn{1}{c}{Student-\textit{t}} & \multicolumn{1}{c}{Skewed Student-\textit{t}}\\ 
\hline \\[-1.8ex] 
 $\nu$ &  & 11.936 & 11.881^{***} \\ 
  &  & () & (1.064) \\ 
  & & & \\ 
 $\gamma_{Mkt.RF}$ &  &  & -0.078 \\ 
  &  &  & (0.054) \\ 
  & & & \\ 
 $\gamma_{SMB}$ &  &  & -0.175^{**} \\ 
  &  &  & (0.077) \\ 
  & & & \\ 
 $\gamma_{Mom}$ &  &  & -0.145^{*} \\ 
  &  &  & (0.073) \\ 
  & & & \\ 
 $\gamma_{HML}$ &  &  & 0.083 \\ 
  &  &  & (0.058) \\ 
  & & & \\   
 $\gamma_{CMA}$ &  &  & 0.001 \\ 
  &  &  & (0.063) \\ 
  & & & \\ 
 $\gamma_{RMW}$ &  &  & 0.095 \\ 
  &  &  & (0.061) \\ 
  & & & \\ 
 $\alpha$ & 0.065 & 0.068 & 0.068^{***} \\ 
  & () & () & (0.007) \\ 
  & & & \\ 
 $\beta$ & 0.915 & 0.913 & 0.913^{***} \\ 
  & () & () & (0.011) \\ 
  & & & \\ 
\hline \\[-1.8ex] 
Observations & \multicolumn{1}{c}{2,766} & \multicolumn{1}{c}{2,766} & \multicolumn{1}{c}{2,766} \\ 
LLH & \multicolumn{1}{c}{2,791} & \multicolumn{1}{c}{2,978} & \multicolumn{1}{c}{2,989} \\ 
No. parameters & \multicolumn{1}{c}{17} & \multicolumn{1}{c}{18} & \multicolumn{1}{c}{24} \\ 
BIC & \multicolumn{1}{c}{-5,447} & \multicolumn{1}{c}{-5,813} & \multicolumn{1}{c}{-5,788} \\ 
Correlation $(Q)$ persistence $(\alpha+\beta)$ & \multicolumn{1}{c}{0.981} & \multicolumn{1}{c}{0.981} & \multicolumn{1}{c}{0.981} \\ 
\bottomrule \\[-1.8ex] 
\textit{Note:}  & \multicolumn{3}{c}{$^{*}$p$<$0.1; $^{**}$p$<$0.05; $^{***}$p$<$0.01} \\ 
\end{tabularx} 
\end{table} 
% potentially subsection exogenous regressors
%

\section{Putting the model to work}
%\label
% now, lets see what this model can say about a few things

\subsection{Mean-variance investing}
%\label
% fama french regressions, what actually happens to MV weights, do they go to zero?
% explain method for out-of-sample distributions, simulation based
% show that weights don't go to zero, not for sample and not for advanced super copula model
% what is the out-of-sample investment performance of relying on the best copula model?
% it looks pretty good but not drastically better than relying on sample average - OOS is hard

\subsection{Conditional diversification benefits}
%\label
% what are the theoretical diversificaiton benefits of including hml? is HML or CMA riskier if one can only choose either or? CDB analysis
