%!TEX root = ../main.tex
\section{Introduction}
\textcite{FF2015} find that the value factor, HML, is redundant in a four-factor model including profitability (RMW) and investment (CMA), as it has no explanatory power on monthly returns in a US 1963–2010 sample. This could mean that the classic value factor is an inferior proxy for what truly comprises value. This paper takes a risk perspective on factor equity strategies and investigates what role HML has in factor investing, given the discovery of RMW and CMA. In addition to the five factors, in \textcite{FF2015}, we also include momentum, as it is frequently used in factor investing.\footnote{See i.a. \textcite{Pedersen2015}, \textcite{Ilmanen2011}}

To study factor strategies, we need a conditional model for return series. In line with stylized facts on financial return series, we find evidence of autocorrelation and volatility clustering in the factor return series. We therefore specify models from the ARMA-GARCH family to capture these dynamics. The ARMA-GARCH models achieve white noise residuals for each factor, but we show that there is still substantial dependence between what the residual series. [Density plots, threshold correlations] More specifically, the series exhibit tail dependence, which can not be captured by standard models of multivariate dependence, such as the dynamic condtional correlation model \textcite{Engle2002} and BEKK [yes?]. Following \textcite{AngChen2002}, we study tail dependence using threshold correlation. Threshold correlation measures the linear correlation for a subset of two return series, when both series realize in the upper and lower tail respectively. For normally distributed variables, threshold corrleations tend to zero in the tails. For factor returns, threshold correlations are significantly positive in the tails, and also exhibit asymmetry above and below the median.

Recently, copula models have attracted considerable attention in the risk management field, as they offer a numerically feasible and flexible way of estimating joint probability distributions. More specifically, the copula model can be estimated with marginal models such as the ARMA-GARCH models as a starting point. The copula then uses the residuals from these models and tries to explain the remaining multivariate dependence in the return series. Furthermore, a copula is flexible enough to generate tail dependence and might improve on the description of return series.

Research questions?
\begin{enumerate}
	\item HML has been around for longer than RMW and CMA. Research points to the riskiness of crowded trades, and HML could potentially be more crowded as it has been known for a longer period. How does the portfolio risk contribution of HML compare to RMW and CMA?
	\item In \textcite{FF2015}, the alpha of HML is subsumed when RMW and CMA factors are added, and the authors conjecture that HML should not improve the mean-variance frontier as compared to the four factor model. What happens to mean-variance investing when HML is added?
	\item (I guess what we are missing or want to make more clear here is: what is the contribution of the copula model to understand factor strategies?)
\end{enumerate}

[The quant crash of 2007 highlighted the risks of crowded trades in factor strategies: During a short period between August 6 and August 9, equity hedge funds reported record losses in an otherwise calm market environment. Although losses were more frequent in quantitatively managed funds, traditional hedge funds also lost substantial amounts of money as the HML and SMB factors crashed in a liquidity spiral, fuelled by margin requirements of leveraged investors \autocite{KhandaniLo2011}. A diversified factor investor would however have been protected from some of the losses during the quant crash, as momentum had high returns.]

[The foundation for the hypotheses is the zero alpha of HML \autocite{FF2015} and the risk of crowded leveraged trades \autocite{Brunnermeier2009}. The intuition is the following: at times, factor strategies experience a negative return shock, which leads to margin calls for some leveraged investors. The margin calls lead to additional selling, for which there is not enough liquidity to maintain prices. In parallel, the return shock leads to tighter risk management, which in itself causes selling. The spiral continues as prices continue to drop and more investors reach risk limits and margin calls, until prices are pushed sufficiently far from equilibrium to attract other capital.]