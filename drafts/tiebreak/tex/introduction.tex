%!TEX root = ../main.tex
\section{Introduction}
\textcite{FF2015} find that the value factor, HML, is redundant in a four-factor model including profitability (RMW) and investment (CMA), as it has no explanatory power on monthly returns in a US 1963–2010 sample. This could mean that the classic value factor is an inferior proxy for what truly comprises value, and the paper has sparked a debate on whether the HML factor is merely a proxy for the value effect.

This paper takes a portfolio choice and risk perspective on factor equity strategies and investigates what role HML has in factor investing, given the discovery of CMA and RMW. We investigate whether HML should be included in factor portfolios, and whether HML provides different diversification benefits than the close alternative CMA. While most factor pairs exhibit relatively low correlation, the HML - CMA pair stands out with an unconditional correlation in our sample period of 0.63, indicating a much higher similarity between these two factors. In fact, there is also reason to expect a certain degree of overlap between the portfolios that comprise HML and CMA. Value firms invest less, and growth firms invest more, and there is a negative empirical relation between past investment and current book-to-market ratios (\textcite{Zhang2005}, \textcite{AndersonGarciaFeijoo2006}). 

The premium of the value factor has been explained both as a rational risk premium, that compensates the bearer for taking on some risk that materializes in bad times, and as a anomaly, that exists due to market frictions or investor irrationality. If the premia earned on factor strategies are compensations for risk, the source of risk in the HML and CMA factors might in fact be the same. A naive investor that then allocates assets to both these factors, might unwillingly double up on exposure to the same risk source. Similarly, if the premia earned are due to market frictions and irrational investor behavior, the naive investor would double up on exposure to the risk that the anomaly goes away. We believe that, regardless of the interpretation of the value premium, there is reason to place additional emphasis on the role of HML relative to CMA from a portfolio choice and risk perspective.

We begin by revisiting the regressions in \textcite{FF2015} that show a zero alpha to the HML factor in the five-factor model, as well as the regressions in \textcite{Asness2015} that include momentum in a six-factor model and resurrect the alpha of the HML factor. \textcite{FF2015} find zero alpha of adding HML to a four-factor portfolio, and conjecture that this implies that a mean-variance investor should in fact not improve the tangent portfolio by including HML. \textcite{Asness2015} challenge the notion that the value factor is subsumed by the addition of investment and profitability. In their study, they add a momentum factor, as well as an enhanced HML factor, and resurrect the alpha of the value factor, suggesting that HML is beneficial in a six-factor portfolio.

We find similar regression results to those in these papers in our weekly data set spanning 1963-2016. However, while the regressions of zero-cost portfolios in \textcite{FF2015} and \textcite{Asness2015} indicate which factors should be included in mean-variance investing, neither paper actually carries out mean-variance optimization of portfolios. This thesis fills this gap by optimizing weights of both five- and six-factor portfolios. Although this could be done on sample data alone, we aim to more thoroughly investigate the matter and complement the sample inputs with simulations from a conditional model of return series, that allows us to study optimal weights over time.

Having concluded the analysis of weights and the performance in mean-variance optimal portfolios, based on both sample and model inputs, we proceed with analyzing the relative diversification benefits among factor strategies. While mean-variance investing optimizes the expected return of a portfolio for a given level of variance, this analysis shifts the focus from the first two moments to the tail of the portfolio return distribution. We investigate a new measure of relative diversification benefit proposed by \textcite{ChristoffersenErrunzaJacobLanglois2012}: the \emph{conditional diversification benefit} (CDB) statistic, which measures how optimally chosen portfolio weights can make the expected shortfall of a factor portfolio closer to the portfolio's Value-at-Risk. Using simulations from the conditional model of returns, the CDB illustrates how well diversified portfolios are over time, and across different allowed asset universes. Specifically, we interest ourselves with the diversification benefits of adding either HML or CMA to a asset universe of the remaining factor strategies.

For both the mean-variance and diversification benefit analysis, the choice of conditional return model is central. While the ARMA-GARCH family of models are the norm of univariate time-series modeling, multivariate modeling has proven harder as the multivariate extensions of such models are often computationally infeasible and ridden with dimensionality problems. Recently, however, copula models have attracted considerable attention in the risk management field, as they offer a numerically stable and flexible way of estimating joint probability distributions. 

Following closely the method of \textcite{ChristoffersenLanglois2013}, we build a copula model of the joint factor returns. The specification we use is designed to recognize time-varying correlations and tail dependence, which is the notion that there might be significantly different dependence patterns when returns simultaneously realize in the lower or upper tail, as opposed to close to the center of the distribution. This becomes especially important when we consider the conditional diversification benefits, as it concerns the tail of the return distribution. The copula model is shown to be capable to be generate the correlation patterns in the data. It can, however, only to a limited extent, explain the tail dependence in the data. As another robustness check of the copula model, we repeat the mean-variance analysis out-of-sample on the last one third of the data set. We re-estimate the model and compare the weights and performance of mean-variance investing using sample and model inputs.

Tease a little results
 
The structure of this thesis is as follows: in \autoref{sec:literature} we provide a brief literature review of factor strategies and the work on copulas relating to factors. In \autoref{sec:data} we present the data used and detail the construction of factors. In \autoref{sec:alpha_reg}, we revisit the abnormal return regressions of \textcite{FF2015} and \textcite{Asness2015}. In \autoref{sec:modeling_of_factor_returns}, we present our model of factor returns, interleaing method and results. In \autoref{sec:mean_variance} we present analysis of mean-variance optimizations. In \autoref{sec:conditional_diversification_benefit}, we examine the diversification benefits of different factor portfolios. \autoref{sec:discussion_conclusion} summarizes our findings and discusses their wider implications.
