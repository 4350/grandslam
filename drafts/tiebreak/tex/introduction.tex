%!TEX root = ../main.tex
\section{Introduction}
\textcite{FF2015} find that the value factor, HML, is redundant in a four-factor model including profitability (RMW) and investment (CMA), as it has no explanatory power on monthly returns in a US 1963–2010 sample. This could mean that the classic value factor is an inferior proxy for what truly comprises value. This paper takes a risk perspective on factor equity strategies and investigates what role HML has in factor investing, given the discovery of RMW and CMA. 

While most factor pairs exhibit relatively low correlation, the HML - CMA pair stands out with an unconditional correlation in our sample period of 0.63, indicating a much higher similarity between these two factors. In fact, there is also reason to expect a certain degree of overlap between the portfolios that comprise HML and CMA. Value firms invest less, and growth firms invest more, and there is a negative empirical relation between past investment and current book-to-market ratios (\textcite{Zhang2005}, \textcite{AndersonGarciaFeijoo2006}). If the premia earned on factor strategies are compensations for risk, the source of risk in the HML and CMA factors might in fact be the same. Allocating assets to both these factors might then increase the exposure to that risk source. We therefore place additional emphasis on the analysis of HML's role relative to CMA.

While analyses could be made on sample data alone, we aim to investigate factors more thoroughly by building a conditional model of return series, that can incorporate dependencies that are hard to capture in sample data, such as time-varying correlations and tail dependence. Tail dependence is the notion that there might be significantly different dependence patterns when returns simultaneously realize in the lower or upper tail, as opposed to close to the center of the distribution. From previous research on factors~\autocite{ChristoffersenLanglois2013}, we have reason to believe that the factor strategies exhibit both tail dependence and dynamic correlations, and show that this is the case also for CMA and RMW.

The choice of model for the joint returns is not easy, as there exist many competing frameworks. While the ARMA-GARCH family of models are the norm of univariate modeling, the multivariate extensions of such models are often computationally infeasible and ridden with dimensionality problems. Recently, however, copula models have attracted considerable attention in the risk management field, as they offer a numerically feasible and flexible way of estimating joint probability distributions. More specifically, a copula model can be estimated with univariate models as a starting point. The main advantage of using a copula as opposed to other multivariate models is that it is both computationally stable and flexible enough to generate tail dependence as well as dynamic correlations, and might improve on the description of return series. 

First, we examine the factor return series on a univariate level and find evidence of autocorrelation and volatility clustering, in line with stylized facts on financial return series. We specify different univariate models from the ARMA-GARCH family, to pinpoint the unique dynamics of each factor strategy. Our univariate models achieve white noise residuals for each factor, but we show that there is still substantial dependence between the residual series. The copula then uses the residual series from these models and tries to explain the interdependence.

Armed with the copula model, we turn back to the core idea of evaluating the role of HML in factor investing. We revisit the regressions of \textcite{FF2015} that show a zero alpha of including HML in a four-factor portfolio of Mkt-RF, SMB, RMW and CMA, but replace monthly data with weekly data and include the momentum factor. Weekly data gives a more granular understanding of risk, as shocks are more smoothed on a monthly basis. Momentum is also included, as it is frequently used in factor equity investing.\footnote{See i.a. \textcite{Pedersen2015}, \textcite{Ilmanen2011}} 

We then test the conjecture in \textcite{FF2015} that the insignificant alpha of adding HML to a four-factor portfolio implies that a mean-variance investor should in fact not include HML in her portfolio. We save one third of the data for out-of-sample analysis, re-estimate our model and construct mean-variance optimal portfolios. As the copula model gives weekly conditional forecasts, we can study optimal weights of different factors, including HML, over time.

In the mean-variance analysis, we also calculate the Sharpe ratios of portfolios including and excluding HML and CMA, respectively. In addition, we quantify higher moment risk measures of competing portfolios, including skewness and maximum drawdown. This out-of-sample exercise allows us to compare the relative risk-reward profiles of HML and CMA as alternative portfolio additions. Simultaneously, the analysis investigates the risk-reward profile of different copula specifications. Does investing based on conditional and asymmetric models improve the Sharpe ratio? And what happens to other risk measures?

Finally, we study the diversification benefits of HML, RMW and CMA. Based on simulations from our copula model of joint returns, we investigate a new measure of relative diversification benefit proposed by \textcite{ChristoffersenErrunzaJacobLanglois2012}. This \emph{conditional diversification benefit} (CDB) statistic measures how close the expected shortfall of a factor portfolio can come to its theoretical optimum, which is the same portfolio's Value-at-Risk, when weights are chosen optimally. This analysis can, given a certain model, shed light on the relative diversification benefit of including one of the similar factors HML and CMA.

Tease a little results
 
The structure of this thesis is as follows: in \autoref{sec:literature} we provide a brief literature review of factor strategies and the work on copulas relating to factors. In \autoref{sec:data} we present the data used and detail the construction of factors. In \autoref{sec:univariate_modeling} through \autoref{sec:model_work}  we present the main empirical analysis by interleaving method and results. In \autoref{sec:discussion_conclusion} we summarize our findings and discuss their wider implications.