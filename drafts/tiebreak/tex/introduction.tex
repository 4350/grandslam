%!TEX root = ../main.tex
\section{Introduction}
\textcite{FF2015} find that the value factor, HML, is redundant in a four-factor model including profitability (RMW) and investment (CMA), as it has no explanatory power on monthly returns in a US 1963–2010 sample. This could mean that the classic value factor is an inferior proxy for what truly comprises value, and the paper has sparked a debate on whether the HML factor is merely a proxy for the value effect. \textcite{Asness2015} challenge the notion that the value factor is subsumed by the addition of investment and profitability. In their study, they add a momentum factor, as well as an enhanced HML factor, and resurrect the alpha of the value factor.

This paper takes a risk perspective on factor equity strategies and investigates what role HML has in factor investing, given the discovery of CMA and RMW.

While most factor pairs exhibit relatively low correlation, the HML - CMA pair stands out with an unconditional correlation in our sample period of 0.63, indicating a much higher similarity between these two factors. In fact, there is also reason to expect a certain degree of overlap between the portfolios that comprise HML and CMA. Value firms invest less, and growth firms invest more, and there is a negative empirical relation between past investment and current book-to-market ratios (\textcite{Zhang2005}, \textcite{AndersonGarciaFeijoo2006}). 

The premium of the value factor has been explained both as a rational risk premium, that compensates the bearer for taking on some risk that materializes in bad times, and as a anomaly, that exists due to market frictions or investor irrationality. If the premia earned on factor strategies are compensations for risk, the source of risk in the HML and CMA factors might in fact be the same. A naive investor that then allocates assets to both these factors, might unwillingly double up on exposure to the same risk source. Similarly, if the premia earned are due to market frictions and irrational investor behavior, the naive investor would double up on exposure to the risk that the anomaly goes away. We believe that, regardless of the interpretation of the value premium, there is reason to place additional emphasis on the role of HML relative to CMA from a portfolio management and risk management perspective.

We begin by revisiting the regressions in \textcite{FF2015} that show a zero alpha to the HML factor in the five-factor model, as well as the regressions in \textcite{Asness2015} that include momentum in a six-factor model and resurrect the alpha of the HML factor. We find similar results to those in these papers in our weekly data set spanning 1963-2016. 

We then proceed by testing the conjecture in \textcite{FF2015} that the insignificant alpha of adding HML to a four-factor portfolio implies that a mean-variance investor should in fact not include HML in her portfolio. 

Although this analysis could be made on sample data alone, we also aim to investigate factors more thoroughly by building a conditional model of return series, that can incorporate dependencies that are hard to capture in sample data, such as time-varying correlations and tail dependence. Tail dependence is the notion that there might be significantly different dependence patterns when returns simultaneously realize in the lower or upper tail, as opposed to close to the center of the distribution. From previous research on factors~\autocite{ChristoffersenLanglois2013}, we have reason to believe that the factor strategies exhibit both tail dependence and dynamic correlations, and show that this is the case also for CMA and RMW.

While the ARMA-GARCH family of models are the norm of univariate time-series modeling, multivariate modeling has proven harder as the multivariate extensions of such models are often computationally infeasible and ridden with dimensionality problems. Recently, however, copula models have attracted considerable attention in the risk management field, as they offer a numerically feasible and flexible way of estimating joint probability distributions, with univariate models such as the ARMA-GARCH as a starting point. The main advantage of using a copula as opposed to other multivariate models is that it is both computationally stable and flexible enough to generate the tail dependence as well as the dynamic correlations in the data, and might improve on the description of return series. Following closely the method of \textcite{ChristoffersenLanglois2013}, we build a copula model of the joint factor returns.

%First, we examine the factor return series on a univariate level and find evidence of autocorrelation and volatility clustering, in line with stylized facts on financial return series. We specify different univariate models from the ARMA-GARCH family, to pinpoint the unique dynamics of each factor strategy. Our univariate models achieve white noise residuals for each factor, but we show that there is still substantial dependence between the residual series. The copula then uses the residual series from these models and tries to explain the interdependence. 

Both sample estimates and copula estimates are used in the mean-variance analysis, which is carried out out-of-sample, on approximately one third of the available data. As the copula model gives weekly conditional forecasts, we are now able to study optimal weights of different factors, including HML, over time. 

In the mean-variance analysis, we also optimize five-factor and six-factor portfolios that exclude either HML or CMA. This exercise allows us to compare the relative risk-reward profiles of HML and CMA as alternative portfolio additions to a portfolio comprising the other four (or five, including momentum) factors, not only based on the Sharpe ratio but also based on the realized skewness and maximum drawdown. 

Having concluded the mean-variance optimization, we finally study the diversification benefits of HML, RMW and CMA in the pure model setting. Based on simulations from our copula model of joint returns, we investigate a new measure of relative diversification benefit proposed by \textcite{ChristoffersenErrunzaJacobLanglois2012}. This \emph{conditional diversification benefit} (CDB) statistic measures how close the expected shortfall of a factor portfolio can come to its theoretical optimum, which is the same portfolio's Value-at-Risk, when weights are chosen optimally. This analysis can, given a certain model, shed light on the relative diversification benefit of including one of the similar factors HML and CMA.

Tease a little results
 
The structure of this thesis is as follows: in \autoref{sec:literature} we provide a brief literature review of factor strategies and the work on copulas relating to factors. In \autoref{sec:data} we present the data used and detail the construction of factors. In \autoref{sec:alpha_reg}, we revisit the abnormal return regressions of \textcite{FF2015} and \textcite{Asness2015}. In \autoref{sec:intro_copula}, we introduce and motivate the use of a copula model. In \autoref{sec:univariate_modeling} through \autoref{sec:dac_copula} we present the copula model interleaving method and results. In \autoref{sec:model_work} we present the main analysis of mean-variance optimizations and diversification benefits. \autoref{sec:discussion_conclusion} summarizes our findings and discusses their wider implications.