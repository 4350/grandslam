%!TEX root = ../main.tex

\section{Modeling of Factor Returns} % (fold)
\label{sec:modeling_of_factor_returns}

%!TEX root = ../main.tex

\subsection{Definition of copula model} % (fold)
\label{sub:definition_of_copula_model}

Each week $t$, the conditional joint density of returns $R_{t+1} = \{r_{i,t+1},\ldots,r_{N,t+1}\}$ is described by a joint density function $f_t(R_{t+1})$. Following~\textcite{ChristoffersenErrunzaJacobLanglois2012}, who build on~\textcite{Patton2006} and~\textcite{Sklar1959}, we decompose the joint density function into the product of a joint copula function $c_t(U_{t+1})$ of uniformly distributed variables $U_{t+1} \sim U(0, 1)$ and marginal densities $f_{i,t}(r_{i,t+1})$:
\begin{align}
  f_t(R_{t+1}) =
    c_t(U_{t+1}) \prod^N_{i=1} f_{i,t}(r_{i,t+1})
  \label{eq:copula_sklar}
\end{align}
The elements of $U_{t+1} = \{u_{i,t+1},\ldots,u_{N,t+1}\}$ are related to the original returns by the probability integral transform, i.e the cumulative distribution of $r_{i,t+1}$:
\begin{align}
  u_{i,t+1} = F_{i,t}(r_{i,t+1}) = \int_{-\infty}^{r_{i,t+1}} f_{i,t}(r)dr
\end{align}
The copula function $c_t(U_{t+1})$ is a multivariate skewed Student's \emph{t} distribution. This distribution is parametrised by a single degrees of freedom parameter $\nu_c$, controlling the degree of dependency, a vector of $N$ skewness parameters $\gamma_c$, controlling the asymmetry in dependency, and a potentially time-varying correlation matrix $\Psi_{t}$. We describe the details of the skewed Student's t distribution, including the expanded form of $c_t$, in~\autoref{app:ghstmv}. The skewed Student's \emph{t} distribution nests the Student's \emph{t} distribution when all $\gamma_{i,c} = 0$ and the standard normal distribution when additionally $\nu_c = \infty$.

The log-likelihood of the model is constructed from~\autoref{eq:copula_sklar}:
\begin{align}
  L =
    \underbrace{\sum_{t=1}^T \log(c_t(U_{t+1}))}_\text{Copula} +
    \underbrace{\sum_{t=1}^T \sum_{i=1}^N \log(f_{i,t}(r_{i,t+1}))}_\text{Marginals}
\end{align}
At this point, it is worth noting that the joint density $c_t(U_{t+1})$ need not be of the same family as the marginal densities $f_{i,t}(r_{i,t+1})$ -- nor are we restricted to modeling $f_{i,t}(r_{i,t+1})$ jointly for all factors. In fact, we take advantage of this flexibility and choose to model the marginal densities independently as ARMA-GARCH processes, which allows us to capture a number of predictable features in the univariate series -- serial correlation, volatility clustering and leverage effects. The marginal models are estimated independently by maximizing the likelihood(s) of the second term, and then the copula is estimated by maximizing the first -- using the residuals of the marginal models.

This procedure is called multi-stage maximum log-likelihood or inference functions for margins and greatly simplifies the estimation procedure while yielding relatively efficient estimates~\autocite{Patton2006,Joe1997}. The modeling and estimation of our ARMA-GARCH models is detailed in the upcoming subsection, whereas the remainder of this subsection describes how we make $\Psi_t$ and thus the dependence between factors dynamic.

The copula is made dynamic by fitting a dynamic conditional correlation (DCC) process for $\Psi_t$ to copula residuals $z_{t+1}^*$~\autocite{Engle2002}. Using the notation from~\textcite{ChristoffersenLanglois2013}:
\begin{align}
  Q_t = (1 - \alpha - \beta) Q
    + \beta Q_{t-1}
    + \alpha \bar{z}_{t-1}^* \bar{z}_{t-1}^{*\top}
  \label{eq:copula_cdcc}
\end{align}
where $Q_t$ is normalized to the correlation matrix $\Psi_t$:
\begin{align}
  \Psi_t = Q_t^{-\frac{1}{2}} Q_t Q_t^{-\frac{1}{2}}
  \label{eq:copula_cdcc_psi}
\end{align}
The $Q_t$ process is comprised of three components that are weighted according to $\alpha, \beta$: (1) a time-invariant component, $Q$ (2) an innovation component from copula shocks, $\bar{z}_{t-1}^{*} \bar{z}_{t-1}^{*\top},$\footnote{Where $\bar{z}_{i,t+1}^* = z_{i,t+1}^* \sqrt{q_{ii,t}}$ is due to a correction by~\textcite{Aielli2013}, that improves the reliability of the estimation procedure.} and (3) an autoregressive component of order one $Q_{t-1}$. In order for the the correlation matrix $\Psi_t$ to be positive definite, $Q_t$ has to be positive definite, which is ascertained by requiring that $\alpha \geq 0$, $\beta \geq 0$ and $(\alpha + \beta) < 1$. The model nests a constant copula when $\alpha = \beta = 0$.

The model for $c_t(U_{t+1})$ is thus comprised of $1 + N$ distribution parameters $\{\nu_c, \gamma_c\}$ and ($2 + (N(N-1) / 2)$) dynamics parameters $\{\alpha, \beta, Q\}$ where the elements of $Q$ are estimated using moment matching, and the remaining $3 + N$ parameters are estimated using maximum likelihood.\footnote{We have relegated the details of the estimation of the dynamic process to~\autoref{app:copula_cdcc}.}

% Gustaf: Här är det svårt -- jag har svårt att beskriva $z_t^*$ när konceptet
% standardized returns är en del av ARMA-GARCH delen. Vet ej riktigt hur man
% ska komma åt det. Ett sätt är att ta tjuren vid hornen och beskriva att 
% f_{i,t}(r_{i,t+1}) = f_i(z_{i,t+1}) = 
% f_i(\varepsilon_{i,t+1}/\sigma_{i,t+1}). men ja.
ARMA-GARCH modeling allows us to filter time-varying effects, leaving independent \emph{standardized returns} (or standardized residuals) $z_{i,t}$ assumed to follow a constant distribution $f_i(z_{i,t})$. These residuals are first transformed into uniform variables $u_{i,t+1}$ by the probability integral transform of the densities above, and then made to follow the \emph{copula} distribution by the \emph{inverse} probability integral transform of the \emph{copula}:
\begin{align}
  z_{i,t+1}^* = F^{-1}_{\nu_c,\gamma_{i,c}}(F_{i}(z_{i,t+1}))
\end{align}



% subsection definition_of_copula_model (end)

%!TEX root = ../main.tex

\subsection{Univariate models} % (fold)
\label{sub:univariate_models}

ARMA-GARCH processes are used to model the marginal densities $f_i(r_{i,t+1})$
by constructing equations for the conditional expectation of returns $\mathbb{E}_t[r_{i,t+1}]$ and volatilies $\sigma_{i,t}$ each week. This results in \emph{standardized returns} $z_{i,t}$ which are independent and assumed to follow a constant density so that:
\begin{align}
  f_{i,t+1}(r_{i,t+1}) = f_i(z_{i,t+1}) = f_i(\frac{r_{i,t+1} - \mathbb{E}_t[r_{i,t+1}]}{\sigma_{i,t}})
\end{align}
The standardized returns.

% Gustaf: Här är det svårt -- jag har svårt att beskriva $z_t^*$ när konceptet
% standardized returns är en del av ARMA-GARCH delen. Vet ej riktigt hur man
% ska komma åt det. Ett sätt är att ta tjuren vid hornen och beskriva att 
% f_{i,t}(r_{i,t+1}) = f_i(z_{i,t+1}) = 
% f_i(\varepsilon_{i,t+1}/\sigma_{i,t+1}). men ja.
ARMA-GARCH modeling allows us to filter time-varying effects, leaving independent \emph{standardized returns} (or standardized residuals) $z_{i,t}$ assumed to follow a constant distribution $f_i(z_{i,t})$. These residuals are first transformed into uniform variables $u_{i,t+1}$ by the probability integral transform of the densities above, and then made to follow the \emph{copula} distribution by the \emph{inverse} probability integral transform of the \emph{copula}:
\begin{align}
  z_{i,t+1}^* = F^{-1}_{\nu_c,\gamma_{i,c}}(F_{i}(z_{i,t+1}))
\end{align}

% subsection univariate_models (end)


% Why a model?
This section presents our model for the joint behavior of returns. A multivariate model of returns allows us to make conditional forecasts of the distribution of returns one week ahead, which take into account the dependence between factors. The model is used in the mean-variance analysis, to provide dynamic inputs that can give us optimal weights over time. The model is also used in the analysis of diversification benefits, when we shift the focus to the tail risk of factor portfolios. First, we describe why we choose the copula model among different multivariate models. Second, we describe the building blocks of the model and analyze which specification best suits the dependence structure in the data. Last, we provide estimates and describe the model we retain for further use.

\subsection{Choosing a multivariate model}
% Why copula?
The ARMA-GARCH family of models has become the norm of modeling univariate financial return series, beginning with \textcite{Bollerslev1986}. The straightforward extension of univariate GARCH models to multiple return series has, however, proven difficult. Unrestricted multivariate GARCH (MGARCH) models that model the conditional covariance matrix directly become impossible to estimate as the number of covariances grows exponentially with the number of series. It thus becomes necessary to restrict the parameter space, where the BEKK model is a common example~\autocite{BEKKModel}.

% Separate modeling of variance and correlation
A parsimonious solution to the dimensionality problem is to separate the modeling of return and volatility dynamics from the modeling of conditional correlations. The separation allows for consistent (albeit inefficient) two-step estimation, and makes large-scale estimation feasible. One such approach is the \emph{dynamic conditional correlation} (DCC) model originally proposed by~\autocite{Engle2002}. In the DCC model, univariate GARCH models are first estimated on each series. Then, an autoregressive process for the correlation matrix is fitted to the standardized residuals ${z_t}$ of those models. 

% Asymmetric cDCC and copula cDCC
DCC is a useful and tractable model for estimating time-varying correlations between return series. However, it is a model of correlations only and is not flexible enough to model the univariate components differently. More specifically, it is not constructed to generate tail dependence, which is the notion that correlation dynamics can be very different in extreme realizations. 

% Enter the copula
Copula models have recently attracted much attention in the field of risk management, as they provide a flexible way to infer a multivariate probability distribution. Copula models are, just like DCC models, based on two-step estimation and work well in large scale applications. Furthermore, copulae are flexible enough to generate tail dependence, which is shown to be an important feature of factor return data~\autocite{ChristoffersenLanglois2013}. 

Copula models are most often constructed by estimating univariate models from the ARMA-GARCH family in the first step. The residuals from the ARMA-GARCH models are then used in the copula function that explains the multivariate dependence, including dynamic correlations and tail dependence.

We choose to work with a copula model, as it can (1) estimate the joint distribution function in large scale applications, (2) model different univariate models for the different factors, and (3) incorporate both tail dependence and dynamic correlations. Next, we define and describe the copula model.

\subsection{Definition of the copula model}
This is section 05_04, but adapted so that ARMA-GARCH is explained when needed.
\subsection{Univariate models}
This describes the selection and estimation of the univariate models.
\subsection{Analysis and choice of dependence structure}
This is the work on residuals from the univariate models, and is what makes us believe that a certain dependence structure is right.
\subsection{Copula estimation results}
Here the main tables that show which model to prefer, discuss parameter estimates.
\subsection{Robustness check of copula}
Here simulated threshold correlations?
\subsection{Model retained for further use}
Here describe again the model we use in MV and CDB. E.g. dynamic std. If the reader doesn't follow anything of the first four subsections, she should at least be able to pick up what we use here.
%
%!TEX root = ../main.tex

\subsection{Copula Modeling of Dependency} % (fold)
\label{sub:05_04_copula}

\subsubsection{Copula Model Specification}

We now turn to modeling the dependency between factors using a copula model. 
Each week, the joint behavior of returns $R_{t+1}$ in the next week is modeled by a joint density function $f_t(R_{t+1})$. Assuming returns are multivariate normally distributed, this would correspond to the multivariate normal density function, fully parametrised by means, volatility and correlation matrix. But neither factor returns $R_t$, nor the standardized residuals $z_t$ are normally distributed.

Copulas are a convenient way of modeling dependency between non-normal returns. Following~\textcite{ChristoffersenErrunzaJacobLanglois2012}, who builds on~\textcite{Patton2006} and~\textcite{Sklar1959}, we decompose the joint density function into the product of a joint copula function $c_t$ of uniform variables $U_{t+1}$ and the marginal univariate distributions $f_{i,t}(r_{i, t+1})$:
\begin{align}
  f_t(R_{t+1}) &=
    c_t(U_{t+1}) \prod^N_{i = 1} f_{i,t}(r_{i, t + 1})
\end{align}
The marginal densities, $f_{i,t}(r_{i, t + 1})$, are modeled by ARMA-GARCH processes while the copula $c_t$ is a model of the joint behavior of their probability integral transforms. $c_t$ can be modeled, and estimated, separately from the marginal densities. This is key to imposing a more sophisticated dependency structure on the returns.

After ARMA-GARCH filtering, the marginal densities of returns are modeled by the constant density functions of standardized residuals. The vector of uniforms are therefore related to returns by the probability integral transform (PIT) of standardized residuals:
\begin{align}
  u_{i, t+1} = \int_{-\infty}^{z_{i,t+1}} f_{i}(z_{i,t+1})
\end{align}
In the most general case, we use a multivariate skewed Student's t \emph{dynamic asymmetric copula} model introduced by~\textcite{ChristoffersenErrunzaJacobLanglois2012}. The joint distribution is parametrised by a single degrees of freedom parameter $\nu_c$, an $N$ vector of skewness parameters $\gamma_{c}$ and a (time-varying) correlation matrix $\Psi_{t}$. We describe the details of the copula distribution in~\autoref{app:ghstmv}.

The normal and Student's t copula are nested in this model, as when $\gamma_{c,i} = 0$, we obtain a multivariate Student's t distribution, and if additionally $\nu_c = \infty$, we obtain the multivariate standard normal distribution.

The copula is made dynamic by evolving the correlation matrix $\Psi_t$ according to an underlying \emph{c}DCC process $Q_t$~\autocites[cf.]{Engle2002,Aielli2013}. Using the notation from~\textcite{ChristoffersenLanglois2013}:\footnote{The difference between ${z_t^*}$ and ${\bar{z}_t^*}$ is due to the \emph{corrected} DCC model; details in the appendix.}
\begin{align}
  Q_t &= (1 - \alpha - \beta) Q
    + \beta Q_{t - 1}
    + \alpha \bar{z}_{t - 1}^* \bar{z}_{t - 1}^{*\top}
  \label{eq:copula_cdcc}
\end{align}
where $Q_t$ is normalized to the correlation matrix
\begin{align}
  \Psi_t = Q_t^{-1/2} Q_t Q_t^{-1/2}
  \label{eq:copula_cdcc_psi}
\end{align}
The $Q_t$ process is comprised of three components that are weighted according to $\alpha, \beta$: (1) a time-invariant component $Q$, (2) an innovation component from copula shocks $\bar{z}_{t-1}^{*} \bar{z}_{t-1}^{*\top}$ and (3) an autoregressive component of order one $Q_{t-1}$. In order for the the correlation matrix $\Psi_t$ to be positive definite, $Q_t$ has to be positive definite, which is ascertained by requiring that $\alpha \geq 0$, $\beta \geq 0$ and $(\alpha + \beta) < 1$. The model nests a constant copula by forcing $\alpha = \beta = 0$.

% XXX NOTATION
% XXX cDCC correction for copula shocks
The parameters of the copula model -- the distribution parameters of the multivariate skewed Student's t distribution and the dynamics parameters of the \emph{c}DCC -- are estimated by maximizing the log-likelihood:
\begin{align}
  \arg\!\max_{\nu_c, \gamma_{ic}, \alpha, \beta} \sum_{t = 1}^T \ln c_t(U_t; \nu_c, \gamma_{ic}, \alpha, \beta)
\end{align}
This estimation takes the uniform residuals from each GARCH model as given. The process of copula estimation with \emph{c}DCC dynamics is quite involved. A detailed description can be found~\autoref{app:copula_cdcc}.

Clearly, the interpretation of the copula parameters is closely associated to the structure of multivariate dependence. By different restrictions on the parameters in the DAC model, we are able to activate or deactivate certain features of the copula: First, the degree of freedom parameter $nu_c$ is to be interpreted as a the measure of tail dependency. When $nu \neq 0$, the lower and upper tails of the joint distribution are fatter than in the normal case, which is coherent with the evidence from threshold correlations in. Second, the skewness parameters $\gamma_{c,i}$ are to be interpreted as the extent of asymmetry in the correlation structure. When $\gamma \neq 0$, there is asymmetry in correlations, which is also coherent with the earlier threshold correlation analysis. Third, the $\alpha$ and $\beta$ parameters determine whether the copula generates time-varying correlations. If $\alpha \neq 0$ and $\beta \neq 0$, the copula is dynamic, which is consistent with the findings of the rolling correlation analysis

\subsubsection{Copula Estimation Results}

We estimate constant and dynamic normal, symmetric and asymmetric copula models on the full dataset of GARCH uniform residuals; results are in~\autoref{tab:copula_estimation}. Looking at the parameter estimates, few of the $\gamma_c$ estimates appear significant, however, $\nu_c$ is clearly not infinite. Additionally, there is little improvement in log-likelihood by going from a symmetric to asymmetric copula. We conclude that the symmetric Student's t copula model is preferred. The insignificance of $\gamma$ can be interpreted as evidence of low asymmetries in the dependency -- or, more likely, that the model is simply unable to capture it.

There is a significant improvement in log-likelihood by going from a constant to dynamic copula, which suggests that time-varying tail dependency is an important feature to capture. The persistence, $\alpha + \beta$ is close to one, which could suggest that the dependency structure of factors is not stationary. We now turn to investigating how well the copula reproduces the dependency patterns observed in the data.

%!TEX root = ../../main.tex

\begin{table}[p]
  \centering
  \footnotesize
  \renewcommand{\arraystretch}{1.2}

  \caption{Copula parameter estimates (1963--2016)}

  \begin{longcaption}
    Models from~\autoref{eq:copula_cdcc} on $N = 2,766$ weekly standardized residuals from the ARMA-GARCH models in~\autoref{tab:garch_estimation}. Stationary bootstrap standard errors in parentheses, following~\autocite{PolitisRomano1994}. $nu_c$ is the degree of freedom parameter and $gamma_i$ are the copula skewness parameters. $\alpha, \beta$ control the correlation dynamics, and Persistence is $\alpha + \beta$. Elements of $\hat{Q}$ are estimated using moment matching (see~\autoref{app:copula_cdcc}). The constant copula is achieved by forcing $\alpha = \beta = 0$. For each model, there are 15 parameters in the $Q$ time-invariant correlation matrix, which are not reported in the table. Significance given by $^{*}p<10\%$; $^{**}p<5\%$; $^{***}p<1\%$
  \end{longcaption}

  \begin{tabularx}{\textwidth}{@{}l ddd X ddd}
    \toprule
    &
      \multicolumn{3}{c}{Constant Copula} &&
      \multicolumn{3}{c}{Dymamic Copula} \\
    \cmidrule{2-4} \cmidrule{6-8}
    &
      \multicolumn{1}{c}{Normal} & \multicolumn{1}{c}{Symmetric \emph{t}} & \multicolumn{1}{c}{Skewed \emph{t}} & &
      \multicolumn{1}{c}{Normal} & \multicolumn{1}{c}{Symmetric \emph{t}} & \multicolumn{1}{c}{Skewed \emph{t}} \\
    \midrule
    $\nu_c$                & & 6.61^{***} & 6.65^{***}  & &             & 11.77^{***} & 11.63^{***} \\
                          & & (0.89)     & (0.10)      & &             & (0.89)      & (0.10) \\
    % $1/\nu_c$             & & 0.15^{***} & 0.15^{***}  & &             & 0.09^{***} & 0.09^{***} \\
    %                       & & (0.01)     & (0.01)      & &             & (0.01)      & (0.01) \\
    \\
    $\gamma_\text{Mkt}$ & &             & -0.06       & &             &              & -0.05 \\
                        & &             & (0.07)      & &             &              & (0.05) \\
    \\
    $\gamma_\text{SMB}$ & &             & -0.11^{*}   & &             &              & -0.14^{**} \\
                        & &             & (0.06)      & &             &              & (0.06) \\
    \\
    $\gamma_\text{Mom}$ & &             & -0.20^{***} & &             &              & -0.12 \\
                        & &             & (0.07)      & &             &              & (0.07) \\
    \\
    $\gamma_\text{HML}$ & &             & 0.10        & &             &              & -0.02 \\
                        & &             & (0.07)      & &             &              & (0.06) \\
    \\
    $\gamma_\text{CMA}$ & &             & 0.08        & &             &              & -0.05 \\
                        & &             & (0.06)      & &             &              & (0.07) \\
    \\
    $\gamma_\text{RMW}$ & &             & 0.02        & &             &              & 0.18^{**} \\
                        & &             & (0.07)      & &             &              & (0.07) \\
    \\
    
    $\alpha$            & &             &              & & 0.07^{***} & 0.07^{***}  & 0.07^{***} \\
                        & &             &              & & (0.01)     & (0.01)      & (0.01) \\
    \\
    $\beta$             & &             &              & & 0.91^{***} & 0.91^{***}  & 0.91^{***} \\
                        & &             &              & & (0.01)     & (0.01)      & (0.01) \\
    \midrule
    \multicolumn{8}{@{}l}{\textbf{Log-likelihood (LLH), Number of parameters (\# params.), BIC and Correlation Persistence (CP)}} \\
    LLH &
      \multicolumn{1}{D{.}{.}{3}}{1,169} & 
      \multicolumn{1}{D{.}{.}{3}}{1,555} & 
      \multicolumn{1}{D{.}{.}{3}}{1,573} & &
      \multicolumn{1}{D{.}{.}{3}}{2,790} & 
      \multicolumn{1}{D{.}{.}{3}}{2,983} & 
      \multicolumn{1}{D{.}{.}{3}}{2,995} \\
    \# params. &
      \multicolumn{1}{D{.}{.}{3}}{15} & 
      \multicolumn{1}{D{.}{.}{3}}{16} & 
      \multicolumn{1}{D{.}{.}{3}}{22} & &
      \multicolumn{1}{D{.}{.}{3}}{17} & 
      \multicolumn{1}{D{.}{.}{3}}{18} & 
      \multicolumn{1}{D{.}{.}{3}}{24} \\
    BIC &
      \multicolumn{1}{D{.}{.}{3}}{-2,337} & 
      \multicolumn{1}{D{.}{.}{3}}{-3,103} & 
      \multicolumn{1}{D{.}{.}{3}}{-3,090} & &
      \multicolumn{1}{D{.}{.}{3}}{-5,564} & 
      \multicolumn{1}{D{.}{.}{3}}{-5,943} & 
      \multicolumn{1}{D{.}{.}{3}}{-5,919} \\
    CP (\%) & & & && 97.73 & 98.01 & 97.98 \\
    \bottomrule
  \end{tabularx}

  \label{tab:copula_estimation}
\end{table}



% subsection copula_model (end)

\input{tex/05_02_univariate}
\input{tex/05_03_residuals}
%!TEX root = ../main.tex
\subsection{Copula specification and estimation results}
Given the results of the dependence structure of residuals, we now discuss the best choice of copula model and present estimation results of the six competing copula specifications.

\subsubsection{Interpreting and choosing copula specification}

The interpretation of the copula parameterization is closely associated to the structure of multivariate dependence. By different restrictions on the parameters in the DAC model, we are able to activate or deactivate certain features of the copula: First, the degree of freedom parameter $nu_c$ is to be interpreted as a the measure of tail dependency. When $nu \neq 0$, the lower and upper tails of the joint distribution are fatter than in the normal case, which is coherent with the evidence from threshold correlations. Second, the skewness parameters $\gamma_{c,i}$ are to be interpreted as the extent of asymmetry in the correlation structure. When $\gamma \neq 0$, there is asymmetry in correlations, which is also coherent with the earlier threshold correlation analysis. Third, the $\alpha$ and $\beta$ parameters determine whether the copula generates time-varying correlations. If $\alpha \neq 0$ and $\beta \neq 0$, the copula is dynamic, which is consistent with the findings of the rolling correlation analysis. An overview of the six copula models is given in \autoref{fig:conceptual}.

%!TEX root=../../main.tex

\begin{table}
  \centering
  \footnotesize
  \renewcommand{\arraystretch}{1.2}

  \caption{Conceptual Matrix of Copula Parameterizations}

  \begin{tabularx}{0.80\textwidth}{@{} lc c >{\centering}Xc >{\centering}Xc >{\centering\arraybackslash}X}
    \toprule
      & && \textbf{Normal} && \textbf{Symmetric \emph{t}} && \textbf{Skewed \emph{t}} \\
      \cmidrule{4-4}
      \cmidrule{6-6}
      \cmidrule{8-8}
      & && $\nu_c = \infty$   && $\nu_c < \infty$   && $\nu_c < \infty$ \\
      & && $\gamma_{i,c} = 0$ && $\gamma_{i,c} = 0$ && $\gamma_{i,c} \neq 0$ \\
      \cmidrule{4-8}
    \cmidrule{1-2}
    \multirow{2}{*}{\textbf{Constant}} & $\alpha = 0$ && Constant && Constant && Constant \\
                              & $\beta = 0$  && Normal   && Symmetric \emph{t} && Skewed \emph{t}      \\
    \cmidrule{1-2}
    \multirow{2}{*}{\textbf{Dynamic}}  & $\alpha > 0$ && Dynamic  && Dynamic && Dynamic \\
                              & $\beta > 0$  && Normal   && Symmetric \emph{t} && Skewed \emph{t}      \\
    \bottomrule
  \end{tabularx}
% ()
%   \begin{tabularx}{\textwidth}{@{\extracolsep{5pt}} c c c c X c X c @{}}
%     \toprule
%   				&			& &	\textbf{Normal}	&	&	\textbf{Student's \textit{t}}	&	&	\textbf{Asymmetric Student's \textit{t}} \\
%   				\\
%   				&			& & 	$\nu = \infty$	&	&	$\nu > 0$	& 	&	$\nu > 0$ \\
%   				&			& & 	$\gamma = 0$	&	&	$\gamma = 0$	& 	&	$\gamma \neq 0$ \\
%           \\
%     \cmidrule{4-8}
%     \\
%      \textbf{Constant} &  & & \text{Constant normal copula} & & \text{Constant symmetric \textit{t} copula} & & \text{Constant asymmetric \textit{t} copula} \\
%      \\
%     	&	$\alpha = 0$  &		&	\textit{Constant correlations but} & & \textit{Constant correlations and} & & \textit{Constant correlations and} \\
%        & $\beta = 0$ & & \textit{no tail dependence} & & \textit{symmetric tail dependence} & & \textit{asymmetric tail dependence} \\
%     \\
%     \cmidrule{4-8}
%     \\
%      \textbf{Dynamic} &  & & \text{Dynamic normal copula} & & \text{Dynamic symmetric \textit{t} copula} & & \text{Dynamic asymmetric \textit{t} copula} \\
%      \\
%       & $\alpha > 0$  &   & \textit{Dynamic correlations but} & & \textit{Dynamic correlations and} & & \textit{Dynamic correlations and} \\
%        & $\beta > 0$ & & \textit{no tail dependence} & & \textit{symmetric tail dependence} & & \textit{asymmetric tail dependence} \\
%     \\
%     \bottomrule
%   \end{tabularx}

  \label{tab:conceptual}	
\end{table}


\subsubsection{Copula estimation results}

We estimate constant and dynamic normal, symmetric and asymmetric copula models on the full dataset of GARCH uniform residuals. Results are presented in~\autoref{tab:copula_estimation}. First, we examine the choice between a normal, symmetric \textit{t} or asymmetric \textit{t} copula. We note that $\nu_c$ is clearly significant and suggests a Student's \textit{t} model with tail dependence over the normal model. Second, we examine the asymmetric specification and find that few of the $\gamma_c$ estimates appear significant, indicating that the asymmetry is hard to capture or not consistent enough to merit modeling. This is supported by the relatively small improvement in log-likelihood by going from a symmetric to asymmetric copula and the fact that the BIC criterion prefers the symmetric model in the dynamic case. 

Second, we examine the choice between a constant and dynamic copula correlation matrix. There is a significant improvement in log-likelihood and BIC when moving from a constant to a dynamic copula, which suggests that time-varying dependence shown by rolling correlation is captured, which improves the model's fit. We also find a high persistence of the correlation process, as $\alpha + \beta$, is close to a unit root.

In summary, we find that the dynamic symmetric Student's \textit{t} copula is the best specification, as it has the lowest BIC, well defined parameters, and is strongly supported by the dependence pattern showcased by threshold and rolling correlation analyses. While the asymmetric Student's \textit{t} is an interesting model, we believe that the asymmetry patterns in data are too irregular to capture well in a copula model with only one asymmetry parameter for each series (this is further discussed in the subsequent robustness discussion, see \autoref{sub:05_robust}).
%what model do we retain (and why)

% section modeling_of_factor_returns (end)
