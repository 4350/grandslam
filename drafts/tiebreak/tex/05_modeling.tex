%!TEX root = ../main.tex

\section{Modeling of Factor Returns} % (fold)
\label{sec:modeling_of_factor_returns}

% Why a model?
This section presents our model for the joint behavior of returns. A multivariate model of returns allows us to make conditional forecasts of the distribution of returns one week ahead, which take into account the dependence between factors. The model is used in the mean-variance analysis, to provide dynamic inputs that can give us optimal weights over time. The model is also used in the analysis of diversification benefits, when we shift the focus to the tail risk of factor portfolios. First, we describe why we choose the copula model among different multivariate models. Second, we describe the building blocks of the model and analyze which specification best suits the dependence structure in the data. Last, we provide estimates and describe the model we retain for further use.

\input{tex/05_01_choosing} % motivating the choice of copula, and among copulas

%!TEX root = ../main.tex

\subsection{Definition of copula model} % (fold)
\label{sub:definition_of_copula_model}

Each week $t$, the conditional joint density of returns $R_{t+1} = \{r_{i,t+1},\ldots,r_{N,t+1}\}$ is described by a joint density function $f_t(R_{t+1})$. Following~\textcite{ChristoffersenErrunzaJacobLanglois2012}, who build on~\textcite{Patton2006} and~\textcite{Sklar1959}, we decompose the joint density function into the product of a joint copula function $c_t(U_{t+1})$ of uniformly distributed variables $U_{t+1} \sim U(0, 1)$ and marginal densities $f_{i,t}(r_{i,t+1})$:
\begin{align}
  f_t(R_{t+1}) =
    c_t(U_{t+1}) \prod^N_{i=1} f_{i,t}(r_{i,t+1})
  \label{eq:copula_sklar}
\end{align}
The elements of $U_{t+1} = \{u_{i,t+1},\ldots,u_{N,t+1}\}$ are related to the original returns by the probability integral transform, i.e the cumulative distribution of $r_{i,t+1}$:
\begin{align}
  u_{i,t+1} = F_{i,t}(r_{i,t+1}) = \int_{-\infty}^{r_{i,t+1}} f_{i,t}(r)dr
\end{align}
The copula function $c_t(U_{t+1})$ is a multivariate skewed Student's \emph{t} distribution. This distribution is parametrised by a single degrees of freedom parameter $\nu_c$, controlling the degree of dependency, a vector of $N$ skewness parameters $\gamma_c$, controlling the asymmetry in dependency, and a potentially time-varying correlation matrix $\Psi_{t}$. We describe the details of the skewed Student's t distribution, including the expanded form of $c_t$, in~\autoref{app:ghstmv}. The skewed Student's \emph{t} distribution nests the Student's \emph{t} distribution when all $\gamma_{i,c} = 0$ and the standard normal distribution when additionally $\nu_c = \infty$.

The log-likelihood of the model is constructed from~\autoref{eq:copula_sklar}:
\begin{align}
  L =
    \underbrace{\sum_{t=1}^T \log(c_t(U_{t+1}))}_\text{Copula} +
    \underbrace{\sum_{t=1}^T \sum_{i=1}^N \log(f_{i,t}(r_{i,t+1}))}_\text{Marginals}
\end{align}
At this point, it is worth noting that the joint density $c_t(U_{t+1})$ need not be of the same family as the marginal densities $f_{i,t}(r_{i,t+1})$ -- nor are we restricted to modeling $f_{i,t}(r_{i,t+1})$ jointly for all factors. In fact, we take advantage of this flexibility and choose to model the marginal densities independently as ARMA-GARCH processes, which allows us to capture a number of predictable features in the univariate series -- serial correlation, volatility clustering and leverage effects. The marginal models are estimated independently by maximizing the likelihood(s) of the second term, and then the copula is estimated by maximizing the first -- using the residuals of the marginal models.

This procedure is called multi-stage maximum log-likelihood or inference functions for margins and greatly simplifies the estimation procedure while yielding relatively efficient estimates~\autocite{Patton2006,Joe1997}. The modeling and estimation of our ARMA-GARCH models is detailed in the upcoming subsection, whereas the remainder of this subsection describes how we make $\Psi_t$ and thus the dependence between factors dynamic.

The copula is made dynamic by fitting a dynamic conditional correlation (DCC) process for $\Psi_t$ to copula residuals $z_{t+1}^*$~\autocite{Engle2002}. Using the notation from~\textcite{ChristoffersenLanglois2013}:
\begin{align}
  Q_t = (1 - \alpha - \beta) Q
    + \beta Q_{t-1}
    + \alpha \bar{z}_{t-1}^* \bar{z}_{t-1}^{*\top}
  \label{eq:copula_cdcc}
\end{align}
where $Q_t$ is normalized to the correlation matrix $\Psi_t$:
\begin{align}
  \Psi_t = Q_t^{-\frac{1}{2}} Q_t Q_t^{-\frac{1}{2}}
  \label{eq:copula_cdcc_psi}
\end{align}
The $Q_t$ process is comprised of three components that are weighted according to $\alpha, \beta$: (1) a time-invariant component, $Q$ (2) an innovation component from copula shocks, $\bar{z}_{t-1}^{*} \bar{z}_{t-1}^{*\top},$\footnote{Where $\bar{z}_{i,t+1}^* = z_{i,t+1}^* \sqrt{q_{ii,t}}$ is due to a correction by~\textcite{Aielli2013}, that improves the reliability of the estimation procedure.} and (3) an autoregressive component of order one $Q_{t-1}$. In order for the the correlation matrix $\Psi_t$ to be positive definite, $Q_t$ has to be positive definite, which is ascertained by requiring that $\alpha \geq 0$, $\beta \geq 0$ and $(\alpha + \beta) < 1$. The model nests a constant copula when $\alpha = \beta = 0$.

The model for $c_t(U_{t+1})$ is thus comprised of $1 + N$ distribution parameters $\{\nu_c, \gamma_c\}$ and ($2 + (N(N-1) / 2)$) dynamics parameters $\{\alpha, \beta, Q\}$ where the elements of $Q$ are estimated using moment matching, and the remaining $3 + N$ parameters are estimated using maximum likelihood.\footnote{We have relegated the details of the estimation of the dynamic process to~\autoref{app:copula_cdcc}.}

% Gustaf: Här är det svårt -- jag har svårt att beskriva $z_t^*$ när konceptet
% standardized returns är en del av ARMA-GARCH delen. Vet ej riktigt hur man
% ska komma åt det. Ett sätt är att ta tjuren vid hornen och beskriva att 
% f_{i,t}(r_{i,t+1}) = f_i(z_{i,t+1}) = 
% f_i(\varepsilon_{i,t+1}/\sigma_{i,t+1}). men ja.
ARMA-GARCH modeling allows us to filter time-varying effects, leaving independent \emph{standardized returns} (or standardized residuals) $z_{i,t}$ assumed to follow a constant distribution $f_i(z_{i,t})$. These residuals are first transformed into uniform variables $u_{i,t+1}$ by the probability integral transform of the densities above, and then made to follow the \emph{copula} distribution by the \emph{inverse} probability integral transform of the \emph{copula}:
\begin{align}
  z_{i,t+1}^* = F^{-1}_{\nu_c,\gamma_{i,c}}(F_{i}(z_{i,t+1}))
\end{align}



% subsection definition_of_copula_model (end)
 % the boom

%!TEX root = ../main.tex

\subsection{Univariate models} % (fold)
\label{sub:univariate_models}

ARMA-GARCH processes are used to model the marginal densities $f_i(r_{i,t+1})$
by constructing equations for the conditional expectation of returns $\mathbb{E}_t[r_{i,t+1}]$ and volatilies $\sigma_{i,t}$ each week. This results in \emph{standardized returns} $z_{i,t}$ which are independent and assumed to follow a constant density so that:
\begin{align}
  f_{i,t+1}(r_{i,t+1}) = f_i(z_{i,t+1}) = f_i(\frac{r_{i,t+1} - \mathbb{E}_t[r_{i,t+1}]}{\sigma_{i,t}})
\end{align}
The standardized returns.

% Gustaf: Här är det svårt -- jag har svårt att beskriva $z_t^*$ när konceptet
% standardized returns är en del av ARMA-GARCH delen. Vet ej riktigt hur man
% ska komma åt det. Ett sätt är att ta tjuren vid hornen och beskriva att 
% f_{i,t}(r_{i,t+1}) = f_i(z_{i,t+1}) = 
% f_i(\varepsilon_{i,t+1}/\sigma_{i,t+1}). men ja.
ARMA-GARCH modeling allows us to filter time-varying effects, leaving independent \emph{standardized returns} (or standardized residuals) $z_{i,t}$ assumed to follow a constant distribution $f_i(z_{i,t})$. These residuals are first transformed into uniform variables $u_{i,t+1}$ by the probability integral transform of the densities above, and then made to follow the \emph{copula} distribution by the \emph{inverse} probability integral transform of the \emph{copula}:
\begin{align}
  z_{i,t+1}^* = F^{-1}_{\nu_c,\gamma_{i,c}}(F_{i}(z_{i,t+1}))
\end{align}

% subsection univariate_models (end)
 % arma garch

\input{tex/05_04_dependence} % threshold and rolling

%!TEX root = ../main.tex
\subsection{Copula specification and estimation results}
Given the results of the dependence structure of residuals, we now discuss the best choice of copula model and present estimation results of the six competing copula specifications.

\subsubsection{Interpreting and choosing copula specification}

The interpretation of the copula parameterization is closely associated to the structure of multivariate dependence. By different restrictions on the parameters in the DAC model, we are able to activate or deactivate certain features of the copula: First, the degree of freedom parameter $nu_c$ is to be interpreted as a the measure of tail dependency. When $nu \neq 0$, the lower and upper tails of the joint distribution are fatter than in the normal case, which is coherent with the evidence from threshold correlations. Second, the skewness parameters $\gamma_{c,i}$ are to be interpreted as the extent of asymmetry in the correlation structure. When $\gamma \neq 0$, there is asymmetry in correlations, which is also coherent with the earlier threshold correlation analysis. Third, the $\alpha$ and $\beta$ parameters determine whether the copula generates time-varying correlations. If $\alpha \neq 0$ and $\beta \neq 0$, the copula is dynamic, which is consistent with the findings of the rolling correlation analysis. An overview of the six copula models is given in \autoref{fig:conceptual}.

%!TEX root=../../main.tex

\begin{table}
  \centering
  \footnotesize
  \renewcommand{\arraystretch}{1.2}

  \caption{Conceptual Matrix of Copula Parameterizations}

  \begin{tabularx}{0.80\textwidth}{@{} lc c >{\centering}Xc >{\centering}Xc >{\centering\arraybackslash}X}
    \toprule
      & && \textbf{Normal} && \textbf{Symmetric \emph{t}} && \textbf{Skewed \emph{t}} \\
      \cmidrule{4-4}
      \cmidrule{6-6}
      \cmidrule{8-8}
      & && $\nu_c = \infty$   && $\nu_c < \infty$   && $\nu_c < \infty$ \\
      & && $\gamma_{i,c} = 0$ && $\gamma_{i,c} = 0$ && $\gamma_{i,c} \neq 0$ \\
      \cmidrule{4-8}
    \cmidrule{1-2}
    \multirow{2}{*}{\textbf{Constant}} & $\alpha = 0$ && Constant && Constant && Constant \\
                              & $\beta = 0$  && Normal   && Symmetric \emph{t} && Skewed \emph{t}      \\
    \cmidrule{1-2}
    \multirow{2}{*}{\textbf{Dynamic}}  & $\alpha > 0$ && Dynamic  && Dynamic && Dynamic \\
                              & $\beta > 0$  && Normal   && Symmetric \emph{t} && Skewed \emph{t}      \\
    \bottomrule
  \end{tabularx}
% ()
%   \begin{tabularx}{\textwidth}{@{\extracolsep{5pt}} c c c c X c X c @{}}
%     \toprule
%   				&			& &	\textbf{Normal}	&	&	\textbf{Student's \textit{t}}	&	&	\textbf{Asymmetric Student's \textit{t}} \\
%   				\\
%   				&			& & 	$\nu = \infty$	&	&	$\nu > 0$	& 	&	$\nu > 0$ \\
%   				&			& & 	$\gamma = 0$	&	&	$\gamma = 0$	& 	&	$\gamma \neq 0$ \\
%           \\
%     \cmidrule{4-8}
%     \\
%      \textbf{Constant} &  & & \text{Constant normal copula} & & \text{Constant symmetric \textit{t} copula} & & \text{Constant asymmetric \textit{t} copula} \\
%      \\
%     	&	$\alpha = 0$  &		&	\textit{Constant correlations but} & & \textit{Constant correlations and} & & \textit{Constant correlations and} \\
%        & $\beta = 0$ & & \textit{no tail dependence} & & \textit{symmetric tail dependence} & & \textit{asymmetric tail dependence} \\
%     \\
%     \cmidrule{4-8}
%     \\
%      \textbf{Dynamic} &  & & \text{Dynamic normal copula} & & \text{Dynamic symmetric \textit{t} copula} & & \text{Dynamic asymmetric \textit{t} copula} \\
%      \\
%       & $\alpha > 0$  &   & \textit{Dynamic correlations but} & & \textit{Dynamic correlations and} & & \textit{Dynamic correlations and} \\
%        & $\beta > 0$ & & \textit{no tail dependence} & & \textit{symmetric tail dependence} & & \textit{asymmetric tail dependence} \\
%     \\
%     \bottomrule
%   \end{tabularx}

  \label{tab:conceptual}	
\end{table}


\subsubsection{Copula estimation results}

We estimate constant and dynamic normal, symmetric and asymmetric copula models on the full dataset of GARCH uniform residuals. Results are presented in~\autoref{tab:copula_estimation}. First, we examine the choice between a normal, symmetric \textit{t} or asymmetric \textit{t} copula. We note that $\nu_c$ is clearly significant and suggests a Student's \textit{t} model with tail dependence over the normal model. Second, we examine the asymmetric specification and find that few of the $\gamma_c$ estimates appear significant, indicating that the asymmetry is hard to capture or not consistent enough to merit modeling. This is supported by the relatively small improvement in log-likelihood by going from a symmetric to asymmetric copula and the fact that the BIC criterion prefers the symmetric model in the dynamic case. 

Second, we examine the choice between a constant and dynamic copula correlation matrix. There is a significant improvement in log-likelihood and BIC when moving from a constant to a dynamic copula, which suggests that time-varying dependence shown by rolling correlation is captured, which improves the model's fit. We also find a high persistence of the correlation process, as $\alpha + \beta$, is close to a unit root.

In summary, we find that the dynamic symmetric Student's \textit{t} copula is the best specification, as it has the lowest BIC, well defined parameters, and is strongly supported by the dependence pattern showcased by threshold and rolling correlation analyses. While the asymmetric Student's \textit{t} is an interesting model, we believe that the asymmetry patterns in data are too irregular to capture well in a copula model with only one asymmetry parameter for each series (this is further discussed in the subsequent robustness discussion, see \autoref{sub:05_robust}). % estimation results and explaining parameterization choice

%!TEX root = ../main.tex
\subsection{Copula robustness check}

This section provides a robustness check of how well the copula models can reproduce the threshold correlations found in ARMA-GARCH residuals. Note that to make this comparison valid, we use the constant copula specifications -- the dynamic version is the workhorse for all continued analysis in the mean-variance and diversification benefit sections.

\subsubsection{Simulated threshold correlations}

By simulating 250,000 weeks of shocks in the copula, and then transforming these shocks into standardized residuals for each of the factors, we can test the constant \textit{t} copula's ability to generate the threshold correlations in the ARMA-GARCH residuals. If the copula specification reasonably well captures tail dependence, the threshold correlations from the empirical and the copula specification should align. The results are presented in~\autoref{fig:threshold_simulated1}.

\begin{figure}[!ht]
  \centering
  \caption{Threshold correlations of simulated constant copula standardized returns \\ \quad \\ Threshold correlations of simulated constant copula, with ARMA-GARCH standardized returns (95\% confidence bounds taking the ARMA-GARCH models as given). The simulated threshold correlations are based on 250,000 simulated returns each.}
  \includegraphics[scale=1]{graphics/threshold_simulated_1.png}  
  \label{fig:threshold_simulated1}
\end{figure}
\begin{figure}[!ht]
  \ContinuedFloat
  \centering
  \caption{Threshold correlations of simulated copula standardized returns (cont.)}
  \includegraphics[scale=1]{graphics/threshold_simulated_2.png}  
\end{figure}

First, we note that for most factors, the normal copula is the far away from generating threshold correlations that correspond to the empirical distribution around the median. This is highly expected, as the normal copula does not generate tail dependence, and highlights the need for the Student's \textit{t} based copula models. The symmetric \textit{t} and asymmetric \textit{t} copulae better capture the threshold correlations, as the fatter tails of the Student's \textit{t} distribution allows for tail dependence. For example, note how the normal copula generates negative threshold correlations for both the Mom--HML and RMW--HML asset pairs, while the Student's \textit{t} based copulae are much closer to the higher values in the data. On the other hand, the Student's \textit{t} based copulae sometimes seem to overshoot the empirical threshold correlation, as in the Mkt-RF--RMW asset pair.

Second, we find that the skewed Student's \textit{t} generates some asymmetry around the mean, which can be seen most clearly for the Mom--RMW and RMW--HML asset pairs. The generated asymmetry does, however, appear to be too small to capture the features of the data.

In conclusion, comparing threshold correlations from empirical data and simulated data shows that our copula approach captures some of the tail dependence. Although the Student's \textit{t} and skewed Student's \textit{t} results do not align perfectly with the data, they constiute clear improvements to the normal copula in modeling tail dependence. We do note that the copula seems to lack flexibility to simultaneously generate all the asymmetries in tail dependence. This is quite expected, as the Student's \textit{t} copula only has one degree of freedom parameter that controls the fatness of tails, and the skewed Student's \textit{t} copula only has one skewness parameter for each series. This imposes limits on how strongly the model can express fat tails or asymmetries between factors A and B and simultaneously express other fat tails or asymmetries (or lack thereof) between factors A and C. For a collection of six factors with heterogenous dependence, this is even harder. This is a clear limitation of the quite parsimonious copula approach, and will be discussed further in the concluding section. Although imperfect, the copula modeling of tail dependence could constitute a significant improvement to alternatives, especially in the field of risk management, where understanding of tail events is paramount.

 % constant threshold corrs simulated, maybe also add robustness check of whether the model captures roll-corrs 

%\subsection{Model retained for further use}
%Here we could describe again the model we use in MV and CDB. E.g. dynamic std. If the reader doesn't follow anything of the first four subsections, she should at least be able to pick up what we use here.

% section modeling_of_factor_returns (end)
