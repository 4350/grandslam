%!TEX root = ../main.tex

\section{Modeling Factor Returns} % (fold)
\label{sec:modeling_factor_returns}

We proceed by estimating parsimonious models of each factor return series, trying to capture the established effects of serial correlation, volatility clustering, non-normality and leverage effects. By estimating appropriate models, we can characterize and filter these effects, while incorporating them in conditional forecasts. In this section, we describe the general case of our models, our systematic model selection approach and estimation results.

\subsection{General univariate model: GJR-GARCH} % (fold)
\label{sub:general_univariate_model_gjr_garch}

ARMA-GARCH covers a broad class of models designed to capture the previously mentioned features of financial return series. We evaluate the GJR-GARCH model which is a parsimonious extension of the standard GARCH(1, 1) model designed to capture leverage effects~\autocite{glosten1993relation}. The GJR-GARCH model is evaluated with and without leverage effects ($\eta = 0$) and under different distributional assumptions (normal, Student's t and skewed Student's t), for each factor.

We estimate conditional mean equations \emph{up to} ARMA(3, 3):
\begin{align}
  r_t &=
    \mu +
    \sum^p \phi_p r_{t - p} +
    \sum^q \theta_q \epsilon_{t - q} + 
    \epsilon_t
\end{align}
The conditional volatility evolves according to the GJR-GARCH specification:
\begin{align}
  \epsilon_t &= \sigma_t z_t \\\
  \sigma_t^2 &=
    \omega +
    (\alpha + \eta I_{t - 1}) \epsilon_{t - 1}^2 +
    \beta \sigma^2_{t - 1} \\
  \intertext{where}
  I_{t - 1} &=
    \begin{cases}
      1 & \text{if}\ \epsilon_{t - 1} \le 0 \\
      0 & \text{if}\ \epsilon_{t - 1} \gt 0      
    \end{cases}
\end{align}
A positive $\eta$ captures the typical leverage effect by increasing the current period's volatility if the previous period's residual $\epsilon$ was below zero. A significant $\eta$ thus introduces asymmetric volatility in the model. For the market factor, it can be expected that $\eta$ is positive, reflecting the leverage effect in the market itself. However, for the other factors, constructed as long-short positions in risky portfolios, the direction of $\eta$ is less obvious~\autocite{ChristoffersenLanglois2013}.

The GJR-GARCH models are estimated on each series using maximum likelihood estimation given an assumed conditional distribution of $z_t$. We evaluate models where the standardized residuals $z_t$ are assumed to follow one of the following distributions: Standard normal, Student's t with $\nu$ degrees of freedom and skewed Student's t with $\nu$ degrees of freedom and skewness $\gamma$. Of these, Student's t allows greater kurtosis than possible with a normal distribution, while skewed Student's t \emph{also} allows for additional asymmetry in the model's behavior beyond that introduced by the leverage effect.

% subsection general_univariate_model_gjr_garch (end)

\subsection{Selection Process} % (fold)
\label{sub:selection_process}

Our selection process is as follows. For each factor strategy, we estimate GJR-GARCH models on the full dataset ($T = 2766$) up to ARMA(3, 3) and GARCH(1, 1) under normal, Student's t and skewed Student's residuals, with and without $\eta$ fixed to zero (in which case we obtain the basic GARCH(1, 1) model). We then compute the Bayesian Information Criterion (BIC~\autocite{Schwarz1978}) for each specification and select the ARMA order with the lowest BIC as our primary candidates.

The candidate models are checked for remaining serial correlation and ARCH effects in the residuals, by using the weighted portmanteau tests of~\autocite{FisherGallagher2012}. We also compute the sign bias statistic in order to choose between GARCH and GJR-GARCH specifications -- if the statistic is insignificant, this suggests that there is no significant leverage effect present in the series. GARCH diagnostics results can be seen in~\autoref{table:garch_diagnostics}. In a well-specified model, we expect there to be no significant serial correlation, ARCH effects or sign bias in the residuals. Furthermore, the QQ-plots of the standardized residuals should show that their empirical distribution is comparable to the theoretical distribution (be distributed around the 45 degree line).

% TODO Sign bias statistic???
% TODO Say more interesting stuff about GARCH diagnostics

The result of our selection procedure is to model each of the six factor series up to a maximum of ARMA(1, 1) with an asymmetric Student's t distribution. The significant sign bias statistics in the GARCH specifications, lead us to preferring GJR-GARCH specifications for the Market and CMA series, while GARCH is selected for the remaining four.

The candidate specifications under normal or Student's t distributed innovations all display misbehaved QQ-plots, see~\autoref{fig:qq_norm,fig:qq_std}. Rather than falling ``around'' the 45 degree line, the empirical distributions deviate, especially in the more extreme quantiles. This is indicative of asymmetry in the residual series (even after accounting for leverage effects). By comparision (see~\autoref{fig:qq_ghst}), the empirical distributions of residuals in the chosen models seem to largely agree with the theoretical asymmetric Student's t distribution.

The lack of significant sign bias in the GARCH specifications for all models except Market and CMA (at 10\% level) is notable, but in line with the previous argument that any leverage effects can ``cancel out'' in a risky long-short portfolio; the Market-RF being short in the risk-free asset therefore shows the significant leverage effect of the market itself. For the Market series, the GJR-GARCH specification appears to remove this sign bias.

We additionally note that the Market model selects order ARMA(0, 0), suggesting that predicting the 1-step return on a market portfolio is hard to do based on serial correlation.

% subsection selection_process (end)

% section modeling_factor_returns (end)
