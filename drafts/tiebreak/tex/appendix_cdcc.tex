%!TEX root = ../main.tex

\section{Copula Estimation Procedure} % (fold)
\label{app:copula_cdcc}

This is a step-by-step description of the procedure used to estimate the copula model, given the set of standardized residuals $\{z_{t}\}$ from each GARCH model. It is adapted from~\textcite{ChristoffersenErrunzaJacobLanglois2012} and uses the cDCC model of~\textcite{Aielli2013}.

For each week, compute uniform residuals by applying the probability integral transform to standardized residuals from each GARCH model:
\begin{align}
  u_{i, t+1} = \int_{-\infty}^{z_{i,t+1}} f_{i}(z_{i,t+1})
\end{align}
Note that while the distributions of returns is time-varying due to GARCH dynamics, the distribution of standardized residuals is assumed constant, and in the case of skewed Student's t is parametrised by the shape $\nu_i$ and skewness parameter $\gamma_i$ (estimated as part of the GARCH models).

Transform the uniform residuals into copula residuals $z_{i,t+1}^*$ by applying the inverse cumulative distribution function of the copula to them:
\begin{align}
  z_{i,t+1}^* = F_{\nu_c,\gamma_i}^{-1}(u_{i,t+1})
\end{align}
Only under the normal copula will these residuals have expectation zero and unit variance -- hence, they are standardized by subtracting the expectation and dividing by the standard deviation of the distribution from~\autoref{app:ghstmv}.

These shocks are now used to fit the corrected DCC process of~\textcite{Aielli2013}. The correction involves the transformation $\bar{z}_{i,t+1}^* = z_{i,t+1}^* \sqrt{q_{ii,t}}$, where $q_{ii,t}$ are the diagonal elements of $Q_t$ and are found by a scalar version of~\autoref{eq:copula_cdcc}:
\begin{align}
  q_{ii,t} = (1 - \alpha - \beta)
    + \alpha (\bar{z}_{i,t-1}^*)^2
    + \beta q_{ii,t-1}
\end{align}
The corrected shocks are used to estimate the time-invariant component $Q$ using moment matching:
\begin{align}
  \hat{Q} = \frac{1}{T} \sum_{t=1}^T \bar{z}_{t}^* \bar{z}_t^{*\top}
\end{align}
Now, the full estimates of $\hat{Q}_t$ are computed using the sample estimate $\hat{Q}$ and the corrected shocks:
\begin{align}
  \hat{Q}_t = (1 - \alpha - \beta) \hat{Q}
    + \beta \hat{Q}_{t-1}
    + \alpha \bar{z}_{t-1}^* \bar{z}_{t-1}^{*\top}
\end{align}
The estimated $\hat{Q}_t$ matrices are standardized to estimates $\hat{\Psi}_t$ of the conditional correlation matrices of the copula using~\autoref{eq:copula_cdcc_psi}. When fitting the model, we thus choose parameters $\nu_c, \gamma_c, \alpha, \beta$ to generate $\hat{\Psi}_t$ which maximize the log-likelihood of observing copula shocks $z_t^*$ in each period.

% section copula_estimation_procedure (end)
