%!TEX root = ../main.tex

\section{Abnormal return (alpha) regressions}
\label{sec:alpha_reg}
\textcite{FF2015} run factor regressions where both the LHS variable and the RHS variables are zero-cost factor portfolios. In this type of zero-cost portfolio regression, the intercept is to be interpreted as the abnormal return, or Jensen's alpha, of adding the LHS factor to a portfolio already consisting of the RHS factors \autocite{Jensen1968}. As a specific example, we begin by considering the regression that has caused the discussion on whether HML is a redundant variable. They run the regression
\begin{align}
  r^{HML}_t = \alpha + \beta_1 r^{Mkt.RF}_t + \beta_2 r^{SMB}_t + \beta_3 r^{RMW}_t + \beta_4 r^{CMA}_t + \epsilon_t
\end{align}
where $r^i_t$ denote monthly returns. The central finding is that HML is completely subsumed by the four factors Mkt.RF, SMB, RMW and CMA -- the alpha of the regression is very small and not statistically significant. In other words, adding HML to a portfolio of the other four factors should give no alpha.

Our regression analysis deviates from that in \textcite{FF2015} as we consider weekly return data and a slightly extended data window. However, the main regression specifications are the same. In \autoref{fig:abnormal}, regressions for the five-factor and six-factor models are presented. Each column represents one unique regression, with one of the factors as the LHS variable and with the remaining four (or five) factors as RHS variables (in rows). The reported standard errors are adjusted for autocorrelation, as such effects are found in the return data (see \autoref{sec:data}).

First, we examine regression (1) in a five-factor model where HML is the LHS variable. We note that the alpha of HML is not significant, indicating that the factor is completely subsumed by the remaining four factors and does not create additional value in a portfolio setting. More specifically, the only factor that explains HML is CMA, with a high coefficient of 0.85, with all other factor loadings insignificant and close to zero. Note, however, that the high loading on CMA does not imply any ordering or causality as the parameter estimate is based on contemporary covariance alone. The R-squared of 45\% indicates that the HML factor has substantial variance that is unexplained by the regression, indicating that the factor comprises unique information beyond that in CMA.

Second, we turn to regression (2) in a five factor model where CMA is the LHS variable. Here, the alpha is significant, indicating that the factor does provide additional value beyond the existing four factors. While the CMA factor loads positively 0.39 on HML, this is substantially lower than HML's loading on CMA of 0.85. Put differently, CMA seems to be explained by other factors than HML to a higher extent than vice versa. With CMA as the LHS variable, there is significant negative loading on RMW and a significant positive loading on Mom.

Now, we move to the six-factor regression results, where we include the momentum factor Mom. First, in regression (3) with HML as the LHS variable, we note that the addition of HML has made the alpha of HML positive and significant. As momentum is correlated with both the LHS and RHS factors, it constituted an omitted variable bias on the beta factor loadings in the five-factor model. HML has a substantial negative loading on the Mom factor of -0.18, while the CMA regression instead has a positive Mom loading of 0.09, indicating that the seemingly similar factors are quite different in terms of momentum properties. The momentum factor has improved the r-squared when HML is the LHS variable, explaining an additional 8\% of the variance in the factor.

In summary, our results are qualitatively similar to the results in \textcite{FF2015} as well as \textcite{Asness2015}, although we employ weekly data. The alpha of HML is only recognized in a model including momentum. This finding indicates that the insignificant alpha of HML might not be meaningful for portfolio choice models, including mean-variance optimization. Furthermore, it highlights a discrepancy between the seemingly similar HML and CMA strategies.

\begin{table}[!htbp] \centering 
  \caption{Abnormal return regressions -- Five and six factor models} 
  \label{fig:abnormal} 
\begin{tabularx}{\textwidth}{X}
\\[-1.8ex]\toprule
\\[-1.8ex] 
\footnotesize Abnormal return regressions on zero-cost equity factor portfolios, following the analysis in \textcite{FF2015} and \textcite{Asness2015}. Heteroskedasticy and autocorrelation robust standard errors in parentheses, following \textcite{NeweyWest1987}. Alpha to be interpreted as abnormal return (Jensen's alpha). Weekly log returns 1963-2016
\end{tabularx}
\begin{tabularx}{\textwidth}{@{\extracolsep{0pt}}lD{.}{.}{-4} D{.}{.}{-4} D{.}{.}{-4} D{.}{.}{-4} } 
\\[-1.8ex]\midrule 
\\[-1.8ex] 
  & \multicolumn{4}{c}{\textit{LHS variable:}} \\ 
\cline{2-5}
\\ 
& \multicolumn{2}{c}{\textit{5 factor model}} & \multicolumn{2}{c}{\textit{6 factor model}} 
\\[-1.8ex] & \multicolumn{4}{c}{ } \\ 
 \\[-1.8ex] & \multicolumn{1}{c}{(1)} & \multicolumn{1}{c}{(2)} & \multicolumn{1}{c}{(3)} & \multicolumn{1}{c}{(4)}\\
\\[-1.8ex] & \multicolumn{1}{c}{HML} & \multicolumn{1}{c}{CMA} & \multicolumn{1}{c}{HML} & \multicolumn{1}{c}{CMA}\\
\hline \\[-1.8ex] 
 Alpha & 0.0002 & 0.0006^{***} & 0.0005^{**} & 0.0004^{***} \\ 
  & (0.0002) & (0.0001) & (0.0002) & (0.0001) \\ 
  & & & & \\ 
 Mkt.RF & -0.0180 & -0.1103^{***} & -0.0320 & -0.0964^{***} \\ 
  & (0.0342) & (0.0187) & (0.0261) & (0.0138) \\ 
  & & & & \\ 
 SMB & 0.0010 & -0.0314 & 0.0079 & -0.0330^{*} \\ 
  & (0.0297) & (0.0194) & (0.0261) & (0.0179) \\ 
  & & & & \\ 
 HML &  & 0.3897^{***} &  & 0.4248^{***} \\ 
  &  & (0.0421) &  & (0.0323) \\ 
  & & & & \\ 
 CMA & 0.8526^{***} &  & 0.8673^{***} &  \\ 
  & (0.0441) &  & (0.0436) &  \\ 
  & & & & \\ 
 RMW & -0.0198 & -0.0942^{**} & 0.0100 & -0.1021^{**} \\ 
  & (0.0931) & (0.0446) & (0.0684) & (0.0404) \\ 
  & & & & \\ 
 Mom &  &  & -0.1814^{***} & 0.0868^{***} \\ 
  &  &  & (0.0440) & (0.0197) \\ 
  & & & & \\ 
\hline \\[-1.8ex] 
R-squared & \multicolumn{1}{c}{39\%} & \multicolumn{1}{c}{46\%} & \multicolumn{1}{c}{47\%} & \multicolumn{1}{c}{49\%} \\ 
\bottomrule \\[-1.8ex] 
\textit{Note:}  & \multicolumn{4}{c}{$^{*}$p$<$0.1; $^{**}$p$<$0.05; $^{***}$p$<$0.01} \\ 
\end{tabularx} 
\end{table}
