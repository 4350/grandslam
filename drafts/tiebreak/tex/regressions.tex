%!TEX root = ../main.tex

\section{Abnormal return (alpha) regressions}
\label{sec:alpha_reg}
\textcite{FF2015} run factor regressions where both the LHS variable and the RHS variables are zero-cost factor portfolios. In this type of zero-cost portfolio regression, the intercept is to be interpreted as the abnormal return, or Jensen's alpha, of adding the LHS factor to a portfolio already consisting of the RHS factors \autocite{Jensen1968}. As a specific example, we begin by considering the regression that has caused the discussion on whether HML is a redundant variable. They run the regression
\begin{align}
  r^{HML}_t = \alpha + \beta_1 r^{Mkt.RF}_t + \beta_2 r^{SMB}_t + \beta_3 r^{RMW}_t + \beta_4 r^{CMA}_t + \epsilon_t
\end{align}
where $r^i_t$ denote monthly returns. The central finding is that HML is completely subsumed by the four factors Mkt.RF, SMB, RMW and CMA -- the alpha of the regression is very small and not statistically significant. In other words, adding HML to a portfolio of the other four factors should give no alpha.

Our regression analysis deviates from that in \textcite{FF2015} as we consider weekly return data and a slightly extended data window. However, the main regression specifications are the same. In \autoref{fig:abnormal_five}, regressions for the five-factor model are presented. Each column represents one unique regression, with one of the five factors as the LHS variable and with the remaining four factors as RHS variables (in rows). 

First, we examine the HML regression. We note that the alpha of HML is not significant, indicating that the factor is completely subsumed by the remaining four factors and does not create additional value in a portfolio setting. More specifically, the only factor that has HML loads significantly on is CMA, with a high loading of 0.85. All remaining factor loadings for HML are insignificant and close to zero. Although HML does load on CMA substantially, the r-squared of 45\% indicates that the HML factor has substantial unexplained variance left, which means that the factor comprises unique information beyond that in CMA. 

Second, we turn to the CMA regression. Here, the alpha is significant, indicating that the factor does provide additional value beyond the existing four factors. While the CMA factor loads positively 0.39 on HML, this is substantially lower than HML's loading on CMA of 0.85. Furthermore, CMA also seems to be explained by other factors that are significantly estimated different from zero, including a negative loading on RMW and a positive loading on Mom.

Now, we move to the results in \autoref{fig:abnormal_six}, where we include the momentum factor Mom, effectively moving to a six-factor model. First, in the HML regression we note that the move to a six-factor universe has made the alpha of HML positive and significant. Possibly, momentum constituted an omitted variable bias in the five-factor regression, confounding the parameter estimates in such a way that the alpha was not recognized. Now, we note that there is a substantial negative loading of HML on the Mom factor of -0.18, while the CMA regression instead has a positive Mom loading of 0.09.

In summary, all of our results are qualitatively similar to the results in \textcite{FF2015} as well as \textcite{Asness2015}, although we employ weekly data. The alpha of HML is only recognized in a model including momentum, which possibly constitutes an omitted variable bias in the five-factor regression. This finding indicates that the insignificant alpha of HML might not be meaningful for portfolio choice models, including mean-variance optimization. Furthermore, it highlights a discrepancy between the seemingly similar HML and CMA strategies.

\begin{table}[!htbp] \centering 
  \caption{Abnormal return regressions -- Five factor model} 
  \label{fig:abnormal_five} 
\begin{tabularx}{\textwidth}{X}
\\[-1.8ex]\toprule
\\[-1.8ex] 
\footnotesize Abnormal return regressions on zero-cost equity factor portfolios, following the analysis in \textcite{FF2015}. Heteroskedasticy robust standard errors in parentheses, following \textcite{White1982}. Alpha to be interpreted as abnormal return (Jensen's alpha). Weekly log returns 1963-2016
\end{tabularx}
\begin{tabularx}{\textwidth}{@{\extracolsep{5pt}}X rrrrr} 
\\[-1.8ex]\midrule 
\\[-1.8ex] 
  & \multicolumn{5}{c}{\textit{LHS variable:}} \\ 
\cline{2-6} 
\\[-1.8ex] & \multicolumn{5}{c}{ } \\ 
 & Mkt.RF & SMB & HML & CMA & RMW \\ 
\\[-1.8ex] & (1) & (2) & (3) & (4) & (5)\\ 
\hline \\[-1.8ex] 
 Alpha & 0.0022$^{***}$ & 0.0008$^{***}$ & 0.0002 & 0.0006$^{***}$ & 0.0009$^{***}$ \\ 
  & (0.0004) & (0.0002) & (0.0002) & (0.0001) & (0.0002) \\ 
  & & & & & \\ 
 Mkt.RF &  & $-$0.0006 & $-$0.0180 & $-$0.1103$^{***}$ & $-$0.0778$^{***}$ \\ 
  &  & (0.0182) & (0.0194) & (0.0111) & (0.0111) \\ 
  & & & & & \\ 
 SMB & $-$0.0018 &  & 0.0010 & $-$0.0314$^{**}$ & $-$0.2366$^{***}$ \\ 
  & (0.0530) &  & (0.0223) & (0.0141) & (0.0256) \\ 
  & & & & & \\ 
 HML & $-$0.0742 & 0.0015 &  & 0.3897$^{***}$ & $-$0.0141 \\ 
  & (0.0819) & (0.0317) &  & (0.0245) & (0.0405) \\ 
  & & & & & \\ 
 CMA & $-$0.9965$^{***}$ & $-$0.0973$^{**}$ & 0.8526$^{***}$ &  & $-$0.1463$^{***}$ \\ 
  & (0.0944) & (0.0445) & (0.0370) &  & (0.0432) \\ 
  & & & & & \\ 
 RMW & $-$0.4522$^{***}$ & $-$0.4726$^{***}$ & $-$0.0198 & $-$0.0942$^{***}$ &  \\ 
  & (0.0650) & (0.0429) & (0.0580) & (0.0269) &  \\ 
  & & & & & \\ 
\bottomrule \\[-1.8ex] 
\textit{Note:}  & \multicolumn{5}{c}{$^{*}$p$<$0.1; $^{**}$p$<$0.05; $^{***}$p$<$0.01} \\ 
\end{tabularx} 
\end{table}

\begin{table}[!htbp] \centering 
  \caption{Abnormal return regressions -- Six factor model} 
  \label{fig:abnormal_six} 
\begin{tabularx}{\textwidth}{X}
\\[-1.8ex]\toprule
\\[-1.8ex] 
\footnotesize Abnormal return regressions on zero-cost equity factor portfolios, following the analysis in \textcite{FF2015}. Heteroskedasticy robust standard errors in parentheses, following \textcite{White1982}. Alpha to be interpreted as abnormal return (Jensen's alpha). Weekly log returns 1963-2016
\end{tabularx}
\begin{tabularx}{\textwidth}{@{\extracolsep{0pt}}X rrrrrr} 
\\[-1.8ex]\midrule 
\\[-1.8ex] 
 & \multicolumn{6}{c}{\textit{LHS variable:}} \\ 
\cline{2-7} 
\\[-1.8ex] & \multicolumn{6}{c}{ } \\ 
 & Mkt.RF & SMB & HML & CMA & RMW & Mom \\ 
\\[-1.8ex] & (1) & (2) & (3) & (4) & (5) & (6)\\ 
\hline \\[-1.8ex] 
 Alpha & 0.0024$^{***}$ & 0.0008$^{***}$ & 0.0005$^{***}$ & 0.0004$^{***}$ & 0.0009$^{***}$ & 0.0016$^{***}$ \\ 
  & (0.0004) & (0.0002) & (0.0002) & (0.0001) & (0.0002) & (0.0003) \\ 
  & & & & & & \\ 
 Mkt.RF &  & 0.0009 & $-$0.0320$^{**}$ & $-$0.0964$^{***}$ & $-$0.0744$^{***}$ & $-$0.0891$^{***}$ \\ 
  &  & (0.0183) & (0.0153) & (0.0087) & (0.0117) & (0.0265) \\ 
  & & & & & & \\ 
 SMB & 0.0026 &  & 0.0079 & $-$0.0330$^{**}$ & $-$0.2367$^{***}$ & 0.0384 \\ 
  & (0.0527) &  & (0.0197) & (0.0134) & (0.0263) & (0.0453) \\ 
  & & & & & & \\ 
 HML & $-$0.1484$^{**}$ & 0.0127 &  & 0.4248$^{***}$ & 0.0080 & $-$0.6535$^{***}$ \\ 
  & (0.0719) & (0.0316) &  & (0.0193) & (0.0352) & (0.0816) \\ 
  & & & & & & \\ 
 CMA & $-$0.9131$^{***}$ & $-$0.1082$^{**}$ & 0.8673$^{***}$ &  & $-$0.1671$^{***}$ & 0.6385$^{***}$ \\ 
  & (0.0832) & (0.0446) & (0.0325) &  & (0.0405) & (0.0871) \\ 
  & & & & & & \\ 
 RMW & $-$0.4302$^{***}$ & $-$0.4749$^{***}$ & 0.0100 & $-$0.1021$^{***}$ &  & 0.1513 \\ 
  & (0.0654) & (0.0429) & (0.0434) & (0.0250) &  & (0.0932) \\ 
  & & & & & & \\ 
 Mom & $-$0.1147$^{***}$ & 0.0171 & $-$0.1814$^{***}$ & 0.0868$^{***}$ & 0.0337$^{*}$ &  \\ 
  & (0.0348) & (0.0203) & (0.0262) & (0.0121) & (0.0203) &  \\ 
  & & & & & & \\ 
\bottomrule \\[-1.8ex] 
\textit{Note:}  & \multicolumn{5}{c}{$^{*}$p$<$0.1; $^{**}$p$<$0.05; $^{***}$p$<$0.01} \\ 
\end{tabularx} 
\end{table}