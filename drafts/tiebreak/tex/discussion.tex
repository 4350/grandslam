%!TEX root = ../main.tex

\section{Discussion and conclusion} % (fold)
\label{sec:discussion_conclusion}

The first key finding of this thesis is that the classic value factor, HML, is still highly relevant for factor investors. 

Zero-cost regressions in the five-factor model suggest that HML may be fully explained by the remaining four factors, but we find evidence on the contrary. The actual implementation of mean-variance optimizations under the only constraint of non-negative weights gives a positive allocation to the HML factor. First, we have shown this using constant sample estimators as inputs in the mean-variance model on data 1963--2016. The sceptic reader will, however, object to the conclusion based on sample data alone, as a model is never better than its inputs; and sample estimates do not give a sense of optimal weights over time.

In finance, mean returns are very hard to predict and covariances follow predictable patterns, as volatility has tendencies to cluster and increase in bad times. We show that, in our data, there is clear evidence of non-normality and time-variation in the dependency across factors. A mean-variance optimization that is run solely on the basis of sample estimates will ignore these features of the data, and the positive allocation to HML might be an uninteresting solution to the optimization problem.

To mitigate these concerns, we also run the mean-variance optimization using dynamic inputs from a copula model, which can generate a conditional distribution of returns and captures both the non-normality and the time-variation in dependency. Under such inputs, the allocation to HML is even greater, indicating that the classic value factor is even more attractive than sample inputs suggest.

Although variance is the staple measure of risk, we also investigate whether risk beyond the first two moments can provide reasons against the HML factor. The copula model has allowed us to infer the full distribution of returns, and we shift our focus to the tail of the distribution. Here, we find that the diversification benefit of HML is at least as great, if not greater, than that of CMA. HML can by no means be considered a worse diversifier against tail risk, as measured by expected shortfall, than CMA.

We believe that the reason why zero-cost regressions indicate that HML does not add value is that the regressions are misspecified. The omission of an important sixth factor strategy, momentum, creates a bias on all the factor loadings, and leads to not recognizing the added value of HML. While the role of HML as a unique addition was already defended by \textcite{Asness2015}, we have formalized the argument and can conclude that optimizations lead to positive allocations to the HML strategy.

---

Our second key finding is that HML is highly similar, but not quite the same as CMA. On a closer look, differences emerge, and we believe that factor investors should combine both factors with consideration.

There is important theoretical and empirical support for a overlap in the stock positions that comprise HML and CMA, which results in a substantially higher correlation for this factor pair than for any other factor pair. The dependence between the two factors is also more stable, and does not exhibit the same non-normality and time-variation as do other factor pairs. 

Still, one factor cannot replace the other -- they do exhibit different properties, as HML firms are more profitable and exhibit less momentum, and generate return premia that mean-variance optimization suggests combining. Beyond the risk-reward of the first two moments, analysis of the full return distribution from the copula model shows that the diversification benefit of adding either HML or CMA to a portfolio is not constant over time. Most of the time, HML is the better diversifier, but in certain bad times, which do seem to coincide with stock market downturns, CMA is significantly better. We see no reason for investors to choose either one or the other, as both provide valuable diversification, and even better, they do so at different times.

However, we strongly believe that investors should consider the two factors jointly when building portfolios: When one of the two factors is included to a factor portfolio already containing the second, the first factor almost exclusively cannabilizes on the weight from the second. This makes us wary of factor weighting schemes that suggest equal-weights, and include both HML and CMA. 

---