%!TEX root = ../main.tex

\subsection{Copula Modeling of Dependency} % (fold)
\label{sub:05_04_copula}

\subsubsection{Copula Model Specification}

We now turn to modeling the dependency between factors using a copula model. 
Each week, the joint behavior of returns $R_{t+1}$ in the next week is modeled by a joint density function $f_t(R_{t+1})$. Assuming returns are multivariate normally distributed, this would correspond to the multivariate normal density function, fully parametrised by means, volatility and correlation matrix. But neither factor returns $R_t$, nor the standardized residuals $z_t$ are normally distributed.

Copulas are a convenient way of modeling dependency between non-normal returns. Following~\textcite{ChristoffersenErrunzaJacobLanglois2012}, who builds on~\textcite{Patton2006} and~\textcite{Sklar1959}, we decompose the joint density function into the product of a joint copula function $c_t$ of uniform variables $U_{t+1}$ and the marginal univariate distributions $f_{i,t}(r_{i, t+1})$:
\begin{align}
  f_t(R_{t+1}) &=
    c_t(U_{t+1}) \prod^N_{i = 1} f_{i,t}(r_{i, t + 1})
\end{align}
The marginal densities, $f_{i,t}(r_{i, t + 1})$, are modeled by ARMA-GARCH processes while the copula $c_t$ is a model of the joint behavior of their probability integral transforms. $c_t$ can be modeled, and estimated, separately from the marginal densities. This is key to imposing a more sophisticated dependency structure on the returns.

After ARMA-GARCH filtering, the marginal densities of returns are modeled by the constant density functions of standardized residuals. The vector of uniforms are therefore related to returns by the probability integral transform (PIT) of standardized residuals:
\begin{align}
  u_{i, t+1} = \int_{-\infty}^{z_{i,t+1}} f_{i}(z_{i,t+1})
\end{align}
In the most general case, we use a multivariate skewed Student's t \emph{dynamic asymmetric copula} model introduced by~\textcite{ChristoffersenErrunzaJacobLanglois2012}. The joint distribution is parametrised by a single degrees of freedom parameter $\nu_c$, an $N$ vector of skewness parameters $\gamma_{c}$ and a (time-varying) correlation matrix $\Psi_{t}$. We describe the details of the copula distribution in~\autoref{app:ghstmv}.

The normal and Student's t copula are nested in this model, as when $\gamma_{c,i} = 0$, we obtain a multivariate Student's t distribution, and if additionally $\nu_c = \infty$, we obtain the multivariate standard normal distribution.

The copula is made dynamic by evolving the correlation matrix $\Psi_t$ according to an underlying \emph{c}DCC process $Q_t$~\autocites[cf.]{Engle2002,Aielli2013}. Using the notation from~\textcite{ChristoffersenLanglois2013}:\footnote{The difference between ${z_t^*}$ and ${\bar{z}_t^*}$ is due to the \emph{corrected} DCC model; details in the appendix.}
\begin{align}
  Q_t &= (1 - \alpha - \beta) Q
    + \beta Q_{t - 1}
    + \alpha \bar{z}_{t - 1}^* \bar{z}_{t - 1}^{*\top}
  \label{eq:copula_cdcc}
\end{align}
where $Q_t$ is normalized to the correlation matrix
\begin{align}
  \Psi_t = Q_t^{-1/2} Q_t Q_t^{-1/2}
  \label{eq:copula_cdcc_psi}
\end{align}
The $Q_t$ process is comprised of three components that are weighted according to $\alpha, \beta$: (1) a time-invariant component $Q$, (2) an innovation component from copula shocks $\bar{z}_{t-1}^{*} \bar{z}_{t-1}^{*\top}$ and (3) an autoregressive component of order one $Q_{t-1}$. In order for the the correlation matrix $\Psi_t$ to be positive definite, $Q_t$ has to be positive definite, which is ascertained by requiring that $\alpha \geq 0$, $\beta \geq 0$ and $(\alpha + \beta) < 1$. The model nests a constant copula by forcing $\alpha = \beta = 0$.

% XXX NOTATION
% XXX cDCC correction for copula shocks
The parameters of the copula model -- the distribution parameters of the multivariate skewed Student's t distribution and the dynamics parameters of the \emph{c}DCC -- are estimated by maximizing the log-likelihood:
\begin{align}
  \arg\!\max_{\nu_c, \gamma_{ic}, \alpha, \beta} \sum_{t = 1}^T \ln c_t(U_t; \nu_c, \gamma_{ic}, \alpha, \beta)
\end{align}
This estimation takes the uniform residuals from each GARCH model as given. The process of copula estimation with \emph{c}DCC dynamics is quite involved. A detailed description can be found~\autoref{app:copula_cdcc}.

Clearly, the interpretation of the copula parameters is closely associated to the structure of multivariate dependence. By different restrictions on the parameters in the DAC model, we are able to activate or deactivate certain features of the copula: First, the degree of freedom parameter $nu_c$ is to be interpreted as a the measure of tail dependency. When $nu \neq 0$, the lower and upper tails of the joint distribution are fatter than in the normal case, which is coherent with the evidence from threshold correlations in. Second, the skewness parameters $\gamma_{c,i}$ are to be interpreted as the extent of asymmetry in the correlation structure. When $\gamma \neq 0$, there is asymmetry in correlations, which is also coherent with the earlier threshold correlation analysis. Third, the $\alpha$ and $\beta$ parameters determine whether the copula generates time-varying correlations. If $\alpha \neq 0$ and $\beta \neq 0$, the copula is dynamic, which is consistent with the findings of the rolling correlation analysis

\subsubsection{Copula Estimation Results}

We estimate constant and dynamic normal, symmetric and asymmetric copula models on the full dataset of GARCH uniform residuals; results are in~\autoref{tab:copula_estimation}. Looking at the parameter estimates, few of the $\gamma_c$ estimates appear significant, however, $\nu_c$ is clearly not infinite. Additionally, there is little improvement in log-likelihood by going from a symmetric to asymmetric copula. We conclude that the symmetric Student's t copula model is preferred. The insignificance of $\gamma$ can be interpreted as evidence of low asymmetries in the dependency -- or, more likely, that the model is simply unable to capture it.

There is a significant improvement in log-likelihood by going from a constant to dynamic copula, which suggests that time-varying tail dependency is an important feature to capture. The persistence, $\alpha + \beta$ is close to one, which could suggest that the dependency structure of factors is not stationary. We now turn to investigating how well the copula reproduces the dependency patterns observed in the data.

%!TEX root = ../../main.tex

\begin{table}[!ht]
  \centering
  \scriptsize
  \renewcommand{\arraystretch}{1.2}

  \caption{Parameter estimates for copula models based on uniform residuals from ARMA-GJR-GARCH models.\\ \quad \\
  Stationary bootstrap standard errors in parentheses, following Politis and Romano (1994). Copula parameters: $\nu_c$ is the degree of freedom, $\gamma_c$ is the vector of skewness parameters, $\alpha$, $\beta$ are the shock loading and autoregressive loading of the cDCC process. The significance test of $\nu_c$ is based on $1/\nu_c$, as this ratio goes to zero when $\nu_c$ goes to infinity (normality). Sample: 1963-07-05--2016-07-01.}
  \begin{tabularx}{\textwidth}{@{}l ddd X ddd @{}}
    \toprule
    &
      \multicolumn{3}{c}{Constant Copula} &&
      \multicolumn{3}{c}{Dymamic Copula} \\
    \cmidrule{2-4} \cmidrule{6-8}
    &
      \multicolumn{1}{c}{Normal} & \multicolumn{1}{c}{Symmetric \emph{t}} & \multicolumn{1}{c}{Skewed \emph{t}} & &
      \multicolumn{1}{c}{Normal} & \multicolumn{1}{c}{Symmetric \emph{t}} & \multicolumn{1}{c}{Skewed \emph{t}} \\
    \midrule
    $\nu_c$ & & 6.625^{**} & 6.671^{**} && & 11.936^{**} & 11.881^{**} \\
    & & (0.636) & (0.264) && & (0.770) & (0.641) \\
    \\
    $\gamma_\text{Mkt}$ & & & -0.057 && & & -0.078 \\
    & & & (0.047) && & & (0.062) \\
    \\
    $\gamma_\text{HML}$ & & & 0.103 && & & 0.083 \\
    & & & (0.036) && & & (0.071) \\
    \\
    $\gamma_\text{SMB}$ & & & -0.103 && & & -0.175 \\
    & & & (0.055) && & & (0.098) \\
    \\
    $\gamma_\text{Mom}$ & & & -0.202^{**} && & & -0.145 \\
    & & & (0.032) && & & (0.073) \\
    \\
    $\gamma_\text{RMW}$ & & & 0.021 && & & 0.095 \\
    & & & (0.035) && & & (0.058) \\
    \\
    $\gamma_\text{CMA}$ & & & 0.076 && & & 0.001 \\
    & & & (0.038) && & & (0.050) \\
    \\
    $\alpha$ & & & && 0.065 & 0.068^{**} & 0.068^{**} \\
    & & & && (0.006) & (0.006) & (0.006) \\
    \\
    $\beta$ & & & && 0.915 & 0.913^{**} & 0.913^{**} \\
    & & & && (0.008) & (0.007) & (0.007) \\
    \midrule
    Log-likelihood & 1169.194 & 1555.683 & 1572.672 && 2790.618 & 2977.648 & 2989.273 \\
    No. of Parameters & 15 & 16 & 22 && 17 & 18 & 24 \\
    % BIC & -348.32 & -122.21 & -316.432 && -243.221 & -342.342 & -396.324 \\
    Persistence & & & && 0.981 & 0.981 & 0.981 \\
    \bottomrule
  \end{tabularx}

  \label{tab:copula_estimation}
\end{table}



% subsection copula_model (end)
