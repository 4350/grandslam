%!TEX root = ../main.tex

\subsection{Copula Modeling of Dependency} % (fold)
\label{sub:05_04_copula}

\subsubsection{Copula Model Specification}

We now turn to modeling the dependency between factors using a copula model. The joint behavior of returns at each point in time is described by their conditional joint density function $f_t(r_{1, t+1}, \ldots, r_{N, t+1})$. Assuming returns are multivariate normally distributed, this would correspond to the multivariate normal density function which is fully parametrised by the mean, volatility and correlation matrix. But neither factor returns $r_{i,t}$, or standardized returns $z_{i,t}$, appear to be normally distributed and thus display patterns of dependency not reflected by a multivariate normal distribution.

Copulas are a parsimonious way of modeling dependency between non-normal returns. Following~\textcite{ChristoffersenErrunzaJacobLanglois2012}, who builds on~\textcite{Patton2006,Sklar1959}, we can decompose the conditional joint density function into the product of a joint copula function $c_t$ of uniform variables $u_{i,t}$ and the marginal univariate distributions $f_{i,t}(r_{i, t+1})$:
\begin{align}
  f_t(r_{1, t+1}, \ldots, r_{N, t+1}) &=
    c_t(u_{1, t+1}, \ldots, u_{N, t+1}) \prod^N_{i = 1}
    f_{i,t}(r_{i, t + 1})
\end{align}
The uniforms $u_{i, t+1}$ are related to the returns by the probability integral transform (PIT):
\begin{align}
  u_{i, t+1} = F_{i,t}(r_{i, t+1}) =
    \int_{-\infty}^{r_{i, t + 1}} f_{i,t}(r_{i, t+1})
\end{align}
In a ARMA-GARCH copula model, the univariate ARMA-GARCH processes describe the conditional densities $f_{i,t}(r_{i, t+1})$, while the copula $c_t$ is a model of the joint behavior of their probability integral transforms. $c_t$ can be modeled, and estimated, separately from the marginal densities. This is key to imposing a more sophisticated dependency structure on the returns.

In the most general case, we use a multivariate skewed Student's t \emph{dynamic asymmetric copula} model introduced by~\textcite{ChristoffersenErrunzaJacobLanglois2012}. The joint distribution is parametrised by a single degrees of freedom parameter $\nu_c$, $N$ skewness parameters $\gamma_{c,i}$ and a (time-varying) correlation matrix $\Psi_{t}$ (details about the joint density function $c_t$ are in~\autoref{app:ghstmv}):
\begin{align}
  c_t(u_{1, t+1}, \ldots, u_{N, t+1}) = t_{\nu_c,\gamma_{c,1},\ldots,\gamma_{c,N},\Psi_t}(z_{1, t+1}^*, \ldots, z_{N, t+1}^*)
\end{align}
$t_{\nu_c,\gamma_{c,1},\ldots,\gamma_{c,N},\Psi_t}$ is the copula's joint density function of \emph{copula shocks} $\{z_t^*\}$ which are constructed from the uniforms by the copula's inverse distribution functions $F_{\nu_c,\gamma_i}$:
\begin{align}
  z_{i, t+1}^* = F_{\nu_c,\gamma_i}^{-1}(u_{i, t+1})
\end{align}
As~\textcite{ChristoffersenErrunzaJacobLanglois2012}, we note that $F^{-1}_{\nu_c,\gamma_i}$ does \emph{not} describe the distribution of standardized residuals ${z_t}$ but of copula residuals ${z_t^*}$.

The normal and Student's t copula are nested in this model, as when $\gamma_{c,i}$, we obtain a multivariate Student's t distribution, and if additionally $\nu_c = \infty$, we obtain the multivariate standard normal distribution. Furthermore, the copula is made dynamic by evolving the correlation matrix $\Psi_t$ according to an underlying \emph{c}DCC process $Q_t$~\autocites[cf.]{Engle2002,Aielli2013}. Using the notation from~\textcite{ChristoffersenLanglois2013}:\footnote{The difference between ${z_t^*}$ and ${\bar{z}_t^*}$ is due to the \emph{corrected} DCC model; details in the appendix.}
\begin{align}
  Q_t &= (1 - \alpha - \beta) Q
    + \beta Q_{t - 1}
    + \alpha \bar{z}_{t - 1}^* \bar{z}_{t - 1}^{*\top}
\end{align}
where $Q_t$ is normalized to the correlation matrix
\begin{align}
  \Psi_t = Q_t^{-1/2} Q_t Q_t^{-1/2}
\end{align}
The $Q_t$ process is comprised of three components that are weighted according to $\alpha, \beta$: (1) a time-invariant component $Q$, (2) an innovation component from copula shocks $\bar{z}_{t-1}^{*} \bar{z}_{t-1}^{*\top}$ and (3) an autoregressive component of order one $Q_{t-1}$. In order for the the correlation matrix $\Psi_t$ to be positive definite, $Q_t$ has to be positive definite, which is ascertained by requiring that $\alpha \geq 0$, $\beta \geq 0$ and $(\alpha + \beta) < 1$. The model nests a constant copula by forcing $\alpha = \beta = 0$.

% XXX NOTATION
% XXX cDCC correction for copula shocks
The parameters of the copula model -- the distribution parameters of the multivariate skewed Student's t distribution and the dynamics parameters of the \emph{c}DCC -- are estimated by maximizing the log-likelihood:
\begin{align}
  \arg\!\max_{\nu_c, \gamma_{ic}, \alpha, \beta} \sum_{t = 1}^T \ln c_t(u_{1, t+1}, \ldots, u_{N, t+1}; \nu_c, \gamma_{ic}, \alpha, \beta)
\end{align}
This estimation takes the uniform residuals from each GARCH model as given. The process of copula estimation with \emph{c}DCC dynamics is quite involved. A detailed description can be found~\autoref{app:copula_cdcc}.

Clearly, the interpretation of the copula parameters is closely associated to the structure of multivariate dependence. By different restrictions on the parameters in the DAC model, we are able to activate or deactivate certain features of the copula: First, the degree of freedom parameter $nu_c$ is to be interpreted as a the measure of tail dependency. When $nu \neq 0$, the lower and upper tails of the joint distribution are fatter than in the normal case, which is coherent with the evidence from threshold correlations in. Second, the skewness parameters $\gamma_{c,i}$ are to be interpreted as the extent of asymmetry in the correlation structure. When $\gamma \neq 0$, there is asymmetry in correlations, which is also coherent with the earlier threshold correlation analysis. Third, the $\alpha$ and $\beta$ parameters determine whether the copula generates time-varying correlations. If $\alpha \neq 0$ and $\beta \neq 0$, the copula is dynamic, which is consistent with the findings of the rolling correlation analysis

\subsubsection{Copula Estimation Results}

We estimate constant and dynamic normal, symmetric and asymmetric copula models on the full dataset of GARCH uniform residuals; results are in~\autoref{tab:copula_estimation}. Looking at the parameter estimates, few of the $\gamma_c$ estimates appear significant, however, $\nu_c$ is clearly not infinite. Additionally, there is little improvement in log-likelihood by going from a symmetric to asymmetric copula. We conclude that the symmetric Student's t copula model is preferred. The insignificance of $\gamma$ can be interpreted as evidence of low asymmetries in the dependency -- or, more likely, that the model is simply unable to capture it.

There is a significant improvement in log-likelihood by going from a constant to dynamic copula, which suggests that time-varying tail dependency is an important feature to capture. The persistence, $\alpha + \beta$ is close to one, which could suggest that the dependency structure of factors is not stationary. We now turn to investigating how well the copula reproduces the dependency patterns observed in the data.

%!TEX root = ../../main.tex

\begin{table}[p]
  \centering
  \footnotesize
  \renewcommand{\arraystretch}{1.2}

  \caption{Copula parameter estimates (1963--2016)}

  \begin{longcaption}
    Models from~\autoref{eq:copula_cdcc} on $N = 2,766$ weekly standardized residuals from the ARMA-GARCH models in~\autoref{tab:garch_estimation}. Stationary bootstrap standard errors in parentheses, following~\autocite{PolitisRomano1994}. $nu_c$ is the degree of freedom parameter and $gamma_i$ are the copula skewness parameters. $\alpha, \beta$ control the correlation dynamics, and Persistence is $\alpha + \beta$. Elements of $\hat{Q}$ are estimated using moment matching (see~\autoref{app:copula_cdcc}). The constant copula is achieved by forcing $\alpha = \beta = 0$. For each model, there are 15 parameters in the $Q$ time-invariant correlation matrix, which are not reported in the table. Significance given by $^{*}p<10\%$; $^{**}p<5\%$; $^{***}p<1\%$
  \end{longcaption}

  \begin{tabularx}{\textwidth}{@{}l ddd X ddd}
    \toprule
    &
      \multicolumn{3}{c}{Constant Copula} &&
      \multicolumn{3}{c}{Dymamic Copula} \\
    \cmidrule{2-4} \cmidrule{6-8}
    &
      \multicolumn{1}{c}{Normal} & \multicolumn{1}{c}{Symmetric \emph{t}} & \multicolumn{1}{c}{Skewed \emph{t}} & &
      \multicolumn{1}{c}{Normal} & \multicolumn{1}{c}{Symmetric \emph{t}} & \multicolumn{1}{c}{Skewed \emph{t}} \\
    \midrule
    $\nu_c$                & & 6.61^{***} & 6.65^{***}  & &             & 11.77^{***} & 11.63^{***} \\
                          & & (0.89)     & (0.10)      & &             & (0.89)      & (0.10) \\
    % $1/\nu_c$             & & 0.15^{***} & 0.15^{***}  & &             & 0.09^{***} & 0.09^{***} \\
    %                       & & (0.01)     & (0.01)      & &             & (0.01)      & (0.01) \\
    \\
    $\gamma_\text{Mkt}$ & &             & -0.06       & &             &              & -0.05 \\
                        & &             & (0.07)      & &             &              & (0.05) \\
    \\
    $\gamma_\text{SMB}$ & &             & -0.11^{*}   & &             &              & -0.14^{**} \\
                        & &             & (0.06)      & &             &              & (0.06) \\
    \\
    $\gamma_\text{Mom}$ & &             & -0.20^{***} & &             &              & -0.12 \\
                        & &             & (0.07)      & &             &              & (0.07) \\
    \\
    $\gamma_\text{HML}$ & &             & 0.10        & &             &              & -0.02 \\
                        & &             & (0.07)      & &             &              & (0.06) \\
    \\
    $\gamma_\text{CMA}$ & &             & 0.08        & &             &              & -0.05 \\
                        & &             & (0.06)      & &             &              & (0.07) \\
    \\
    $\gamma_\text{RMW}$ & &             & 0.02        & &             &              & 0.18^{**} \\
                        & &             & (0.07)      & &             &              & (0.07) \\
    \\
    
    $\alpha$            & &             &              & & 0.07^{***} & 0.07^{***}  & 0.07^{***} \\
                        & &             &              & & (0.01)     & (0.01)      & (0.01) \\
    \\
    $\beta$             & &             &              & & 0.91^{***} & 0.91^{***}  & 0.91^{***} \\
                        & &             &              & & (0.01)     & (0.01)      & (0.01) \\
    \midrule
    \multicolumn{8}{@{}l}{\textbf{Log-likelihood (LLH), Number of parameters (\# params.), BIC and Correlation Persistence (CP)}} \\
    LLH &
      \multicolumn{1}{D{.}{.}{3}}{1,169} & 
      \multicolumn{1}{D{.}{.}{3}}{1,555} & 
      \multicolumn{1}{D{.}{.}{3}}{1,573} & &
      \multicolumn{1}{D{.}{.}{3}}{2,790} & 
      \multicolumn{1}{D{.}{.}{3}}{2,983} & 
      \multicolumn{1}{D{.}{.}{3}}{2,995} \\
    \# params. &
      \multicolumn{1}{D{.}{.}{3}}{15} & 
      \multicolumn{1}{D{.}{.}{3}}{16} & 
      \multicolumn{1}{D{.}{.}{3}}{22} & &
      \multicolumn{1}{D{.}{.}{3}}{17} & 
      \multicolumn{1}{D{.}{.}{3}}{18} & 
      \multicolumn{1}{D{.}{.}{3}}{24} \\
    BIC &
      \multicolumn{1}{D{.}{.}{3}}{-2,337} & 
      \multicolumn{1}{D{.}{.}{3}}{-3,103} & 
      \multicolumn{1}{D{.}{.}{3}}{-3,090} & &
      \multicolumn{1}{D{.}{.}{3}}{-5,564} & 
      \multicolumn{1}{D{.}{.}{3}}{-5,943} & 
      \multicolumn{1}{D{.}{.}{3}}{-5,919} \\
    CP (\%) & & & && 97.73 & 98.01 & 97.98 \\
    \bottomrule
  \end{tabularx}

  \label{tab:copula_estimation}
\end{table}



% subsection copula_model (end)
