%!TEX root = ../main.tex

\subsection{Choosing a multivariate model}
\label{sub:05_01_choosing}
% Why copula?
The ARMA-GARCH family of models has become the norm of modeling univariate financial return series, beginning with \textcite{Bollerslev1986}. The straightforward extension of univariate GARCH models to multiple return series has, however, proven difficult. Unrestricted multivariate GARCH (MGARCH) models that model the conditional covariance matrix directly become impossible to estimate as the number of covariances grows exponentially with the number of series. It thus becomes necessary to restrict the parameter space, where the BEKK model is a common example~\autocite{BEKKModel}.

% Separate modeling of variance and correlation
A parsimonious solution to the dimensionality problem is to separate the modeling of return and volatility dynamics from the modeling of conditional correlations. The separation allows for consistent (albeit inefficient) two-step estimation, and makes large-scale estimation feasible. One such approach is the \emph{dynamic conditional correlation} (DCC) model originally proposed by~\autocite{Engle2002}. In the DCC model, univariate GARCH models are first estimated on each series. Then, an autoregressive process for the correlation matrix is fitted to the standardized residuals ${z_t}$ of those models. 

% Asymmetric cDCC and copula cDCC
DCC is a useful and tractable model for estimating time-varying correlations between return series. However, it is a model of correlations only and is not flexible enough to model the univariate components differently. More specifically, it is not constructed to generate tail dependence, which is the notion that correlation dynamics can be very different in extreme realizations. 

% Enter the copula
Copula models have recently attracted much attention in the field of risk management, as they provide a flexible way to infer a multivariate probability distribution. Copula models are, just like DCC models, based on two-step estimation and work well in large scale applications. Furthermore, copulas are flexible enough to generate tail dependence, which is shown to be an important feature of factor return data~\autocite{ChristoffersenLanglois2013}. 

Copula models are most often constructed by estimating univariate models from the ARMA-GARCH family in the first step. The residuals from the ARMA-GARCH models are then used in the copula function that explains the multivariate dependence, including dynamic correlations and tail dependence.

Among copula models, there are three main routes of interest: (1) Archimedean copulas, (2) multivariate normal and Student's \textit{t} copulas, and (3) vine copulas. While Archimedean copulas, such as the Gumbel and Clayton specifications, are attractive in many settings, they fail to generate both threshold correlations and simultaneously high threshold correlations, as is often the case in factor return series \textcite{ChristoffersenLanglois2013}. Vine copulas, or pairwise copula constructions, are made up of a combination of bivariate copulas in a tree structure (hence the name vine copulas), and pose an interesting alternative to multivariate normal and Student's \textit{t} copulas. However, the vine method is far less parsimonious as both the number of bivariate combinations and the number of different tree structures increase exponentially with the number of assets modeled \autocite{Aas2009}.

We choose to work with multivariate normal and Student's \textit{t} copula models, as they can (1) estimate the joint distribution function in large scale applications, (2) model different univariate models for the different factors, and (3) incorporate both tail dependence and dynamic correlations. Next, we define and describe the copula model.