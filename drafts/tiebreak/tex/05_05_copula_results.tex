%!TEX root = ../main.tex
\subsection{Copula specification and estimation results}
Given the results of the dependence structure of residuals, we now discuss the best choice of copula model and present estimation results of the six competing copula specifications.

\subsubsection{Interpreting and choosing copula specification}

The interpretation of the copula parameterization is closely associated to the structure of multivariate dependence. By different restrictions on the parameters in the DAC model, we are able to activate or deactivate certain features of the copula: First, the degree of freedom parameter $nu_c$ is to be interpreted as a the measure of tail dependency. When $nu \neq 0$, the lower and upper tails of the joint distribution are fatter than in the normal case, which is coherent with the evidence from threshold correlations. Second, the skewness parameters $\gamma_{c,i}$ are to be interpreted as the extent of asymmetry in the correlation structure. When $\gamma \neq 0$, there is asymmetry in correlations, which is also coherent with the earlier threshold correlation analysis. Third, the $\alpha$ and $\beta$ parameters determine whether the copula generates time-varying correlations. If $\alpha \neq 0$ and $\beta \neq 0$, the copula is dynamic, which is consistent with the findings of the rolling correlation analysis. An overview of the six copula models is given in \autoref{fig:conceptual}.

%!TEX root=../../main.tex

\begin{table}
  \centering
  \footnotesize
  \renewcommand{\arraystretch}{1.2}

  \caption{Conceptual matrix of copula parameterizations}

  \begin{tabularx}{0.80\textwidth}{@{} lc c >{\centering}Xc >{\centering}Xc >{\centering\arraybackslash}X}
    \toprule
      & && \textbf{Normal} && \textbf{Symmetric \emph{t}} && \textbf{Skewed \emph{t}} \\
      \cmidrule{4-4}
      \cmidrule{6-6}
      \cmidrule{8-8}
      & && $\nu_c = \infty$   && $\nu_c < \infty$   && $\nu_c < \infty$ \\
      & && $\gamma_{i,c} = 0$ && $\gamma_{i,c} = 0$ && $\gamma_{i,c} \neq 0$ \\
      \cmidrule{4-8}
    \cmidrule{1-2}
    \multirow{2}{*}{\textbf{Constant}} & $\alpha = 0$ && Constant && Constant && Constant \\
                              & $\beta = 0$  && Normal   && Symmetric \emph{t} && Skewed \emph{t}      \\
    \cmidrule{1-2}
    \multirow{2}{*}{\textbf{Dynamic}}  & $\alpha > 0$ && Dynamic  && Dynamic && Dynamic \\
                              & $\beta > 0$  && Normal   && Symmetric \emph{t} && Skewed \emph{t}      \\
    \bottomrule
  \end{tabularx}
% ()
%   \begin{tabularx}{\textwidth}{@{\extracolsep{5pt}} c c c c X c X c @{}}
%     \toprule
%   				&			& &	\textbf{Normal}	&	&	\textbf{Student's \textit{t}}	&	&	\textbf{Asymmetric Student's \textit{t}} \\
%   				\\
%   				&			& & 	$\nu = \infty$	&	&	$\nu > 0$	& 	&	$\nu > 0$ \\
%   				&			& & 	$\gamma = 0$	&	&	$\gamma = 0$	& 	&	$\gamma \neq 0$ \\
%           \\
%     \cmidrule{4-8}
%     \\
%      \textbf{Constant} &  & & \text{Constant normal copula} & & \text{Constant symmetric \textit{t} copula} & & \text{Constant asymmetric \textit{t} copula} \\
%      \\
%     	&	$\alpha = 0$  &		&	\textit{Constant correlations but} & & \textit{Constant correlations and} & & \textit{Constant correlations and} \\
%        & $\beta = 0$ & & \textit{no tail dependence} & & \textit{symmetric tail dependence} & & \textit{asymmetric tail dependence} \\
%     \\
%     \cmidrule{4-8}
%     \\
%      \textbf{Dynamic} &  & & \text{Dynamic normal copula} & & \text{Dynamic symmetric \textit{t} copula} & & \text{Dynamic asymmetric \textit{t} copula} \\
%      \\
%       & $\alpha > 0$  &   & \textit{Dynamic correlations but} & & \textit{Dynamic correlations and} & & \textit{Dynamic correlations and} \\
%        & $\beta > 0$ & & \textit{no tail dependence} & & \textit{symmetric tail dependence} & & \textit{asymmetric tail dependence} \\
%     \\
%     \bottomrule
%   \end{tabularx}

  \label{tab:conceptual}	
\end{table}


\subsubsection{Copula estimation results}

We estimate constant and dynamic normal, symmetric and asymmetric copula models on the full dataset of GARCH uniform residuals. Results are presented in~\autoref{tab:copula_estimation}. First, we examine the choice between a normal, symmetric \textit{t} or asymmetric \textit{t} copula. We note that $\nu_c$ is clearly significant and suggests a Student's \textit{t} model with tail dependence over the normal model. Second, we examine the asymmetric specification and find that few of the $\gamma_c$ estimates appear significant, indicating that the asymmetry is hard to capture or not consistent enough to merit modeling. This is supported by the relatively small improvement in log-likelihood by going from a symmetric to asymmetric copula and the fact that the BIC criterion prefers the symmetric model in the dynamic case. 

Second, we examine the choice between a constant and dynamic copula correlation matrix. There is a significant improvement in log-likelihood and BIC when moving from a constant to a dynamic copula, which suggests that time-varying dependence shown by rolling correlation is captured, which improves the model's fit. We also find a high persistence of the correlation process, as $\alpha + \beta$, is close to a unit root.

In summary, we find that the dynamic symmetric Student's \textit{t} copula is the best specification, as it has the lowest BIC, well defined parameters, and is strongly supported by the dependence pattern showcased by threshold and rolling correlation analyses. While the asymmetric Student's \textit{t} is an interesting model, we believe that the asymmetry patterns in data are too irregular to capture well in a copula model with only one asymmetry parameter for each series (this is further discussed in the subsequent robustness discussion, see \autoref{sub:05_robust}).