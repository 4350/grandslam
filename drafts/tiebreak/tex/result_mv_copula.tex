%!TEX root = ../main.tex

\section{Modeling Multivariate Dependence: Copula Approach} % (fold)
\label{sec:modeling_multivariate_dependence_copula_approach}

\subsection{Multivariate Model Selection} % (fold)
\label{sub:multivariate_model_selection}

% Dimensionality of MGARCH
Accurate modeling of the joint dynamics of returns is an important and well-studied problem in finance. Unfortunately, straightforward extension of univariate GARCH models to multiple dimensions has proven difficult. Unrestricted multivariate GARCH (MGARCH) models, modeling the conditional covariance matrix directly, become impossible to estimate as the number of covariances grows exponentially with the number of series~\autocite{WhyMGARCHSucks}. It thus becomes necessary to restrict the parameter space, where the BEKK model is a common example~\autocite{BEKKModel}.

% Separate modeling of variance and correlation
A parsimonious solution is to separate the modeling of return and variance dynamics (modeled using ARMA-GARCH) from that of the conditional correlation dynamics. One such approach is encapsulated by the \emph{dynamic conditional correlation} (DCC) model originally proposed by~\autocite{Engle2002} (and its correction \emph{c}DCC by~\autocite{Aielli2013}). The separation allows for consistent (albeit inefficient) two-step estimation. First, univariate GARCH models are estimated on each series (as we do). Second, an autoregressive process for the correlation matrix is fitted to the standardized residuals $\varepsilon_t$ of those models. The separation makes large-scale estimation feasible.

% Asymmetric cDCC
DCC is a useful and tractable model for estimating time-varying correlations between return series. However, as it is a model of correlations only, it cannot capture the asymmetric tail dependency seen in the threshold correlations of residuals. Introducing indicator variables for the strength of correlation, similar to how the leverage effect is captured by GJR-GARCH, is a potential solution (as in the AGDCC model by~\autocite{Cappiello2006}).

% Copula cDCC
Our model for the dependency of factors is the copula version of \emph{c}DCC model called \emph{dynamic asymmetric copula} (DAC), introduced in~\autocite{ChristoffersenErrunzaJacobLanglois2012}. In contrast to fitting the \emph{c}DCC process to standardized residuals directly, we fit a \emph{c}DCC model to a transformation of these shocks. The theoretical motivation behind this transformation is to jointly model returns with separate distributions (e.g. normal and skewed Student's t) in a single framework. The copula approach has the added benefit of separating modeling of the returns themselves from that of the dependency structure.

% subsection multivariate_model_selection (end)

\subsection{Modeling dependence with the DAC model} % (fold)
\label{sub:copula}

% How copula?
Following~\autocite{ChristoffersenErrunzaJacobLanglois2012}, who builds on~\autocite{Patton2006,Sklar1959}, we can decompose the conditional joint density of returns $f_t(r_{1, t+1}, \ldots, r_{N, t+1})$ into the product of a joint copula function $c_t$ of uniform variables $u_{i,t}$ and the marginal univariate distributions $f_{i,t}(r_{i, t+1})$:
\begin{align}
  f_t(r_{1, t+1}, \ldots, r_{N, t+1}) &=
    c_t(u_{1, t+1}, \ldots, u_{N, t+1}) \prod^N_{i = 1}
    f_{i,t}(r_{i, t + 1})
\end{align}
The uniforms $u_{i, t+1}$ are related to the returns by the probability integral transform (PIT):
\begin{align}
  u_{i, t+1} = F_{i,t}(r_{i, t+1}) =
    \int_{-\infty}^{r_{i, t + 1}} f_{i,t}(r_{i, t+1})
\end{align}
In a ARMA-GARCH copula model, the univariate ARMA-GARCH processes describe the conditional densities $f_{i,t}(r_{i, t+1})$, while the copula $c_t$ is a model of the joint behavior of their probability integral transforms. The key benefit of this approach is that $c_t$ is independent of the marginal distributions, which allows separate modeling and estimation. Two-step quasi maximum likelihood, also known as inference-by-margins (IFM) introduced by~\autocite{Joe1997}, first estimates the marginal distributions and subsequently the copula distribution, taking the marginal densities as given. Because the joint density is separate from the marginal densities, we can use different assumptions for each component.

In the most general case, we use a multivariate skewed Student's t DAC model introduced by~\autocite{ChristoffersenErrunzaJacobLanglois2012}. The joint distribution is parametrised by a single degrees of freedom parameter $\nu_c$, $N$ skewness parameters $\gamma_{c,i}$ and a (time-varying) correlation matrix $\Psi_{t}$ (details about the joint density function $c_t$ are in~\autoref{app:ghstmv}). The normal and Student's t copula are nested in this model, as when $\gamma_{c,i}$, we obtain a multivariate Student's t distribution, and if additionally $\nu_c = \infty$, we obtain the multivariate standard normal distribution.

% Interpretation of parameters \gamma, \nu and \Psi?

The copula is made dynamic by evolving the correlation matrix $\Psi_t$ according to an underlying \emph{c}DCC process $Q_t$~\autocites[cf.]{Engle2002,Aielli2013}. Using the notation from~\autocite{ChristoffersenLanglois2013}:
\begin{align}
  Q_t &= (1 - \alpha - \beta) Q
    + \beta Q_{t - 1}
    + \alpha z_{t - 1} z_{t - 1}^\top \\
  \intertext{where $Q_t$ is normalized to the correlation matrix}
  \Psi_t &= Q_t^{-1/2} Q_t Q_t^{-1/2}
\end{align}
The $Q_t$ process is comprised of three components that are weighted according to $\alpha, \beta$: (1) a time-invariant component $Q$, (2) an innovation component from copula shocks $z_{t-1} z_{t-1}^\top$ and (3) an autoregressive component of order one $Q_{t-1}$. In order for the the correlation matrix $\Psi_t$ to be positive definite, $Q_t$ has to be positive definite, which is ascertained by requiring that $\alpha \geq 0$, $\beta \geq 0$ and $(\alpha + \beta) < 1$. The model nests a constant copula by forcing $\alpha = \beta = 0$.

% XXX NOTATION
% XXX cDCC correction for copula shocks

The parameters of the copula model -- the distribution parameters of the multivariate skewed Student's t distribution and the dynamics parameters of the cDCC -- are estimated by maximizing the log-likelihood:
\begin{align}
  \arg\!\max_{\nu_c, \gamma_{ic}, \alpha, \beta} \sum_{t = 1}^T \ln c_t(u_{1, t+1}, \ldots, u_{N, t+1}; \nu_c, \gamma_{ic}, \alpha, \beta)
\end{align}
The process of copula estimation with \emph{c}DCC dynamics is quite involved. A detailed description can be found~\autoref{app:copula_cdcc}.

% subsection copula (end)

% section modeling_multivariate_dependence_copula_approach (end)
