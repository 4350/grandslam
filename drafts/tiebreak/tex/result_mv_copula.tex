%!TEX root = ../main.tex

\section{Modeling dependence in ARMA-GARCH residuals: The DAC model} % (fold)
\label{sec:dac_copula}

Analysis of the dependence between univariate residual series from the ARMA-GARCH models led to the conclusion that there exist tail dependency and time-varying dependency between factor strategies, which can be captured by a DAC copula model. We begin by describing the copula specification and the estimation procedure. Then, we explain how different parameterizations of the copula corresponds to different features of the dependence structure. Last, we analyze the estimation results.

% How copula?
Following~\autocite{ChristoffersenErrunzaJacobLanglois2012}, who builds on~\autocite{Patton2006,Sklar1959}, we can decompose the conditional joint density of returns $f_t(r_{1, t+1}, \ldots, r_{N, t+1})$ into the product of a joint copula function $c_t$ of uniform variables $u_{i,t}$ and the marginal univariate distributions $f_{i,t}(r_{i, t+1})$:
\begin{align}
  f_t(r_{1, t+1}, \ldots, r_{N, t+1}) &=
    c_t(u_{1, t+1}, \ldots, u_{N, t+1}) \prod^N_{i = 1}
    f_{i,t}(r_{i, t + 1})
\end{align}
The uniforms $u_{i, t+1}$ are related to the returns by the probability integral transform (PIT):
\begin{align}
  u_{i, t+1} = F_{i,t}(r_{i, t+1}) =
    \int_{-\infty}^{r_{i, t + 1}} f_{i,t}(r_{i, t+1})
\end{align}
In a ARMA-GARCH copula model, the univariate ARMA-GARCH processes describe the conditional densities $f_{i,t}(r_{i, t+1})$, while the copula $c_t$ is a model of the joint behavior of their probability integral transforms. The key benefit of this approach is that $c_t$ is independent of the marginal distributions, which allows separate modeling and estimation. Two-step quasi maximum likelihood, also known as inference-by-margins (IFM) introduced by~\autocite{Joe1997}, first estimates the marginal distributions and subsequently the copula distribution, taking the marginal densities as given. Because the joint density is separate from the marginal densities, we can use different assumptions for each component.

In the most general case, we use a multivariate skewed Student's t DAC model introduced by~\autocite{ChristoffersenErrunzaJacobLanglois2012}. The joint distribution is parametrised by a single degrees of freedom parameter $\nu_c$, $N$ skewness parameters $\gamma_{c,i}$ and a (time-varying) correlation matrix $\Psi_{t}$ (details about the joint density function $c_t$ are in~\autoref{app:ghstmv}). The normal and Student's t copula are nested in this model, as when $\gamma_{c,i}$, we obtain a multivariate Student's t distribution, and if additionally $\nu_c = \infty$, we obtain the multivariate standard normal distribution. Furthermore, the copula is made dynamic by evolving the correlation matrix $\Psi_t$ according to an underlying \emph{c}DCC process $Q_t$~\autocites[cf.]{Engle2002,Aielli2013}. Using the notation from~\autocite{ChristoffersenLanglois2013}:
\begin{align}
  Q_t &= (1 - \alpha - \beta) Q
    + \beta Q_{t - 1}
    + \alpha z_{t - 1} z_{t - 1}^\top \\
  \intertext{where $Q_t$ is normalized to the correlation matrix}
  \Psi_t &= Q_t^{-1/2} Q_t Q_t^{-1/2}
\end{align}
The $Q_t$ process is comprised of three components that are weighted according to $\alpha, \beta$: (1) a time-invariant component $Q$, (2) an innovation component from copula shocks $z_{t-1} z_{t-1}^\top$ and (3) an autoregressive component of order one $Q_{t-1}$. In order for the the correlation matrix $\Psi_t$ to be positive definite, $Q_t$ has to be positive definite, which is ascertained by requiring that $\alpha \geq 0$, $\beta \geq 0$ and $(\alpha + \beta) < 1$. The model nests a constant copula by forcing $\alpha = \beta = 0$.

% XXX NOTATION
% XXX cDCC correction for copula shocks
The parameters of the copula model -- the distribution parameters of the multivariate skewed Student's t distribution and the dynamics parameters of the cDCC -- are estimated by maximizing the log-likelihood:
\begin{align}
  \arg\!\max_{\nu_c, \gamma_{ic}, \alpha, \beta} \sum_{t = 1}^T \ln c_t(u_{1, t+1}, \ldots, u_{N, t+1}; \nu_c, \gamma_{ic}, \alpha, \beta)
\end{align}
The process of copula estimation with \emph{c}DCC dynamics is quite involved. A detailed description can be found~\autoref{app:copula_cdcc}.

Clearly, the interpretation of the copula parameters is closely associated to the structure of multivariate dependence. By different restrictions on the parameters in the DAC model, we are able to activate or deactivate certain features of the copula: First, the degree of freedom parameter $nu_c$ is to be interpreted as a the measure of tail dependency. When $nu \neq 0$, the lower and upper tails of the joint distribution are fatter than in the normal case, which is coherent with the evidence from threshold correlations in \autoref{subsec:threshold_corr}. Second, the skewness parameters $gamma_{c_i}$ are to be interpreted as the extent of asymmetry in the correlation structure. When $\gamma \neq 0$, there is asymmetry in correlations, which is also coherent with the earlier threshold correlation analysis. Third, the $\alpha$ and $\beta$ parameters determine whether the copula generates time-varying correlations. If $\alpha \neq 0$ and $\beta \neq 0$, the copula is dynamic, which is consistent with the findings of the rolling correlation analysis (\autoref{subsec:roll_corr}). The possible parameter restrictions lead to six potential copula models, which are illustrated graphically in~\autoref{copula_infographic}.

Next, we examine the constant copula specifications and then proceed to the dynamic models.
% subsection copula (end)

% section modeling_multivariate_dependence_copula_approach (end)
