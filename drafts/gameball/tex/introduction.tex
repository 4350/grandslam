%!TEX root = ../main.tex
\section{Introduction}
Factor strategies have become popular not only because of their added alpha to holding the market, but because of the low correlation across factors, which provides diversification in
a portfolio setting. Although factor correlations are low in
general, they surge in bad times, when diversification is needed the most. Furthermore, there is evidence of asymmetric tail dependence between factor strategies.

Fama and French (2015) find that the value factor, HML, is redundant in a four-factor model including profitability (RMW) and investment (CMA), as it has no explanatory power on monthly returns
in a US 1963–2010 sample. This could mean that the classic value factor is an inferior proxy for what truly comprises value. We believe that the substantial amount of leveraged capital in the classic value strategy can be another reason to replace it with what we label modern value (RMW, CMA). The quant crash of 2007 highlighted the risks of crowded trades in factor strategies: During a short period between August 6 and August 9, equity hedge funds reported record losses in an otherwise calm market environment. Although losses were more frequent in quantitatively managed funds, also traditional hedge funds lost substantial amounts of money as value and size factors crashed in a liquidity spiral, fuelled by margin requirements of leveraged investors \autocite{KhandaniLo2011}.

This paper takes a risk perspective on factor strategies and investigates what role classic value (HML) has in factor investing, given the discovery of modern value (RMW, CMA). We investigate the effects of replacing classic value with modern value in a portfolio context. Our key hypotheses are:
\begin{enumerate}
	\item RMW and CMA offer greater diversification benefits than HML
	\item Out of sample, a mean-variance investor excluding HML from the investible universe will realize nearly the same Sharpe ratio as in the unconstrained case, but significantly decrease other risk measures including skewness and maximum drawdown
	\item The additional risk of crowded trades is driven by leveraged hedge fund capital
\end{enumerate}
The foundation for the hypotheses is the zero alpha of HML \autocite{FF2015} and the risk of crowded leveraged trades \autocite{Brunnermeier2009}. The intuition is the following: at times, factor strategies experience a negative return shock, which leads to margin calls for some leveraged investors. The margin calls lead to additional selling, for which there is not enough liquidity to maintain prices. In parallel, the return shock leads to tighter risk management, which in itself causes selling. The spiral continues as prices continue to drop and more investors reach risk limits and margin calls, until prices are pushed sufficiently far from equilibrium to attract other capital.

To answer these questions, we need a conditional model of factor return series that incorporates stylized facts of financial return data. In particular, we want the model to capture fat tails, volatility clustering and leverage effects at the marginal level and (asymmetric) tail dependency and time-varying dependency at the multivariate level. Recently, copula models have attracted considerable attention in the risk management field, as they offer a numerically feasible and flexible way of estimating joint probability distributions. A model of the joint return distribution allows us to evaluate our hypotheses in a portfolio context. First, to evaluate our first hypothesis, we consider how close the expected shortfall of different portfolios can get to the lower bound \autocite{ChristoffersenErrunzaJacobLanglois2012}. Second, to evaluate our second hypothesis, we exercise mean-variance optimal investing and compare the Sharpe ratio and risk measures of different portfolios.

We follow, relatively closely, the method for a skewed Student-\textit{t} copula in \textcite{ChristoffersenLanglois2013}. First, we fit ARMA-GARCH models to the marginal return series. Uniform transforms of residuals from the marginal models are then used to estimate the copula function, which gives us the joint return distribution. In this copula, the correlation matrix of the copula model is also allowed to vary over time, following a \textit{c}DCC model (\textcite{Engle2002}, \textcite{Aielli2013}). The \textit{c}DCC copula can be augmented by considering a trend regressor for the dependence structure, as in \textcite{ChristoffersenErrunzaJacobLanglois2012}, who investigate whether international diversification benefits are disappearing. Where they include a time trend, we propose to evaluate our third hypothesis by including the assets under management of equity-market-neutral hedge funds as a trend regressor. [NOT YET IMPLEMENTED]

As much of this paper depends on the quality of the copula model, we begin by careful examination of the multivariate dependence structure of the factor series. Rolling correlations show that linear correlations are time-varying, rejecting constant correlation models. Threshold correlations show that there is marked and asymmetric tail dependence across factors, rejecting any model based on only linear correlations and Gaussian assumptions. Diagnostic tests on our copula on the marginal and multivariate level are accepted. Simulations from our copula model then show that the empirical dependency pattern is fairly well generated. 

Having selected the best copula specification, we proceed to the main tests of our hypotheses. Diversification benefits are shown to improve when classic value is replaced by modern value, as the expected shortfall for modern value comes closer to its lower bound. [No conclusion for remaining two hypotheses, this analysis <is still pending]