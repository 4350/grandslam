%!TEX root = ../main.tex
\section{Literature review}
\textcite{FF2015} introduces two additional factors to complement the \textcite{FamaFrench1993} three-factor asset pricing model. In what is referred to as the five-factor model, the traditional factors (market, value and size) are complemented by a profitability factor and an investment factor. Both factors represent zero-cost portfolios: the profitability factor, denoted RMW (robust minus weak), is long firms with high operating profitability and short firms with low operating profitability, and the investment factor, denoted CMA (conservative minus aggressive), is long firms with low investment rate and short firms with high investment rate. The five-factor model is found to be a siginificant improvement over the three-factor model, but an even more important finding is that the new factors appear to have made the value factor, HML (high minus low), redundant. More specifically, the authors show that there is no significant intercept in regressions of HML on the remaining four factors, while each of the other factors have significant intercepts in similar regressions. The high average return of the HML factor is fully explained by the remaining four factors. In an investment context, this can be interpreted as the HML factor adding no alpha to a portfolio holding the remaining four factors.

Factor equity strategies have become increasingly popular, as they earn a premia over holding the market. It is debated whether the premia are rewards for bearing additional risk, or whether they are the consequence of market frictions, irrational investors, or both. For the value factor (HML), there are some appealing rational stories for the risk premia. \textcite{FamaFrench1993} show that the HML factor is related to systematic patterns of profitability and growth, and could proxy for a common risk source. This is supported by \textcite{LiewVassalou2000}, who show that the value factor can predict real GDP growth on data in several markets. \textcite{Zhang2005} uses a neoclassical model with rational expectations and competitive equilibrium to show that value firms have more tangible assets and are burdened by industry over-capacity in downturns, leading to higher down-market betas. \textcite{PetkovaZhang2005} find that the conditional betas of value stocks covary positively with the expected market risk premium.

Despite there being a number of rational theories, they all predict effects that are fairly small and cannot fully motivate the value premium. So far, the most pervasive explanations of the value premium are based on market imperfections and irrational behavior.\footnote{See i.a. \textcite{Ilmanen2011}}

An influential early paper by \textcite{LakonishokShleiferVishny1994} argues that the value factor is driven by investors' overreaction to changes in earnings. They show that value firms have often experieced a decline in earnings over the last three years, lowering their book to market ratio. When earnings have gone down, investors as a group extrapolate the trend into the future and push prices away from fundamentals, giving rise to higher average returns for value firms and vice versa.

Similar to \textcite{LakonishokShleiferVishny1994}, \textcite{BarberisHuang2001} draw on the fact that value firms have experienced decreasing earnings, but suggest that the premium is driven by investors' loss aversion bias. Current value firms have had falling earnings, leading to lower share prices and negative returns. As investors are deterred by the past performance of negative returns in itself, they are less willing to hold value stocks.

[\textcite{LakonishokShleiferVishny1992} find an explanation in the structure of the money manager industry. They suggest that there money managers have incentives to avoid value stocks as they are more likely to go bankrupt, and make the fund performance look bad, than growth stocks, which are more widely held by the reference index.]

Variations of the long-short strategy of the value factor has become a staple strategy of both quantitative and qualitative hedge funds, often under the "equity market neutral" label. Factor equity strategies have also become increasingly accessible for retail investors, especially with the advent of smart beta ETFs.\footnote{See i.a. \textcite{Pedersen2015}, \textcite{AQREMN} and \textcite{McKEMN}.} 

As highlighted by the "Quant crash" in August 2007, there are substantial risks associated with crowded trades. [talk about this]

Less equity hedge fund and investors chasing RMW and CMA so far. Why would they work?

---

To study risk it's good to have a multivariate distribution function. Copulas provide a way to get this.

Copulas are widely used in risk management.

Copulas can model asymmetric tail dependence.

"Factor strategies have become popular not only because of their added alpha to holding the
market, but because of the low correlation across factors, which provides diversification in
a portfolio setting."

Copulas can model non-elliptical dependence in general.