%!TEX root = ./main.tex

\begin{titlepage}
  \title{Risk Analysis of Factor Strategies:\\A Copula Approach}
  \author{
    Gustaf Soldan \\ \href{mailto:22107@student.hhs.se}{22107@student.hhs.se}\and
    Victor Andrée \\ \href{mailto:22584@student.hhs.se}{22584@student.hhs.se}
  }

  \maketitle

  \begin{abstract}
    \noindent We evaluate how factor equity strategies are optimally combined, focusing on the role of the value factor (HML) against the background of a recent academic discussion about its potential redundancy, given the discovery of the investment (CMA) and profitability (RMW) factors. The analysis is centered around a conditional joint return distribution from a dynamic copula model, which allows for a time-varying and non-normal dependence structure. We create portfolios of the six most common equity factors (Market (Mkt.RF), Size (SMB), Value (HML), Investment (RMW), Profitability (RMW) and Momentum (Mom)) and apply two optimization strategies: mean-variance and conditional diversification benefit, where the latter is based on expected shortfall. Our results indicate that HML remains an important factor that increases the Sharpe Ratio and also decreases tail risk. However, HML should only be combined carefully with CMA, as they overlap to some extent. In parallel, we find that RMW is drastically different from HML and CMA and that the factor is significantly more impactful on the risk-reward profile of portfolios.
  \end{abstract}

  \thispagestyle{empty}
\end{titlepage}
