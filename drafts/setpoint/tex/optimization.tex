%!TEX root = ../main.tex

\section{Optimizing factor allocations} % (fold)
\label{sec:optimizing_factor_allocations}

We now turn to the issue of optimizing factor allocations using our estimated copula model.

We optimize portfolio weights according to two techniques: Optimal Mean-Variance (MV) investing and Optimal Conditional Diversification Benefit (CDB) investing, where the latter is a measure introduced by~\textcite{ChristoffersenErrunzaJacobLanglois2012}.

The goal of the first optimization exercise is to determine the risk-reward profile of including HML, CMA and RMW respectively. MV results will test the conjecture in~\textcite{FF2015} that the inclusion of HML does not improve the Sharpe Ratio of the tangency portfolio. 

The goal of the second optimization is to consider risk beyond variance by examining the impact on tail risk of including HML, CMA and RMW respectively. CDB analysis studies whether the non-normal features in the data (i.e. high skewness, kurtosis and tail dependence) give any additional reason, beyond the MV results, to include or exclude the factors HML, CMA and RMW, respectively.

CDB is based on the portfolio expected shortfall (ES), i.e. the expected loss in case the return realizes below its Value-at-Risk (VaR), and therefore concerns the properties of the lower tail of the portfolio distribution. Naturally, such features are not captured by MV analysis.

The remainder of this section presents the construction of the CDB measure and presents the general optimization problem for MV and CDB. Results of the optimizations are presented in the following sections.

%Based on zero-cost portfolio regressions and the work on mean-variance portfolios in \textcite{HubermanKandel1987}, \textcite{FF2015} conjecture that the HML factor will not improve the mean-variance tangency portfolio when added to a portfolio of Mkt.RF, SMB, RMW and CMA. In MV investing, an improvement of a portfolio is to be understood as a higher Sharpe Ratio of the optimal portfolio when HML is included. Differently put, this is equivalent to HML having a non-zero weight in mean-variance optimization.

%We evaluate this based on static weights (based on sample estimates of expected returns and covariances) and dynamic weights (based on copula model estimates).

\subsection{Conditional Diversification Benefit} % (fold)
\label{sub:conditional_diversification_benefit}

This description of CDB follows~\textcite{ChristoffersenErrunzaJacobLanglois2012}. Define ES as the expected loss in some bottom percentile $q$:
\begin{align}
    \text{ES}_{i,t}^q(r_{i,t}) = -\mathbb{E}[r_{i,t} | r_{i,t} \leq F_{i,t}^{-1}(q)]
\end{align}
where $F_{i,t}^{-1}(q)$ is the inverse CDF of simple returns $r_{i,t}$ at $q$ (equivalent to the $q\%$ Value-at-Risk). 

The Expected Shortfall represents the expected loss when returns realize below the Value-at-Risk of the portfolio. Depending on the distribution at hand, the ES can be closer or further to the Value-at-Risk. Intuitively, if assets offer little diversification, then no combination of assets will reduce total portfolio risk; and ES will be higher. 

For a portfolio of assets with weights $w_t$, the portfolio ES $\text{ES}_t^q(w_t)$ has an upper bound equal to the weighted average of each asset's ES, corresponding to the case of no diversification~\autocite{Artzner1999}:
\begin{align}
  \overline{\text{ES}}_t^q(w_t) = \sum_{i=1}^N w_{i,t} \text{ES}_{i,t}^q(r_{i,t})
\end{align}
A lower bound on portfolio ES is given by the portfolio's Value-at-Risk ($-F_{t}^{-1}(w_t, q)$), corresponding to the case of perfect diversification:
\begin{align}
  \underline{\text{ES}}_t^q(w_t) = -F_{t}^{-1}(w_t, q)
\end{align}
CDB is defined as the portfolio's ES scaled by its lower and upper bounds:
\begin{align}
  \text{CDB}_t^q(w_t) = \frac{\overline{\text{ES}}_t^q(w_t) - \text{ES}_t^q(w_t)}{\overline{\text{ES}}_t^q(w_t) - \underline{\text{ES}}_t^q(w_t)}
\end{align}
CDB is a number between zero and one (we report CDB scaled to 0--100) and measures how close to its Value-at-Risk the Expected Shortfall of a portfolio gets. Note that the level of expected return does not enter into the measure -- CDB only measures how powerful a group of assets are at achieving low tail risk.

% subsection conditional_diversification_benefit (end)

\subsection{Optimization Problem} % (fold)
\label{sub:optimization_problem}

Each week, we choose portfolio weights to maximize the Sharpe Ratio (SR) in the mean-variance case, and CDB in the CDB case. We impose two restrictions on our optimization problem. First, all factor weights must be positive. This reduces the problem with extreme weights, as seen in the unconstrained optimization problem, and reflects a view that an investor will not bet against factors that have generated a history of positive premia. Second, factor weights must sum to unity, i.e. the portfolio is fully invested across factors. In light of~\textcite{Asness2015}, we consider portfolios with and without momentum (five- and six-factor portfolios, respectively).

Together, these restrictions make the maximized Sharpe Ratio reflect tangency portfolio weights subject to a constraint of no negative weights. Due to the restrictions, the standard analytical solution to the mean-variance problem is not equal to our optimal tangency portfolio. Similarly, no analytical solution exists for optimizing CDB. Hence, we perform a numerical optimization where we choose weights $w_t$ to maximize the objective function $\Omega_t(w_t)$, subject to the restrictions above:
\begin{align}
  \arg\!\max_{w_t} \Omega_t(w_t)
    &&\text{\quad s.t.\quad} & \sum_{i=1}^N w_{i,t} = 1 \\
    &&                       & w_{i,t} \ge 0 \notag
\end{align}
For the MV case, the objective function is the one-week Sharpe Ratio (SR):
\begin{align}
  \Omega_t(w_t) = \frac{w_t^\top \mathbb{E}_t[r_{t+1}]}{\sqrt{w_t^\top \mathbb{E}_t[\Sigma_{t+1}] w_t}}
\end{align}
$\mathbb{E}_t[r_{t+1}]$ is the conditional one-step-ahead expected factor return and $\mathbb{E}_t[\Sigma_{t+1}]$ is the conditional one-step-ahead variance-covariance matrix. 

For the CDB case, the objective function is the one-week CDB:
\begin{align}
  \Omega_t(w_t) = \text{CDB}_t^q(w_t)
\end{align}
We perform MV and CDB optimization using in-sample simulated distributions from our dynamic copula model. Each week, we simulate $10\,000$ returns from the estimated copula model and use those outcomes as the conditional distribution of factor returns. We also perform MV optimization based on sample means and covariances, in which case $\mathbb{E}_t[r_{t+1}]$ and $\mathbb{E}_t[\Sigma_{t+1}]$ are constant and equal to the full-sample sample estimators.

% subsection optimization_problem (end)

% section optimizing_factor_allocations (end)
