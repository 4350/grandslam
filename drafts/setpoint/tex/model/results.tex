%!TEX root = ../main.tex
\subsection{Copula specification and estimation results}
Given the results of the dependence structure of residuals, we now discuss the best choice of copula model and present estimation results of the six competing copula specifications.

We have estimated constant and dynamic normal, symmetric and skewed copula models on the full dataset of GARCH uniform residuals. Results are presented in~\autoref{tab:copula_estimation}. 

First, we examine the choice between a normal, symmetric \textit{t} or skewed \textit{t} copula. We note that $\nu_c$ is clearly significant and suggests a Student's \textit{t} model with tail dependence over the normal model. Second, we examine the asymmetric specification and find that few of the $\gamma_c$ estimates appear significant This indicates that the asymmetry is hard to capture, or that it is not well represented by this type of model. This is supported by the relatively small improvement in log-likelihood by going from a symmetric to skewed copula and the fact that the BIC criterion prefers the symmetric model in the dynamic case. 

Second, we examine the choice between a constant and dynamic copula correlation matrix. There is a significant improvement in log-likelihood and BIC when moving from a constant to a dynamic copula, which suggests that time-varying dependence shown by rolling correlation is captured and improves the model's fit. We also find a high persistence of the correlation process, as $\alpha + \beta$, is close to a unit root.

In summary, we find that the dynamic symmetric \emph{t} copula is the best specification, as it has the lowest BIC, well defined parameters, and is strongly supported by the dependence pattern showcased by threshold and rolling correlation analyses. While the asymmetric \emph{t} is an interesting model, we believe that the asymmetry patterns in data are too irregular to be well captured by a copula model with only one asymmetry parameter for each series (this is further discussed in the subsequent robustness discussion, see \autoref{sub:05_robust}).

%!TEX root = ../../main.tex

\begin{table}[!ht]
  \centering
  \scriptsize
  \renewcommand{\arraystretch}{1.2}

  \caption{Parameter estimates for copula models based on uniform residuals from ARMA-GJR-GARCH models.\\ \quad \\
  Stationary bootstrap standard errors in parentheses, following Politis and Romano (1994). Copula parameters: $\nu_c$ is the degree of freedom, $\gamma_c$ is the vector of skewness parameters, $\alpha$, $\beta$ are the shock loading and autoregressive loading of the cDCC process. The significance test of $\nu_c$ is based on $1/\nu_c$, as this ratio goes to zero when $\nu_c$ goes to infinity (normality). Sample: 1963-07-05--2016-07-01.}
  \begin{tabularx}{\textwidth}{@{}l ddd X ddd @{}}
    \toprule
    &
      \multicolumn{3}{c}{Constant Copula} &&
      \multicolumn{3}{c}{Dymamic Copula} \\
    \cmidrule{2-4} \cmidrule{6-8}
    &
      \multicolumn{1}{c}{Normal} & \multicolumn{1}{c}{Symmetric \emph{t}} & \multicolumn{1}{c}{Skewed \emph{t}} & &
      \multicolumn{1}{c}{Normal} & \multicolumn{1}{c}{Symmetric \emph{t}} & \multicolumn{1}{c}{Skewed \emph{t}} \\
    \midrule
    $\nu_c$ & & 6.625^{**} & 6.671^{**} && & 11.936^{**} & 11.881^{**} \\
    & & (0.636) & (0.264) && & (0.770) & (0.641) \\
    \\
    $\gamma_\text{Mkt}$ & & & -0.057 && & & -0.078 \\
    & & & (0.047) && & & (0.062) \\
    \\
    $\gamma_\text{HML}$ & & & 0.103 && & & 0.083 \\
    & & & (0.036) && & & (0.071) \\
    \\
    $\gamma_\text{SMB}$ & & & -0.103 && & & -0.175 \\
    & & & (0.055) && & & (0.098) \\
    \\
    $\gamma_\text{Mom}$ & & & -0.202^{**} && & & -0.145 \\
    & & & (0.032) && & & (0.073) \\
    \\
    $\gamma_\text{RMW}$ & & & 0.021 && & & 0.095 \\
    & & & (0.035) && & & (0.058) \\
    \\
    $\gamma_\text{CMA}$ & & & 0.076 && & & 0.001 \\
    & & & (0.038) && & & (0.050) \\
    \\
    $\alpha$ & & & && 0.065 & 0.068^{**} & 0.068^{**} \\
    & & & && (0.006) & (0.006) & (0.006) \\
    \\
    $\beta$ & & & && 0.915 & 0.913^{**} & 0.913^{**} \\
    & & & && (0.008) & (0.007) & (0.007) \\
    \midrule
    Log-likelihood & 1169.194 & 1555.683 & 1572.672 && 2790.618 & 2977.648 & 2989.273 \\
    No. of Parameters & 15 & 16 & 22 && 17 & 18 & 24 \\
    % BIC & -348.32 & -122.21 & -316.432 && -243.221 & -342.342 & -396.324 \\
    Persistence & & & && 0.981 & 0.981 & 0.981 \\
    \bottomrule
  \end{tabularx}

  \label{tab:copula_estimation}
\end{table}
