%!TEX root = ../../main.tex

\subsection{Definition of copula model} % (fold)
\label{sub:definition_of_copula_model}

Each week $t$, the conditional joint density of returns on $N$ assets $R_{t+1} = \{r_{i,t+1},\ldots,r_{N,t+1}\}$ is described by a joint density function $f_t(R_{t+1})$. Following~\textcite{ChristoffersenErrunzaJacobLanglois2012}, who build on~\textcite{Patton2006} and~\textcite{Sklar1959}, we decompose the joint density function into the product of a joint copula function $c_t(U_{t+1})$ of uniformly distributed variables $U_{t+1} \sim U(0, 1)$ and marginal densities $f_{i,t}(r_{i,t+1})$:
\begin{align}
  f_t(R_{t+1}) =
    c_t(U_{t+1}) \prod^N_{i=1} f_{i,t}(r_{i,t+1})
  \label{eq:copula_sklar}
\end{align}
The elements of $U_{t+1} = \{u_{i,t+1},\ldots,u_{N,t+1}\}$ are related to the original returns by the probability integral transform, i.e the cumulative distribution of $r_{i,t+1}$:
\begin{align}
  u_{i,t+1} = F_{i,t}(r_{i,t+1}) = \int_{-\infty}^{r_{i,t+1}} f_{i,t}(r)dr
\end{align}
The copula function $c_t(U_{t+1})$ is a multivariate skewed \emph{t} distribution. This distribution is parameterized by a single degrees of freedom parameter $\nu_c$, controlling the degree of dependence, a vector of $N$ skewness parameters $\gamma_c$, controlling the asymmetry in dependence, and a potentially time-varying correlation matrix $\Psi_{t}$.\footnote{We describe the details of the skewed \emph{t} distribution, including the expanded form of $c_t$, in~\autoref{app:ghstmv}.} The skewed \emph{t} distribution nests the standard (hereafter symmetric) \emph{t} distribution when all $\gamma_{i,c} = 0$ and the standard normal distribution when additionally $\nu_c = \infty$.

The log-likelihood of the model is constructed from~\autoref{eq:copula_sklar}:
\begin{align}
  L =
    \underbrace{\sum_{t=1}^T \log(c_t(U_{t+1}))}_\text{Copula} +
    \underbrace{\sum_{t=1}^T \sum_{i=1}^N \log(f_{i,t}(r_{i,t+1}))}_\text{Marginals}
\end{align}
At this point, it is worth noting that the joint density $c_t(U_{t+1})$ need not be of the same family as the marginal densities $f_{i,t}(r_{i,t+1})$ -- nor are we restricted to modeling $f_{i,t}(r_{i,t+1})$ jointly for all factors. In fact, we take advantage of this flexibility and choose to model the marginal densities independently as ARMA-GARCH processes, which allows us to capture a number of predictable features in the univariate series -- serial correlation, volatility clustering and leverage effects. The marginal models are estimated independently by maximizing the likelihood(s) of the second term, and then the copula is estimated by maximizing the first term -- using the residuals of the marginal models as given.

This procedure is called multi-stage maximum log-likelihood or inference functions for margins and greatly simplifies the estimation procedure, while yielding relatively efficient estimates~\autocite{Patton2006,Joe1997}. The modeling and estimation of our ARMA-GARCH models is detailed in the upcoming subsection, whereas the remainder of this subsection describes how we make the correlation matrix $\Psi_t$, and thus the dependence between factors, dynamic.

The copula is made dynamic by fitting a dynamic conditional correlation (DCC) process for $\Psi_t$ to copula residuals $z_{t+1}^*$~\autocite{Engle2002}. Using the notation from~\textcite{ChristoffersenLanglois2013}:
\begin{align}
  Q_t = (1 - \alpha - \beta) Q
    + \beta Q_{t-1}
    + \alpha \bar{z}_{t-1}^* \bar{z}_{t-1}^{*\top}
  \label{eq:copula_cdcc}
\end{align}
where $Q_t$ is normalized to the correlation matrix $\Psi_t$:
\begin{align}
  \Psi_t = Q_t^{-\frac{1}{2}} Q_t Q_t^{-\frac{1}{2}}
  \label{eq:copula_cdcc_psi}
\end{align}
The $Q_t$ process is comprised of three components that are weighted according to $\alpha, \beta$: (1) a time-invariant component: $Q$, (2) an innovation component from copula shocks: $\bar{z}_{t-1}^{*} \bar{z}_{t-1}^{*\top},$\footnote{Where $\bar{z}_{i,t+1}^* = z_{i,t+1}^* \sqrt{q_{ii,t}}$ is due to a correction by~\textcite{Aielli2013}, that improves the reliability of the estimation procedure.} and (3) an autoregressive component of order one: $Q_{t-1}$. In order for the the correlation matrix $\Psi_t$ to be positive definite, $Q_t$ has to be positive definite, which is ascertained by requiring that $\alpha \geq 0$, $\beta \geq 0$ and $(\alpha + \beta) < 1$. The model nests a constant copula when $\alpha = \beta = 0$.

The model for $c_t(U_{t+1})$ is comprised of $1 + N$ distribution parameters $\{\nu_c, \gamma_c\}$ and $2 + \frac{N(N-1)}{2}$ dynamics parameters $\{\alpha, \beta, Q\}$, where the elements of $Q$ are estimated using moment matching, and the remaining parameters $\{\alpha, \beta, \nu_c, \gamma_c\}$ are estimated using maximum likelihood.\footnote{A detailed description of the copula estimation procedure can be found in~\autoref{app:copula_cdcc}.}

ARMA-GARCH modeling allows us to filter time-varying effects, leaving independent \emph{standardized residuals} $z_{i,t}$, which are assumed to follow a constant distribution $f_i(z_{i,t})$. These residuals are first transformed into uniform variables $u_{i,t+1}$ by the probability integral transform of the densities above, and then made to follow the \emph{copula} distribution by the \emph{inverse} probability integral transform of the \emph{copula}:
\begin{align}
  z_{i,t+1}^* = F^{-1}_{\nu_c,\gamma_{i,c}}(F_{i}(z_{i,t+1}))
\end{align}

The interpretation of the copula parameterization is closely associated to the structure of multivariate dependence. By different restrictions on the parameters in the DAC model, we are able to activate or deactivate certain features of the copula: First, the degree of freedom parameter $\nu_c$ is to be interpreted as the measure of tail dependency. When $\nu \neq 0$, the lower and upper tails of the joint distribution are fatter than in the normal case, which is coherent with earlier evidence of threshold correlations~\autocite{ChristoffersenLanglois2013}. Second, the skewness parameters $\gamma_{c,i}$ are to be interpreted as the extent of asymmetry in the correlation structure. When $\gamma \neq 0$, there is asymmetry in correlations. Third, the $\alpha$ and $\beta$ parameters determine whether the copula generates time-varying correlations. If $\alpha \neq 0$ and $\beta \neq 0$, the copula is dynamic. An overview of the six copula models is given in \autoref{tab:conceptual}.

%!TEX root=../../main.tex

\begin{table}
  \centering
  \footnotesize
  \renewcommand{\arraystretch}{1.2}

  \caption{Conceptual matrix of copula parameterizations}

  \begin{tabularx}{0.80\textwidth}{@{} lc c >{\centering}Xc >{\centering}Xc >{\centering\arraybackslash}X}
    \toprule
      & && \textbf{Normal} && \textbf{Symmetric \emph{t}} && \textbf{Skewed \emph{t}} \\
      \cmidrule{4-4}
      \cmidrule{6-6}
      \cmidrule{8-8}
      & && $\nu_c = \infty$   && $\nu_c < \infty$   && $\nu_c < \infty$ \\
      & && $\gamma_{i,c} = 0$ && $\gamma_{i,c} = 0$ && $\gamma_{i,c} \neq 0$ \\
      \cmidrule{4-8}
    \cmidrule{1-2}
    \multirow{2}{*}{\textbf{Constant}} & $\alpha = 0$ && Constant && Constant && Constant \\
                              & $\beta = 0$  && Normal   && Symmetric \emph{t} && Skewed \emph{t}      \\
    \cmidrule{1-2}
    \multirow{2}{*}{\textbf{Dynamic}}  & $\alpha > 0$ && Dynamic  && Dynamic && Dynamic \\
                              & $\beta > 0$  && Normal   && Symmetric \emph{t} && Skewed \emph{t}      \\
    \bottomrule
  \end{tabularx}
% ()
%   \begin{tabularx}{\textwidth}{@{\extracolsep{5pt}} c c c c X c X c @{}}
%     \toprule
%   				&			& &	\textbf{Normal}	&	&	\textbf{Student's \textit{t}}	&	&	\textbf{Asymmetric Student's \textit{t}} \\
%   				\\
%   				&			& & 	$\nu = \infty$	&	&	$\nu > 0$	& 	&	$\nu > 0$ \\
%   				&			& & 	$\gamma = 0$	&	&	$\gamma = 0$	& 	&	$\gamma \neq 0$ \\
%           \\
%     \cmidrule{4-8}
%     \\
%      \textbf{Constant} &  & & \text{Constant normal copula} & & \text{Constant symmetric \textit{t} copula} & & \text{Constant asymmetric \textit{t} copula} \\
%      \\
%     	&	$\alpha = 0$  &		&	\textit{Constant correlations but} & & \textit{Constant correlations and} & & \textit{Constant correlations and} \\
%        & $\beta = 0$ & & \textit{no tail dependence} & & \textit{symmetric tail dependence} & & \textit{asymmetric tail dependence} \\
%     \\
%     \cmidrule{4-8}
%     \\
%      \textbf{Dynamic} &  & & \text{Dynamic normal copula} & & \text{Dynamic symmetric \textit{t} copula} & & \text{Dynamic asymmetric \textit{t} copula} \\
%      \\
%       & $\alpha > 0$  &   & \textit{Dynamic correlations but} & & \textit{Dynamic correlations and} & & \textit{Dynamic correlations and} \\
%        & $\beta > 0$ & & \textit{no tail dependence} & & \textit{symmetric tail dependence} & & \textit{asymmetric tail dependence} \\
%     \\
%     \bottomrule
%   \end{tabularx}

  \label{tab:conceptual}	
\end{table}



% subsection definition_of_copula_model (end)
