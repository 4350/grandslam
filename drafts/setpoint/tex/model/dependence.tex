%!TEX root = ../../main.tex

\subsection{Multivariate dependence} % (fold)

In this subsection, we demonstrate that the standardized residuals $z_{i,t}$ in our chosen ARMA-GARCH models display both asymmetric and time-varying dependence, shown by threshold and rolling correlations. ARMA-GARCH filtering has little effect on the (time-varying) correlations between factors. However, the use of the skewed \emph{t} distribution does remove a degree of asymmetry in the dependence. These patterns in multivariate dependence are the motivating reasons for a copula model, as they suggest that factor returns are not independent of each other after filtering for univariate effects, and that this dependence is not well-approximated by a normal model.\footnote{A visual comparison of dependence measures on returns compared to standardized residuals can be found in~\autoref{app:supplementary}.}

% XXX Här finns en liten karamell att suga på; gammal text nedan

% , measured by threshold correlations, and time-varying correlations, measured by rolling correlations, features we will attempt to model in the copula model. In the context of modeling, we are interested in the standardized residuals, rather than the returns themselves, as they are filtered of the variance dynamics present in the returns themselves~\autocite{ChristoffersenLanglois2013}.\footnote{The visual patterns for both threshold and rolling correlations are quite similar between returns and standardized residuals; results are available in the appendix.}

\subsubsection{Threshold correlations}

Threshold (or exceedance) correlations have previously been used to highlight the asymmetric dependence structure of i.a. country equity indices~\autocite{LonginSolnik2001}, portfolios by industry, size, value and momentum~\autocite{AngChen2002} and factor strategies~\autocite{ChristoffersenLanglois2013}. The following analysis is still new as it adds the investment (CMA) and profitability (RMW) factors. We follow~\textcite{ChristoffersenLanglois2013} definition of threshold correlation:
\begin{align}
\label{eq:th_corr}
    ThCorr(r_i, r_j) = 
    \begin{cases} 
        Corr\Big(r_i, r_j \,|\, r_i < F_i^{-1}(p), r_j < F_j^{-1}(p)\Big)  & \text{for } p < 0.5 \\
        Corr\Big(r_i, r_j \,|\, r_i \geq F_i^{-1}(p), r_j \geq F_j^{-1}(p)\Big)  & \text{for } p \geq 0.5
    \end{cases}
\end{align}
% XXX Be consistent with percentile and quantile
where $F_i^{-1}(p)$ is the empirical quantile of $r_i$ at percentile $p$. Threshold correlations thus reflect how series correlate when both are simulataneously realizing in their respective tails. This subsetting of data is illustrated in \autoref{fig:illustrate_threshold}. In the left hand plots, we see the scatter of ARMA-GARCH residuals of Mkt.RF and HML respectively, and how the threshold $p$, found on the $x$-axis of the right hand plot, determines the subset of data that is included in the correlation calculation. We note that the unconditional (standard) correlation, given by the dashed line in the right hand plot, is clearly negative, while threshold correlations in the first and third quadrants are significantly more positive, which shows that not taking threshold correlations into account provides a vaguer picture of the dependence structure when both factor series realize in the tails.

\begin{figure}[H]
  \centering
  \includegraphics[scale=1]{graphics/threshold_explain_res.png}
  \footnotesize
  \caption{Illustration of threshold correlations}
  \begin{longcaption}
    ARMA-GARCH residuals from the Mkt.RF--HML asset pair. 95\% shaded confidence bounds. The unconditional correlation is given by the dashed line. Based on weekly data 1963--2016.
  \end{longcaption}
  \label{fig:illustrate_threshold}
\end{figure}

We now plot threshold correlations without the adjacent scatter graph. \autoref{fig:threshold1} displays threshold correlations for HML, CMA and RMW against each other as well as against the the other factors Mkt.RF, SMB and Mom. We note that for most asset pairs, the threshold correlation is significantly different from the unconditional correlation coefficient given by the dashed line.

We also note that there is asymmetry around the median for some factor pairs, including the Mom--CMA, RMW--HML, RMW--CMA, and to a lesser extent Mkt.RF--RMW, asset pairs. For example, in the Mom-CMA asset pair, the threshold correlation jumps up for the first percentile below the median, indicating that the correlation is higher when both realize below the median than when both realize above the median. This type of asymmetric property, where downside (below the median) correlation is higher than upside correlation is unwanted, as it reflects a poorer diversification in bad times. The opposite type of asymmetry can be seen for the HML--RMW and CMA--RMW asset pairs; When these factors simultaneously realize above the median, they are significantly more correlated. This pattern presents no diversification problem.

Although estimated with substantial uncertainty, the threshold correlations do not seem to be constant as the threshold $p$ approaches either zero or one. For example, the Mkt.RF--HML asset pair seems to have a downward pattern, where correlations are the most positive in the lowest percentiles of residuals and the most negative in the highest percentiles of residuals. In fact, this pattern is unwanted from a diversification perspective, as series tend to coincide more in extreme negative events. 

The CMA--HML pair stands out from the other factors. The pair exhibits an unusually high correlation in excess of $0.60$ with a virtually flat threshold pattern. HML and CMA are also generally similar to each other in their respective threshold patterns to other factors -- most notably in the HML/CMA--RMW pairs. They differ in the presence of a break around the median in Mom--CMA not present in Mom--HML.

RMW is the only factor to be virtually uncorrelated with Mkt.RF in the lower tails, suggesting that it is a good diversifier in market downturns. This is very different from the pattern of higher threshold correlations as $p$ approaches zero for e.g. Mkt.RF--HML. For Mkt.RF--HML, the higher lower tail correlation could be related to the industry over-capacity hypothesis discussed in~\autoref{sec:literature}, i.e. that value firms are particularly sensitive to market downturns due to unproductive capital.

While the patterns in threshold correlations are interesting, we are careful not to draw conclusions regarding diversification benefit based on solitary threshold correlation graphs -- what is interesting is the total pattern, and our key point is that there seems to be tail dependence that should not be ignored in the copula specification.

\begin{figure}[p]
  \centering
  \includegraphics[scale=1]{graphics/threshold1.png}
  \footnotesize
  \caption{Threshold correlations of ARMA-GARCH standardized residuals}

  \begin{longcaption}
    The formula for threshold correlations for a threshold $p$ is given in~\autoref{eq:th_corr}. 95\% shaded confidence bounds, taking the model as given. The unconditional correlation is given by the dashed line. Based on weekly data 1963--2016.
  \end{longcaption}
  \label{fig:threshold1}
\end{figure}
\begin{figure}[p]
  \ContinuedFloat
  \centering
  \includegraphics[scale=1]{graphics/threshold2.png}
  \footnotesize
  \caption{Threshold correlations of ARMA-GARCH standardized residuals (cont.)}
\end{figure}

\subsubsection{Rolling correlations}

We compute rolling 52-week correlations between the factors on standardized residuals of our ARMA-GARCH models, according to the formula: 
\begin{align}
    RCorr(r_{i, t}, r_{j, t})_t^{52} = \frac{\sum^{t}_{t-51}(r_{i, t} - \bar{r}_i)(r_{j,t} - \bar{r}_j)}{\sqrt{\sum^{t}_{t-51} (r_{i,t} - \bar{r}_i)^2} \sqrt{\sum^{t}_{t-51} (r_{j,t} - \bar{r}_j)^2}}
\end{align}
where $r_i$, $r_j$ are the different pairs of the factor strategies' ARMA-GARCH residuals.\footnote{Rolling correlations for the returns themselves are available in \autoref{fig:appendix_rolling1} (\autoref{app:supplementary}).} Results are presented in ~\autoref{fig:rolling1}.
% plots
\begin{figure}[!p]
  \centering
  \includegraphics[scale=1]{graphics/rolling1.png}
  \footnotesize
  \caption{Rolling correlations of ARMA-GARCH standardized residuals}
  \begin{longcaption}
    95\% shaded confidence bounds, taking the model as given. The unconditional correlation is given by the dashed line. Based on weekly data 1963--2016.
  \end{longcaption}
  \label{fig:rolling1}
\end{figure}
\begin{figure}[!p]
  \ContinuedFloat
  \centering
  \includegraphics[scale=1]{graphics/rolling2.png}  
  \footnotesize
  \caption{Rolling correlations of ARMA-GARCH standardized residuals (cont.)}
\end{figure}
% talk
First, we note that for most factor pairs, the rolling 52-week correlations are time-varying, and indeed appear to swing wildly. The unconditional correlation of Mkt--HML is negative in the studied time period, but rolling correlations range between -0.75 and 0.75. Also of note is the momentum factor's rapid shifts between positive and negative correlations to the other factors. 

% Value and profitability in crisis times?
Second, by visual inspection, we see no obvious trend in the correlations between factor pairs. There are, however, notable patterns around the 2000--2001 bubble period -- here, the correlations of HML--RMW, CMA--RMW and Mkt--CMA appear to jump. Another interesting pattern is that the correlations of Mkt--RMW went down sharply around this period -- in line with the idea that profitable firms are stronger and better at weathering crises than the average firm~\autocite{NovyMarx2013}. The 2000--2001 period may represent a structural break in the dependency patterns between factors, with the appearance of persistent differences before and after -- however, there is not enough post-2000 data to support such a conclusion, yet.

Third, the HML--CMA factor pair again stands out as different from other factor pairs. The unconditional correlation is much closer to the rolling estimates than for other factor pairs, with a dip in the 2000--2010 period that appears to have gone away. Clearly, the HML--CMA pair is the most strongly correlated factor pair, even when considering subperiods of the data.

Our key takeaway from the rolling correlations is that there seems to be persistency in the time-variation in correlations. This could be incorporated in the copula specification, which then needs to have a time-varying correlation matrix, $\Psi_t$.

% \subsubsection{Takeaways from Analysis of Multivariate Dependence}

% Univariate residuals appear to be white noise series with no remaining autocorrelation or volatility clustering. However, there is important dependence between residuals of different strategies. First, threshold correlations show that there is tail dependence -- in times when factor pairs simultaneously realize in their best and worst percentiles, correlations are significantly different from the unconditional correlations. In fact, threshold correlations are substantially higher than the unconditional correlations, which indicates that diversification benefits are smaller than expected when factors simultaneously experience bad (or good) times. Second, rolling correlations show that correlations between series are highly time-varying and seem to exhibit persistence. A copula model that incorporates both tail dependence and time-varying dependence is likely to improve on the description of joint returns.

% Both analyses also show that the HML-CMA asset pair is quite different from the other pairs, exhibiting a much higher and more stable dependence. Differently put, they look quite similar as opposed to any other factor pair, and the merit of including both in factor portfolios seems more unclear.

\label{sub:threshold_and_rolling_correlations_of_residuals}

% subsection threshold_and_rolling_correlations_of_residuals (end)
