%!TEX root = ../main.tex

\section{Univariate diagnostic tests}
\label{app:univariate_diagnostics}

\textbf{Autocorrelation test}

The autocorrelation test is a weighted Ljung-Box test, following~\textcite{FisherGallagher2012} and~\textcite{LjungBox1978}. Under the null of a correctly specified model with no serial correlation, the weighted Ljung-Box test has been shown to generate results closer to its asymptotic distribution than the standard Ljung-Box test. The test statistic is given by
\begin{align}
  Q_W = T (T+2) \sum\limits^m_{k = 1} \frac{m-k+1}{m} \frac{\hat{r}_{k}^{2} (\hat{\epsilon}_{t} / \hat{\sigma}_{t})}{T-k}
\end{align}
where $T$ is the number of observations, $\hat{r}^{2}_{k} ( \hat{\epsilon}_{t} / \hat{\sigma}_{t} )$ is the squared sample autocorrelation of standardized residuals with lag order $k$ and max lag order $m$. Under the null, the test statistic is asymptotically distributed $\sum\limits^m_{k = 1} \chi^2_k \gamma_k$, where $\{\chi^2_k\}$ are independent chi-squared random variables with one degree of freedom and $\{\gamma_k\}$ are eigenvalues of a weighting matrix. We consider two maximum lag orders, 5 and 10 weeks. The maximum lag length was chosen by visual inspection of the autocorrelation functions for standardized residuals.

\textbf{Volatility clustering test}

For ARCH effects, we use the weighted LM test, following~\textcite{FisherGallagher2012} and~\textcite{LiMak1994}. The test has the null of no autocorrelation in standardized squared residuals from the model, and the test statistic is given by:
\begin{align}
  LM_W = T \sum\limits_{k = b + 1}^{m} \frac{m - k + (b+1)}{m} \hat{r}^{2}_{k} (\hat{\epsilon}^{2}_{t} / \hat{\sigma}_{t})
\end{align}
where $T$ is the number of observations, $b$ the number of autoregressive lags in the GARCH ($b=1$), $\hat{r}^2_k (\hat{\epsilon}^2_t / \hat{\sigma}_t)$ is the squared sample autocorrelation of standardized squared residuals with lag order $k$ and max lag order $m$. Under the null, the test statistic is asymptotically distributed $\sum\limits^m_{k = 1} \chi^2_k w_k$, where $\{\chi^2_k\}$ are independent chi-squared random variables with one degree of freedom and $\{w_k\}$ are the weighting parameters ($w = (m - k + (b+1))/m$). The maximum lag length was chosen by visual inspection of the autocorrelation functions for standardized squared residuals.

\textbf{Leverage effect test}

We use the sign bias test of~\textcite{EngleNg1993} to determine whether there are significant leverage effects in the factor returns. Run the regression
\begin{align}
  \hat{z}_t^2 = c_0 + c_1 I_{\hat{\epsilon}_{t-1} < 0} + c_2 I_{\hat{\epsilon}_{t-1} < 0} \cdot \hat{\epsilon}_{t-1} + c_3 I_{\hat{\epsilon}_{t-1} \geq 0} \cdot \hat{\epsilon}_{t-1} + u_t
\end{align}
where $\hat{z}_t^2$ are the standardized squared residuals of the ARMA-GARCH model, $I_\cdot$ are indicator functions that are equal to one when the subscript conditions are true, and $\hat{\epsilon}_{t-1}$ are the lagged ARMA-GARCH residuals. For the test of negative sign bias (i.e. leverage effect), the null hypothesis is $H_0: c_2 = 0$, and for the test of positive sign bias (i.e. reverse leverage effect), the null hypothesis is $H_0: c_3 = 3$. The Wald test statistics are asymptotically distributed $\chi^2$ with one degree of freedom.
