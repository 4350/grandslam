%!TEX root = ../main.tex

\section{Skewed Student's \emph{t} distribution and copula density} % (fold)
\label{app:ghstmv}

We use the skewed Student's t distribution in modeling both univariate series as well as for the joint distribution under the copula. We use the definition of~\textcite{Hansen1994} and the following description is based on~\textcite{ChristoffersenLanglois2013}. A random vector $X$ that follows a multivariate skewed Student's distribution has the stochastic description
\begin{align}
  X = \sqrt{W}Z + \gamma W
\end{align}
where $\gamma$ is a vector of asymmetry parameters, $Z$ follows a standard multivariate normal distribution with correlation matrix $\Psi$, $W$ follows an inverse gamma distribution $\text{IG}(\dfrac{\nu}{2}, \dfrac{\nu}{2})$. Thus, the parameters of the multivariate distribution are degrees of freedom $\nu$, asymmetries $\gamma$ and an underlying correlation matrix $\Psi$.

The distribution has expectation and covariance matrix:
\begin{align}
  \mathbb{E}[X] &= \frac{\nu}{\nu - 2} \gamma \\
  \text{Cov}(X) &=
    \frac{\nu}{\nu - 2} \Psi +
    \frac{2 \nu^2 \gamma \gamma^\top}{(\nu - 2)(\nu - 4)}
\end{align}
i.e. $\nu \geq 4$ for these to be well-defined. Note that if $\gamma = 0$ (element-wise), $X$ follows a multivariate Student's \emph{t} distribution, and additionally if $\nu = \infty$, $X$ follows a multivariate standard normal distribution. Hypotheses $\gamma = 0$ and $1/\nu = 0$ can therefore be used to test for symmetry and normality, respectively.

% TODO Re-add the copula density function here; how cumulative density and quantile function does not have a closed form.

The copula joint density function $c_t$ always takes the form of the ratio between a joint density function $f^c_{t}(z_{t+1})$ (i.e. the multivariate normal, Student's t or skewed Student's t PDF, respectively) of copula shocks $z_{t+1}^*$ and the product of the univariate density functions $f^c_{i,t}(z_{i,t+1})$ (i.e. the univariate normal, Student's or skewed Student's t PDF, respectively) of the individual shocks $z_{i,t+1}^*$:
\begin{align}
  c_t(U_{t+1}) = 
    \frac{f^c_t(z^*_{t+1})}{\prod_{i=1}^N f^c_{i,t}(z^*_{i,t+1})}
\end{align}
where the relationship between $z^*_{i,t+1}$ and $u_{i,t+1}$ is governed by the inverse cumulative distribution function, as detailed in the subsequent appendix. Note that if the copula distribution and marginal distributions are the same, the denominator cancels in~\autoref{eq:copula_sklar} and the copula is directly the joint distribution of the marginal densities.

% section skewed_student (end)
