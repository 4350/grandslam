%!TEX root = ../main.tex

\section{Discussion and conclusion} % (fold)
\label{sec:discussion_conclusion}

% XXX 
The first key finding of this thesis is that the classic value factor, HML, is still highly relevant for factor investors. Zero-cost regressions in the five-factor model suggest that HML may be fully explained by the remaining four factors, we find evidence to the contrary when accounting for the momentum factor in a weekly dataset. We believe that the reason why zero-cost regressions indicate that HML does not add value is that the regressions are misspecified. The omission of an important sixth factor strategy, momentum, creates a bias on the factor loadings, and leads to not recognizing the added value of HML. While the role of HML as a unique addition was already defended by \textcite{Asness2015}, we have pursued the argument and can conclude that optimizations lead to positive allocations to the HML strategy.

We run the mean-variance optimization using dynamic inputs from a copula model, which can generate a conditional distribution of returns that captures the time-variation in dependence. The actual implementation of mean-variance optimizations under the only constraint of non-negative weights gives a significant positive allocation to the HML factor, which improves the realized risk-return trade-off. These results are present, but less pronounced for static sample estimators of means and covariances.

Although variance is the staple measure of risk, we also investigate whether risk beyond the first two moments can provide reasons against the HML factor. The copula model has allowed us to infer the full distribution of returns, and we shift our focus to the tail of the distribution. Here, we find that the diversification benefit of HML is similar to that of CMA. HML can by no means be considered a worse diversifier against tail risk, as measured by expected shortfall of a factor portfolio, than CMA.

Our second key finding is that HML is highly similar, but not quite the same as CMA. On a closer look, differences emerge, and we believe that factor investors should combine both factors with consideration. There is important theoretical and empirical support for an overlap in the stock positions that comprise HML and CMA, which results in a substantially higher correlation for this factor pair than for any other factor pair. The dependence between the two factors is also more stable, and does not exhibit the same pattern of tail dependence as do other factor pairs. 

Still, one factor cannot replace the other. HML and CMA exhibit different properties, as HML firms are more profitable and exhibit less momentum, and generate return premia that mean-variance optimization suggests useful. Beyond the risk-reward of the first two moments, analysis of the full return distribution from the copula model shows that the diversification benefit of adding either HML or CMA to a portfolio is not constant over time. Sometimes HML is the better diversifier, sometimes CMA is better, but no pattern emerges as to which factor is better than the other. We see no reason for investors to choose either one or the other, as both provide valuable diversification, and even better, they do so at different times.

We believe that investors should consider the two factors jointly when building portfolios. When one of the two factors is included in a factor portfolio already containing the second, the first factor almost exclusively cannibalizes on the weight from the second. Our findings therefore support the existing theoretical and empirical evidence of an overlap in the firms that comprise the two strategies. All in all, we are wary of factor weighting schemes that suggest pure equal-weights for HML and CMA. While such schemes are valuable for factor investing in general, as they avoid the pitfall of factor (mis)timing, they should be designed in a manner that takes the close link between HML and CMA into account. This pair has a very different dependence from all other pairs -- and so the allocation policy to these factors should be different.

A third finding of this thesis is the strength of the profitability factor, RMW. This new factor co-varies negatively with most factors and receives zero or negative factor loadings in zero-cost regressions. The exclusion of RMW in our diversification benefit analysis completely pulls the plug on diversification, making periods of low diversification both more frequent and much more severe. Furthermore, the factor receives high allocations and contributes large improvements to all portfolio performance measures. The fact that there are no strong explanations of the profitability factor as a risk premium makes these findings even more puzzling. Our takeaway is that all funds and investors in the factor space should seriously consider adding this new factor.

During the writing of this thesis, we also considered studying additional emerging factors, in particular the low volatility and betting-against-beta factors. At a first glance, we found highly diversifying characteristics of these factors, which remind us of RMW. It would be especially interesting to study their impact on tail risk, following the CDB analysis. In the end, we did not include them as we hone in on the discussion regarding HML's role in the five- and six-factor models.

A substantial part of this thesis is built upon a rather involved copula model. While we are generally comfortable with the estimation procedure and robustness of the model, we acknowledge that it does lack the power to properly explain asymmetries in tail dependence. A natural route for an extension to a more flexible methodology would be to consider vine copulas in place of the multivariate copula we use. We also believe that the advances of regime switching models could prove fruitful in the factor setting, as such models can more rapidly adjust to shocks.

In unreported results, we have also studied out-of-sample investing with factor timing based on the copula model, but find results to be lackluster. While the copula model can ex post shed light on the roles of different factors, it is not useful for a priori portfolio allocation. This is consistent with the preference of money managers to use static weights, but stands in contrast to the work of \textcite{ChristoffersenLanglois2013}, upon whose work this thesis is largely based.\footnote{See i.a. \textcite{AQRSiren}, \textcite{BlackRock}, \textcite{MSCI} and \textcite{Robeco}.} Out-of-sample factor timing is hard, and please note that we do not purport to create a model for investment uses -- our contribution is only possible ex post. While MV and CDB analysis based on dynamic weights may seem counter-intuitive at first, as investors use static weights, we argue that the dynamic analysis is a powerful tool for evaluation purposes.

A final thought regarding factor strategies is that we should be careful in interpreting the factors as long-only strategies. While it is highly likely that some of the factors generate alpha both in the long and short positions, \textcite{Wang2013} shows that the profitability factor is mainly due to alpha on the short positions. Therefore, we emphasize that our findings on factors are applicable only directly to the long-short version of the factors.
