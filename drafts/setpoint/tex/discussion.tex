%!TEX root = ../main.tex

\section{Discussion and conclusion} % (fold)
\label{sec:discussion_conclusion}

The first key finding of this thesis is that the classic value factor, HML, is still highly relevant for factor investors. 

Zero-cost regressions in the five-factor model suggest that HML may be fully explained by the remaining four factors, but we find evidence to the contrary. The actual implementation of mean-variance optimizations under the only constraint of non-negative weights gives a positive allocation to the HML factor. First, we have shown this using constant sample estimators as inputs in the mean-variance model on data 1963--2016. The sceptic reader will, however, object to the conclusion based on sample data alone, as a model is never better than its inputs; and sample estimates do not give a sense of optimal weights over time.

We show that, in our data, there is clear evidence of non-normality and time-variation in the dependency across factors. A mean-variance optimization that is run solely on the basis of sample estimates will ignore these features of the data, and the positive allocation to HML might be an uninteresting solution to the optimization problem.

To mitigate these concerns, we also run the mean-variance optimization using dynamic inputs from a copula model, which can generate a conditional distribution of returns that captures both the non-normality and the time-variation in dependency. Under such inputs, the allocation to HML is even greater, indicating that the classic value factor is even more attractive than sample inputs suggest.

Although variance is the staple measure of risk, we also investigate whether risk beyond the first two moments can provide reasons against the HML factor. The copula model has allowed us to infer the full distribution of returns, and we shift our focus to the tail of the distribution. Here, we find that the diversification benefit of HML is at least as great, if not greater, than that of CMA. HML can by no means be considered a worse diversifier against tail risk, as measured by expected shortfall, than CMA.

We believe that the reason why zero-cost regressions indicate that HML does not add value is that the regressions are misspecified. The omission of an important sixth factor strategy, momentum, creates a bias on all the factor loadings, and leads to not recognizing the added value of HML. While the role of HML as a unique addition was already defended by \textcite{Asness2015}, we have pursued the argument and can conclude that optimizations lead to positive allocations to the HML strategy.

Our second key finding is that HML is highly similar, but not quite the same as CMA. On a closer look, differences emerge, and we believe that factor investors should combine both factors with consideration.

There is important theoretical and empirical support for a overlap in the stock positions that comprise HML and CMA, which results in a substantially higher correlation for this factor pair than for any other factor pair. The dependence between the two factors is also more stable, and does not exhibit the same pattern of tail dependece as do other factor pairs. 

Still, one factor cannot replace the other -- they do exhibit different properties, as HML firms are more profitable and exhibit less momentum, and generate return premia that mean-variance optimization suggests combining. Beyond the risk-reward of the first two moments, analysis of the full return distribution from the copula model shows that the diversification benefit of adding either HML or CMA to a portfolio is not constant over time. Sometimes HML is the better diversifier, sometimes CMA is better, but no pattern emerges as to which factor is better than the other. We see no reason for investors to choose either one or the other, as both provide valuable diversification, and even better, they do so at different times.

However, we strongly believe that investors should consider the two factors jointly when building portfolios: When one of the two factors is included to a factor portfolio already containing the second, the first factor almost exclusively cannabilizes on the weight from the second. Our findings support the existing theoretical and empirical evidence of an overlap in the firms that comprise the two strategies. All in all, we are wary of factor weighting schemes that suggest pure equal-weights for HML and CMA. While such schemes are valuable for factor investing in general, as they avoid the pitfall of factor (mis)timing, they should be designed in a manner that takes the close link between HML and CMA into account. This pair has a very different dependence from all other pairs -- and so should also the allocation policy be different.

A third finding of this thesis is the strength of the profitability factor. The factor covaries negatively with most factors and receives zero or negative factor loadings in zero-cost regressions. The exclusion of RMW in our diversification benefit analysis completely pulls the plug on diversification, making periods of low diversification both more frequent and much more severe. Furthermore, the factor receives high allocations both using sample and model inputs in the mean-variance framework. The fact that there are no strong explanations of the profitability factor as a risk premium makes these findings even more puzzling: While investing in quality firms is by no means a new notion, there are always two investors to every trade, and the reasons for not holding profitable firms are unclear. Our takeaway is that all funds and investors in the factor space should seriously consider adding this new factor.

During the writing of this thesis, we also considered studying additional emerging factors, in particular the low volatility and betting against beta factors. At a first glance, we found highly diversifying characteristics of these factors, which remind us of RMW. It would be especially interesting to study their impact on tail risk, following the conditional diverisfication benefit analysis. We did in the end not include them as we hone in on the discussion regarding HML's role in the five- and six-factor models.

A substantial part of this thesis is built upon a rather opaque and involved copula model. While we are generally comfortable with the estimation procedure and robustness of the model, we acknowledge that it does lack the power to properly explain asymmetries in tail dependence. A natural route for an extension to a more flexible methodology would be to consider vine copulas in place of the multivariate copula we use. We also believe that the advances of regime switching models could prove fruitful in the factor setting, as such models can more rapidly adjust to shocks.

In unreported results, we have also studied out-of-sample investing with factor timing based on the copula model, but find results to be lackluster. While the copula model can ex post shed light on the roles of different factors, it is not useful for a priori portfolio allocation. This is, however, highly coherent with the preference of money managers to use static weights.\footnote{See i.a. \textcite{AQRSiren}, \textcite{BlackRock}, \textcite{MSCI} and \textcite{Robeco}.} Out-of-sample factor timing is hard, and please note that we do not purport to create a model for investment uses -- our contribution is only possible ex-post. While MV and CDB analysis based on dynamic weights may seem counter-intuitive at first, as investors use static weights, we argue that the dynamic analysis is a powerful tool for evaluation purposes.