%!TEX root = ../main.tex

\section{CDB optimization} % (fold)
\label{sec:cdb_optimization}

We now consider the relative diversification benefit of HML, CMA and RMW. We study the evolution of optimal CDB over time for different five- and six-factor universes, and experiment with the exclusion of one of HML, CMA and RMW at a time. The intuition behind this exercise is to see how much diversification is lost if we can no longer invest in a given factor. Does excluding HML make the portfolio less diversified than excluding CMA? And what is the impact of the other new factor, RMW? We present results based on a Value-at-Risk cut-off of 5\%.\footnote{Results based on lower values (e.g. 1\%) are found to be qualitatively similar.} 

% Picture different between 5-factor and 6-factor
\autoref{fig:cdb} plots optimal conditional diversification benefit measures of the five- and six-factor asset universes, where we experiment by excluding HML, CMA and RMW one at a time. We have smoothed the plots using quarterly moving averages in order to make them easier to read. We proceed with a number of interesting results that emerge from this picture:

\begin{figure}[!ht]
  \centering
  \footnotesize
  \renewcommand{\arraystretch}{1.2}
  \caption{5\% Optimal Conditional Diversification Beneift (CDB)}

  \begin{longcaption}
    Five (without Momentum) and six factor universes. The line has been smoothed with a moving average on a quarterly window to make it easier to read.
  \end{longcaption}
  \label{fig:cdb}
  \includegraphics[scale = 1]{graphics/cdb/CDB.png}
\end{figure}

% CDB is very high; factor strategies are good diversifiers
First, we note that regardless of whether momentum is included or not, factor strategies appear to offer high levels of diversification. In absolute terms, all strategies fluctuate in the 80--95 range for the majority of the studied time period. 

% Dips
Second, there are notable dips in the diversification benefit measure. The dips represent times when diversification is relatively hard to come by, and roughly coincide in the five- and six-factor models. Interestingly, the periods of low diversification do not seem be stock market crises, as the CDB measure remains relatively high during the 1999-2000 bubble and the 2007-2009 recession.

% HML is a better diversifier on average
% CMA better when diversification is hard to come by
Third, the level decreases in diversification benefit of removing HML or CMA seem quite small. Furthermore, this decrease is highly similar; At certain times, portfolios including HML are more diversified and vice versa, but no pattern emerges. However, we note that the exclusion of RMW is dramatically different. Without RMW, the level decrease is substantial and dips in CDB become much more pronounced and frequent. 

%!TEX root = ../../main.tex

\begin{table}
  \centering
  \footnotesize
  \renewcommand{\arraystretch}{1.2}

  \caption{CDB Optimization with Dynamic Copula Model (1963--2016)}

  \begin{longcaption}
    Average weights are averages of dynamic CDB optimal weights based on simulations of the return distribution from the copula model. Differences in average weights are expressed relative to the full five- and six-factor models. Performance measures are based on realized returns. SR is the annualized Sharpe Ratio. VaR, ES and CDB are all based on the one-week-ahead 5\% lower tail of the return distribution. Differences in CDB are to be read as column model minus row model and its associated standard errors (in parentheses) are computed taking the copula model as given.
  \end{longcaption}

  \label{tab:cdb_model}

  \begin{tabularx}{\textwidth}{@{} l dddd X dddd @{}}
    \toprule
    &
      \multicolumn{4}{c}{Five Factors} &&
      \multicolumn{4}{c}{Six Factors} \\
    \cmidrule{2-5}
    \cmidrule{7-10}
    &
      \multirow{2}{*}{All} &
      \multicolumn{1}{c}{Excl.} &
      \multicolumn{1}{c}{Excl.} &
      \multicolumn{1}{c}{Excl.} & &
      \multirow{2}{*}{All} &
      \multicolumn{1}{c}{Excl.} &
      \multicolumn{1}{c}{Excl.} &
      \multicolumn{1}{c}{Excl.} \\
    &
      &
      \multicolumn{1}{c}{HML} &
      \multicolumn{1}{c}{CMA} &
      \multicolumn{1}{c}{RMW} &&
      &
      \multicolumn{1}{c}{HML} &
      \multicolumn{1}{c}{CMA} &
      \multicolumn{1}{c}{RMW} \\
    \midrule
    \multicolumn{1}{@{}l}{\textbf{Average weights}} \\
    Mkt.RF & 11.1 & 10.5 & 11.5 & 19.2 & & 10.5 & 10.1  & 10.6 & 15.5 \\
    SMB    & 16.6 & 18.3 & 19.1 & 22.6 & & 15.8 & 17.9 & 18.1 & 19.2 \\
    HML    & 17.4 &      & 30.1 & 26.7 & & 18.1 &      & 28.7 & 24.9 \\
    CMA    & 21.2 & 35.0 &      & 31.6 & & 18.7 & 32.2 &      & 24.1 \\
    RMW    & 33.8 & 36.2 & 39.3 &      & & 28.1 & 31.8 & 32.6 & \\
    Mom    &      &      &      &      & &  8.8 & 8.1  & 10.0 & 16.2 \\
    \midrule
    \multicolumn{1}{@{}l}{\textbf{Difference weights}} \\
    Mkt.RF & & -0.6  & 0.4   & 8.1   & & & -0.4  & 0.2   & 5.1 \\
    SMB    & & 1.7   & 2.5   & 6.0   & & & 2.1   & 2.3   & 3.4 \\
    HML    & & -17.4 & 12.7  & 9.3   & & & -18.1 & 10.6  & 6.8 \\
    CMA    & & 13.9  & -21.2 & 10.4  & & & 13.5  & -18.7 & 5.4 \\
    RMW    & & 2.4   & 5.5   & -33.8 & & & 3.7   & 4.4   & -28.1     \\
    Mom    & &       &       &       & & & -0.7  & 1.3   & 7.5 \\
    \midrule
    \multicolumn{1}{@{}l}{\textbf{Performance}} \\
    Mean (\%)      & 2.77  & 2.94  & 2.85  & 3.37  & & 3.37  & 3.39  & 3.41  & 3.90 \\
    SD (\%)        & 2.42  & 2.49  & 2.69  & 3.89  & & 2.37  & 2.52  & 2.56  & 3.49 \\
    SR             & 1.14  & 1.18  & 1.06  & 0.87  & & 1.42  & 1.34  & 1.33  & 1.12 \\
    Avg. VaR  (\%) & 0.46  & 0.48  & 0.52  & 0.78  & & 0.45  & 0.47  & 0.49  & 0.69 \\
    Avg. ES  (\%)  & 0.61  & 0.64  & 0.69  & 1.04  & & 0.60  & 0.64  & 0.66  & 0.92 \\
    Avg. CDB       & 90.42 & 89.29 & 89.19 & 83.42 & & 91.33 & 90.24 & 90.51 & 86.62 \\
    \midrule
    \multicolumn{1}{@{}l}{\textbf{Difference CDB}} \\
    All       & & -1.13  & -1.23  & -7.01  & & & -1.09  & -0.82  & -4.71 \\
              & & (0.02) & (0.03) & (0.11) & & & (0.02) & (0.03) & (0.10) \\
              \\
    Excl. HML & &        & -0.10  & -5.88  & & &        & 0.27   & -3.62 \\
              & &        & (0.04) & (0.11) & & &        & (0.04) & (0.10) \\
              \\
    Excl. CMA & &        &        & -5.78  & & &        &        & -3.89 \\
              & &        &        & (0.11) & & &        &        & (0.10) \\
    \bottomrule
  \end{tabularx}
\end{table}


\autoref{tab:cdb_model} displays CDB summary statistics and results of paired t-tests of CDB difference between strategies (column less row strategy). This table tells largely the same story as the previous graph. In a five-factor model, excluding CMA leads to significantly lower CDB compared to excluding HML (i.e. CMA is more important as a diversifier), however, the effect is reversed in six-factor model. Furthermore, the average differences, $-0.10$ and $0.27$ respectively, are not very large -- especially compared to the effect of excluding RMW.\footnote{Standard errors are computed ignoring uncertainty in the copula model parameters, which means that the significance is overestimated.}

The difference in how CDB and MV allocates can be seen in~\autoref{fig:mv_cdb_weights}.

\begin{figure}[!ht]
  \centering
  \footnotesize
  \renewcommand{\arraystretch}{1.2}
  \caption{Mean-variance and CDB optimal weights}

  \begin{subfigure}{0.45\textwidth}
    \includegraphics[width=\textwidth]{graphics/weights/compare_Weights_CDB_MV_5F.png}
    \caption{Five Factors}
  \end{subfigure}
  ~
  \begin{subfigure}{0.45\textwidth}
    \includegraphics[width=\textwidth]{graphics/weights/compare_Weights_CDB_MV_6F.png}
    \caption{Six Factors}
  \end{subfigure}  

  \label{fig:mv_cdb_weights}
\end{figure}

In summary, we find that the high similarity of HML and CMA indicates that tail diversification benefits are not dramatically improved by including both the factors, which is coherent with fact that they are closely related and overlap. This does not mean that both factors should not be considered, however, as it could improve the conventional risk-return tradeoff in a mean-variance setting. The RMW factor, on the other hand, is shown to be very important for diversification purposes and should be considered by all factor investors concerned with tail risk.

% section cdb_optimization (end)
