%!TEX root = ../main.tex
\section{Introduction}
% Intro to the discussion of HML's role
\textcite{FF2015} find that the value factor (HML) is redundant in a five-factor model including profitability (RMW) and investment (CMA), as it has no explanatory power on monthly returns in a US 1963–2010 sample. This could mean that the classic value factor is an inferior proxy for what truly comprises value, and the paper has sparked a debate on whether the HML factor is a poor proxy for the value effect. In response to~\textcite{FF2015},~\textcite{Asness2015} resurrect HML by including momentum as a factor.

% Where do we start
In this paper, we put ourselves in the shoes of an investor who optimizes factor exposures of her portfolio, and is curious about what the new factors bring to the table. We consider the six factors most commonly discussed in the literature: Market (Mkt.RF), Value (HML), Size (SMB), Momentum (Mom), Profitability (RMW) and Investment (CMA), as available from Kenneth French's data library. Specifically, we focus on the role of HML in factor investing, against the background of~\textcite{FF2015,Asness2015}, and its apparent similarity to CMA. We also consider the impact on risk and reward of including RMW.

% Ok, so tell me more about why HML -- CMA is interesting to look at
While most factor pairs exhibit relatively low correlation, the HML--CMA pair stands out with an unconditional correlation in our sample period (1963-2016) of 0.63. There is reason to expect a degree of overlap between the portfolios that comprise HML and CMA: Value firms invest less, and growth firms invest more, and there is a negative empirical relation between past investment and current book-to-market ratios~\autocite{Zhang2005,AndersonGarciaFeijoo2006}.

% That is interesting. What does that mean for an investor who doesn't consider the similarity?
The value premium has been explained both as a rational risk premium, that compensates the bearer for taking on some risk that materializes in bad times, and as an anomaly, that exists due to market frictions or investor irrationality. If the premia earned on these strategies are compensations for risk, the source of risk in the HML and CMA factors might in fact be the same. An investor that then allocates assets to both these factors, might unwillingly double up on exposure to the same risk source. Similarly, if the premia earned are due to market frictions and irrational investor behavior, the naïve investor would double up on exposure to the risk that the anomaly goes away. We believe that, regardless of the interpretation of the value premium, there is reason to place additional emphasis on the role of HML relative to CMA from a portfolio choice and risk perspective.

% Ok, so exactly what was the question?
Our main research question can thus be expressed as: What role should HML play in factor investing given the discovery of CMA and RMW?  Our secondary research question regards RMW: What is the risk and reward characteristics of including the profitability factor? We use both a conventional risk-reward perspective using mean-variance analysis, and then consider tail risk using a new measure of diversification benefits, which based on expected shortfall.

% Interesting, where do we begin?
We begin by revisiting the regressions in \textcite{FF2015} that show a zero alpha to the HML factor in the five-factor model, as well as the regressions in \textcite{Asness2015} that include momentum in a six-factor model. \textcite{FF2015} find zero alpha of adding HML to a four-factor portfolio, and conjecture that this implies that a mean-variance investor would in fact not improve the tangent portfolio by including HML. \textcite{Asness2015} challenge the notion that the value factor is subsumed by the addition of investment and profitability. In their study, they resurrect the alpha of the value factor by adding a momentum factor and slightly modifying the HML factor, suggesting that HML is beneficial in a six-factor portfolio.

% Any results?
We find similar regression results to those in these papers in our weekly data set spanning 1963--2016. However, while the regressions of zero-cost portfolios in \textcite{FF2015} and \textcite{Asness2015} indicate which factors should be included in mean-variance investing, neither paper actually carries out mean-variance optimization of portfolios. This thesis fills that gap by optimizing weights of both five- and six-factor portfolios. In addition to a sample analysis with constant weights, we investigate the optimal dynamic weights, based on simulations from a conditional model of return series, which incorporates time-varying and non-normal properties of the data.

% What then? You talked about tail risk?
Having concluded the analysis of weights and the performance in mean-variance optimal portfolios, we proceed with analyzing the relative diversification benefits among factor strategies. While mean-variance investing optimizes the expected return to volatility tradeoff, this analysis shifts the focus from the first two moments to the tail of the portfolio return distribution. Could it be that HML should receive no allocation due to higher order risk features, which are not present in CMA, and vice versa? 

We investigate a new measure of relative diversification benefit proposed by \textcite{ChristoffersenErrunzaJacobLanglois2012}: the \emph{conditional diversification benefit} (CDB) statistic, which measures how close a set of portfolio weights makes the expected shortfall of a factor portfolio to the portfolio's Value-at-Risk. Using simulations from the conditional model of returns, the CDB illustrates how well diversified portfolios are over time, and across different allowed asset universes. In particular, we interest ourselves with the difference in diversification benefits of adding either HML or CMA to a asset universe of the remaining factor strategies.

% Any results?
% OK, what did you base it all on?
For both the mean-variance and diversification benefit analysis, the choice of conditional return model is central. While the ARMA-GARCH family of models are the norm of univariate time-series modeling, multivariate modeling has proven harder as the multivariate extensions of such models are often computationally infeasible and ridden with dimensionality problems. Recently, however, copula models have attracted considerable attention in the risk management field, as they offer a numerically stable and flexible way of estimating joint probability distributions. 

Following closely the method of \textcite{ChristoffersenLanglois2013}, we build a copula model of the joint factor returns. The specification we use is designed to recognize time-varying correlations and tail dependence, which is the notion that there might be significantly different dependence patterns when returns simultaneously realize in the lower or upper tail, as opposed to close to the center of the joint distribution. This becomes especially important when we consider the conditional diversification benefit, as it concerns the tail of the return distribution. In robustness tests, the copula model is shown to be capable to be generate correlation patterns in the data. It can also, to a limited extent, explain tail dependence.

Based on both sample and model inputs, our mean-variance analysis shows that HML does indeed improve the tangency portfolio, subject to the constraint of non-negative weights. However, it also highlights the risk of over-allocating to HML as the factor is highly similar to CMA.

Results from the diversification benefit analysis gives no reason to believe that HML and CMA differ substantially in terms of tail risk. We also find that removing one of HML or CMA from a five- or six-factor portfolio has a very modest impact on tail risk, while removing RMW significantly worsens the diversification benefits.

 
% Vaska?
% The structure of this thesis is as follows: in \autoref{sec:literature} we provide a brief literature review of factor strategies and the work on copulas relating to factors. In \autoref{sec:data} we present the data used and detail the construction of factors. In \autoref{sec:alpha_reg}, we revisit the abnormal return regressions of \textcite{FF2015} and \textcite{Asness2015}. In \autoref{sec:modeling_of_factor_returns}, we present our model of factor returns, interleaving method and results. In \autoref{sec:mean_variance} we present analysis of mean-variance optimizations. In \autoref{sec:conditional_diversification_benefit}, we examine the diversification benefits of different factor portfolios. \autoref{sec:discussion_conclusion} summarizes our findings and discusses their wider implications.
