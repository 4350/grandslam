%!TEX root = ../main.tex
\section{Introduction}
% Intro to the discussion of HML's role
\textcite{FF2015} find that the value factor (HML) is redundant in a five-factor model including investment (CMA) and profitability (RMW), as it has no explanatory power on monthly returns in a US 1963--2013 sample. This could mean that the classic value factor is an inferior proxy for what truly comprises value, and the paper has sparked a debate on whether the HML factor is a poor proxy for the value effect. In response to~\textcite{FF2015},~\textcite{Asness2015} resurrect the explanatory power and value of HML in a portfolio context, by including momentum as a factor.

% Where do we start
In this paper, we put ourselves in the shoes of an investor who optimizes factor exposures of a portfolio, and is curious about what the new factors bring to the table. We consider the six factors most commonly discussed in the literature: market (Mkt.RF), size (SMB), value (HML), investment (CMA), profitability (RMW) and momentum (Mom), as available from Kenneth French's data library. Specifically, we focus on the role of value (HML) in factor investing, and its apparent similarity to investment (CMA), against the background of~\textcite{FF2015} and \textcite{Asness2015}. We also consider the impact on risk and reward of including the other new factor, profitability (RMW).

% Build up story why HML--CMA is interesting
While most factor pairs exhibit relatively low correlation, the HML--CMA pair stands out with an unconditional correlation in our sample (weekly returns 1963--2016) of 0.63. Value firms have been found to invest less, and growth firms to invest more, and there is a negative empirical relation between past investment and current book-to-market ratios~\autocite{Zhang2005,AndersonGarciaFeijoo2006}. Therefore, there is reason to expect a degree of overlap between the portfolios that comprise HML and CMA.

The value premium has been explained both as a rational risk premium, that compensates the bearer for taking on some risk that materializes in bad times, and as an anomaly, that exists due to market frictions or investor irrationality. If the premia earned on these strategies are compensations for risk, the source of risk in the HML and CMA factors might in fact be the same. An investor who allocates to both these factors, might unwillingly double up on exposure to the same risk source. Similarly, if the premia earned are due to market frictions and irrational investor behavior, the naïve investor would double up on exposure to the risk that the anomaly goes away. We believe that, regardless of the interpretation of the value premium, there is reason to place additional emphasis on the role of HML relative to CMA from a portfolio choice and risk perspective.

% HML--CMA question
Our main research question can thus be expressed as: What role should value (HML) play in factor investing given the discovery of investment (CMA) and profitability (RMW)?

% Build up story why RMW is interesting
CMA was included in the five-factor model jointly with RMW. RMW is typically considered an anomaly, as it has proven to be especially hard to rationalize as a risk premium~\autocite{Wang2013}. High profitability is generally indicative of a favorable market position, a strong brand and a host of competitive advantages, which all contribute to lower risk. The reverse holds for the unprofitable firms. Furthermore, RMW has had high positive returns in times of crisis, indicating that the factor shields against risk at times when protection is needed the most. There seem to be strong benefits to including such a factor in a portfolio, and we believe that the RMW factor deserves to be analyzed separately.

%RMW question
Our secondary research question therefore considers: What is the impact on the risk-reward trade-off of including profitability (RMW) in factor portfolios? 

To answer these questions, we optimize portfolios based on two different strategies and compare the resulting risk-reward profiles of including HML, CMA and RMW respectively.

%Regressions
Before delving into optimization methods, we revisit the zero-cost regressions of \textcite{FF2015} and \textcite{Asness2015}. In these regressions, each factor is separately regressed on all the remaining factors, to determine whether there is any additional abnormal return to the left-hand side (LHS) factor after accounting for the variation explained by the right-hand side (RHS) factors. In other words, such regressions examine whether the LHS factor is subsumed by the RHS factors, in which case it has no significantly estimated intercept. In fact, the intercept is also equivalent to the Jensen's alpha of including the LHS variable to a portfolio of the RHS factors~\autocite{Jensen1968}. \textcite{FF2015} find zero alpha of adding HML to a four-factor portfolio, and conjecture that this implies that a mean-variance (MV) investor would in fact not improve the tangent portfolio's Sharpe Ratio (SR) by including HML. \textcite{Asness2015} challenge the notion that HML factor is subsumed by the addition of CMA and RMW. In their study, they resurrect the alpha of the value factor by adding a momentum factor to the regressions and by modifying the HML factor.

%Regression results
We find similar regression results in our weekly data set of factor returns 1963--2016. However, while the regressions of zero-cost portfolios in \textcite{FF2015} and \textcite{Asness2015} indicate which factors should be included in MV investing, neither paper actually carries out MV optimization of portfolios including CMA and RMW. This thesis fills that gap by optimizing weights of both five-factor (as in \textcite{FF2015}) and six-factor portfolios (including momentum as in \textcite{Asness2015}). 

% Optim 1 MV
Thus, the first optimization method is mean-variance (MV) analysis -- a conventional risk-reward perspective, where weights are chosen to maximize the Sharpe Ratio of the portfolio. However, MV analysis only considers the first two central moments of the return distribution. Factor strategies have been shown to be inherently non-normal: they have high levels of skewness and kurtosis and also exhibit tail dependence, i.e. the notion that there might be significantly different dependence patterns when returns simultaneously realize in the lower or upper tail, as opposed to close to the center of the joint distribution~\autocite{ChristoffersenLanglois2013}. Means and covariances provide an incomplete description of factor return distributions, and therefore also of the risk in factor portfolios. Regardless of the result of MV analysis, the non-normal features might constitute another reason altogether to include (or exclude) either HML, CMA or RMW.

% Optim 2 CDB
To analyze the lower tail of the distribution, the second optimization method is based on a measure of diversification benefit, the \emph{conditional diversification benefit} (CDB) statistic, introduced in~\textcite{ChristoffersenErrunzaJacobLanglois2012}. CDB is based on Expected Shortfall (ES) and measures the diversification benefit in the tail of the distribution. The statistic studies how close a portfolio's ES is to the portfolio's Value-at-Risk (VaR), and provides an additional dimension to the risk-reward trade-off in the MV setting. In optimization, weights are chosen to maximize tail diversification (i.e. maximize the CDB statistic).

% Intro model
While MV analysis could be carried out using static sample estimators of means and covariances alone, CDB analysis is impossible without a model of returns, from which ES and VaR can be simulated. Furthermore, a conditional model also allows us to study dynamic portfolio weights. 

The choice of a return model is central. While the ARMA-GARCH model family is the norm of univariate time-series modeling, multivariate modeling has proven harder as the multivariate extensions of such models are often computationally infeasible and ridden with dimensionality problems. Recently, however, copula models have attracted considerable attention in the risk management field, as they offer a numerically stable and flexible way of estimating joint probability distributions. 

% Model specifics
Following closely the method of~\textcite{ChristoffersenLanglois2013}, we build a copula model of the joint factor returns. The specification we use is designed to recognize time-varying correlation and tail dependence, which are two important features of factor returns~\autocite{ChristoffersenLanglois2013}. We measure time-varying correlation with rolling one-year correlation and tail dependence with threshold correlation (also known as exceedance correlation), i.e. the linear correlation when factor returns simultaneously realize in the upper or lower tail~\autocite{AngChen2002}. In in-sample robustness tests, our copula model is shown to generate the time-varying correlation patterns in the data. It can also, to a limited extent, reproduce the tail dependence.

% Results MV
Based on copula estimates of means and covariances, our MV optimization shows that HML does indeed improve the tangency portfolio's Sharpe Ratio, subject to the constraint of non-negative weights. HML receives an average portfolio weight of 18\% and improves the SR by 0.16 in the five-factor model. We therefore agree with the conclusions of~\textcite{Asness2015}, and suggest that the discussion about HML's redundancy is likely to be caused by omitting the momentum factor from the zero-cost regressions in~\textcite{FF2015}.

However, our MV analysis also highlights the risk of over-allocating to the value factor as HML is highly similar to CMA. When HML is included, it mainly cannibalizes on the weight that CMA had before, and vice versa, indicating that the variables proxy for each other to a high extent. This is in line with the theoretical support for an overlap in the stocks that comprise HML and CMA. Investors who do not consider this similarity risk over-allocating to the same return premium.

We also find that RMW has a much greater impact on the tangency portfolio than do HML and CMA. When RMW is excluded, the realized Sharpe Ratio falls 0.36, compared to a drop of 0.16 and 0.10 for HML and CMA, respectively, in the five-factor model. This illustrates the unique diversifying nature of RMW, which is not captured or proxied well by the other factors.

We find that all of our MV results are qualitatively similar, albeit less pronounced, when full sample estimators of means and covariances are used instead of model inputs.

% Results CDB
Having considered only means and variances in the MV analysis, we shift the focus to the tail of the distribution in the CDB analysis and investigate whether HML and CMA differ substantially in terms of their contribution to tail risk. The CDB is alternately higher for the exclusion of HML or CMA, but no pattern emerges. 

Based on the CDB measure of tail risk, there is no reason to remove HML from factor investing. However, we find that excluding either one of HML or CMA has a very modest impact on tail risk. We interpret this as another sign of the overlap between HML and CMA. As in the mean-variance analysis, the factors proxy for each other, making an exclusion less dramatic. Still, we see no reason for investors to choose either one or the other, as both provide valuable diversification, and even better, they do so at different times.

Excluding RMW has much greater impact. The CDB drastically worsens, with substantially greater and more frequent declines. We find that the RMW factor is very efficient in reducing tail risk.
 
% Vaska?
% The structure of this thesis is as follows: in \autoref{sec:literature} we provide a brief literature review of factor strategies and the work on copulas relating to factors. In \autoref{sec:data} we present the data used and detail the construction of factors. In \autoref{sec:alpha_reg}, we revisit the abnormal return regressions of \textcite{FF2015} and \textcite{Asness2015}. In \autoref{sec:modeling_of_factor_returns}, we present our model of factor returns, interleaving method and results. In \autoref{sec:mean_variance} we present analysis of mean-variance optimizations. In \autoref{sec:conditional_diversification_benefit}, we examine the diversification benefits of different factor portfolios. \autoref{sec:discussion_conclusion} summarizes our findings and discusses their wider implications.
