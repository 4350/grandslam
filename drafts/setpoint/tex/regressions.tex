%!TEX root = ../main.tex

\section{Zero-cost portfolio regressions}
\label{sec:alpha_reg}
\textcite{FF2015} and \textcite{Asness2015} run factor regressions where both the LHS variable and the RHS variables are zero-cost factor portfolios. The intercept in this type of regression is to be interpreted as the abnormal return, or Jensen's alpha, of adding the LHS factor to a portfolio already consisting of the RHS factors \autocite{Jensen1968}. In this section, we replicate and discuss the regressions where HML, CMA and RMW are the LHS variables, and find that previous results hold up in our weekly data set.

\subsection{Model specification}

As a specific example, we begin by considering the regression that has caused the discussion on whether HML is a redundant factor. \textcite{FF2015} run the regression
\begin{align}
  r^{HML}_t = \alpha + \beta_1 r^{Mkt.RF}_t + \beta_2 r^{SMB}_t + \beta_3 r^{RMW}_t + \beta_4 r^{CMA}_t + \epsilon_t
\end{align}
where $r^i_t$ denote monthly returns. The central finding is that HML is completely subsumed by the four factors Mkt.RF, SMB, RMW and CMA -- i.e. the alpha of the regression is very small and not statistically significant. In other words, adding HML to a portfolio of the other four factors should give no abnormal return.

Our regression analysis deviates from that in \textcite{FF2015} as we consider weekly return data and approximately two more years of recent data. However, the main regression specifications are the same. We also consider a six-factor model, as done by \textcite{Asness2015}, who show that there is in fact added value of HML when momentum is included. We use standard errors that are adjusted for serial correlation found in the return data, following~\textcite{NeweyWest1987}.

\subsection{Regression results}

In \autoref{fig:abnormal}, regressions for the five-factor (excluding momentum) and six-factor (including momentum) models are presented. Each column represents one unique regression, with one of the factors as the LHS variable and with the remaining four (or five) factors as RHS variables (in rows).

First, we examine regression (1) in a five-factor model where HML is the LHS variable. We note that the alpha of HML is not significant, indicating that the factor is completely subsumed by the remaining four factors and does not create additional value in a portfolio setting, in line with~\textcite{FF2015}. More specifically, the only factor that explains HML is CMA, with a high coefficient of 0.85, with all other factor loadings insignificant and close to zero. Put differently, this suggests that for a factor portfolio already loaded on Mkt.RF, SMB, CMA and RMW, adding HML will load up additionally on CMA risk plus the idiosyncratic risk of HML, without adding any additional return.

Second, we turn to regression (2) in a five factor model where CMA is the LHS variable. Here, the alpha is significant, indicating that the factor does provide an additional 0.06\% weekly beyond the existing four factors. While the CMA factor loads positively 0.39 on HML, this is substantially lower than HML's loading on CMA of 0.85. With CMA as the LHS variable, there is a significant negative loading on Mkt.RF, a significant negative loading on RMW and a significant positive loading on Mom. The CMA portfolio loads relatively less on the market, relatively more on unprofitable stocks (negative RMW) and relatively more on stocks with high recent returns (positive Mom) than does HML.

Third, we study regression (3) where RMW is the LHS variable. The alpha is significant, indicating that the RMW factor adds abnormal return of 0.09\% weekly to the four existing factors. In terms of factor loadings, profitability loads zero or negatively on all four remaining factors in the five-factor model. This is evidence of the diversification that RMW provides. The low explanatory power of the other factors is summarized by a low 15\% $R$-squared.

%!TEX root = ../../main.tex

\begin{table}
  \centering
  \footnotesize
  \renewcommand{\arraystretch}{1.2}

  \caption{Zero-cost portfolio regressions (1963--2016)}

  \begin{longcaption}
    Six regressions of zero-cost equity factor portfolios on $N = 2766$ weekly returns 1963--2016, following the analysis of~\textcite{FF2015} and~\textcite{Asness2015}. Alpha and Beta (factor loadings) of the column's portfolio on other factors. Heteroskedacity and autocorrelation robust standard errors in parentheses, following~\textcite{NeweyWest1987}. Significance given by $^{*}p<10\%$; $^{**}p<5\%$; $^{***}p<1\%$
  \end{longcaption}
  \label{fig:abnormal} 
\begin{tabularx}{\textwidth}{@{\extracolsep{0pt}}X d d d d d d d } 
\toprule
& \multicolumn{3}{c}{Five Factors} & & \multicolumn{3}{c}{Six Factors} \\ 
\cmidrule{2-4} \cmidrule{6-8}
 & \multicolumn{1}{c}{(1)} & \multicolumn{1}{c}{(2)} & \multicolumn{1}{c}{(3)}   & & \multicolumn{1}{c}{(4)} & \multicolumn{1}{c}{(5)} & \multicolumn{1}{c}{(6)} \\
 & \multicolumn{1}{c}{HML} & \multicolumn{1}{c}{CMA} & \multicolumn{1}{c}{RMW}   & & \multicolumn{1}{c}{HML} & \multicolumn{1}{c}{CMA} & \multicolumn{1}{c}{RMW} \\
\midrule \\ 
 \text{Alpha (\%)} & 0.02       & 0.06^{***}  & 0.09^{***}  & & 0.05^{**}   & 0.04^{***}  & 0.09^{***} \\
                   & (0.02)     & (0.01)      & (0.02)      & & (0.02)      & (0.01)      & (0.02) \\
  \\
 \text{Mkt.RF}     & -0.02      & -0.11^{***} & -0.08^{***} & & -0.03       & -0.10^{***} & -0.07^{***} \\
                   & (0.03)     & (0.02)      & (0.01)      & & (0.03)      & (0.01)      & (0.01) \\
  \\
 \text{SMB}        & 0.001      & -0.03       & -0.24^{***} & & 0.01        & -0.03^{*}   & -0.24^{***} \\
                   & (0.03)     & (0.02)      & (0.04)      & & (0.03)      & (0.02)      & (0.05) \\
  \\
 \text{HML}        &            & 0.39^{***}  & -0.01       & &             & 0.42^{***}  & 0.01 \\
                   &            & (0.04)      & (0.06)      & &             & (0.03)      & (0.06) \\
  \\
 \text{CMA}        & 0.85^{***} &             & -0.15^{**}  & & 0.87^{***}  &             & -0.17^{***} \\
                   & (0.04)     &             & (0.07)      & & (0.04)      &             & (0.06) \\
  \\
 \text{RMW}        & -0.02      & -0.09^{**}  &             & & 0.01        & -0.10^{**}  & \\
                   & (0.09)     & (0.04)      &             & & (0.07)      & (0.04)      & \\
  \\
 \text{Mom}        &            &             &             & & -0.18^{***} & 0.09^{***}  & 0.03 \\
                   &            &             &             & & (0.04)      & (0.02)      & (0.03) \\
\midrule
$R^2$ (\%) &
  \multicolumn{1}{D{.}{.}{2}}{39} &
  \multicolumn{1}{D{.}{.}{2}}{46} &
  \multicolumn{1}{D{.}{.}{2}}{15} & &
  \multicolumn{1}{D{.}{.}{2}}{47} &
  \multicolumn{1}{D{.}{.}{2}}{49} &
  \multicolumn{1}{D{.}{.}{2}}{15} \\ 
\bottomrule
\end{tabularx} 
\end{table}


Now, we move to the six-factor regression results, where we include the momentum factor Mom. First, in regression (4) with HML as the LHS variable, we note that the addition of HML makes the alpha of HML positive and significant -- in line with~\textcite{Asness2015}. As momentum is correlated with both the LHS and RHS factors, it constitutes an omitted variable bias on the beta factor loadings in the five-factor model. HML has a substantial negative loading on the Mom factor of -0.18, while the CMA regression instead has a positive Mom loading of 0.09, suggesting that the seemingly similar factors HML and CMA are quite different in terms of momentum properties. The momentum factor explains an additional 8\% of the variance in the HML factor. 

CMA and RMW are to a lesser extent than HML correlated with Mom, and the factor loadings in regression (5, 6) therefore change less as we go to the six-factor model.

Although we employ weekly data, our results are qualitatively similar to the results in \textcite{FF2015} as well as in \textcite{Asness2015}. The alpha of HML is only recognized in a model including momentum. This indicates that the insignificant alpha of HML in the five-factor model might be due to the omission of an important control variable, momentum, that is included in the six-factor-model.


