%!TEX root = ../main.tex
\section{Literature review}
\label{sec:literature}

We review previous literature on equity factor strategies and the summarize discussion of why there are factor premia. We also discuss the work on factor models using copula models.

\subsection{The five- and six-factor models}

\textcite{FF2015} introduce two additional factors to complement the \textcite{FamaFrench1993} three-factor asset pricing model. In what is referred to as the five-factor model, the traditional factors (market, value and size) are complemented by a profitability factor and an investment factor. Both factors represent zero-cost portfolios: the profitability factor, denoted RMW (robust minus weak), is long firms with high operating profitability and short firms with low operating profitability, and the investment factor, denoted CMA (conservative minus aggressive), is long firms with low investment rate and short firms with high investment rate. The five-factor model is found to be a siginificant improvement (in terms of explaining cross-sectional returns) to the three-factor model, and the two new factors appear to have made the value factor (HML, high minus low) redundant. More specifically, the authors show that there is no significant intercept in regressions of HML on the remaining four factors, while each of the other factors have significant intercepts in similar regressions. The returns of the HML factor appear to be fully explained by the remaining four factors. In an investment context, this can be interpreted as the HML factor adding no alpha to a portfolio holding the remaining four factors.

\textcite{Asness2015} challenge the notion that the value factor is subsumed by the addition of investment and profitability. In their study, they add a momentum factor, as well as an enhanced HML factor, and resurrect the alpha of the value factor in a six-factor model. This leads us to believe that momentum could play an important role in recognizing the effect of HML. The momentum factor was originally studied by \textcite{JegadeeshTitman1993} and has since been shown to be present in many financial return series~\autocite{AsnessMoskovitzPedersen2013}.

The profitability and investment factors have only recently made their way into academic literature. \textcite{NovyMarx2013} study the return differences between firms with high and low gross profitability and shows significant abnormal returns to a profitability factor. Profitability is also shown to have approximately the same power in predicting the cross-section of stock returns as does value (book-to-market). Furthermore, the profitability strategy is negatively correlated with value, and can improve the investing performance of a value strategy. 

\textcite{CooperGulenSchill2008} investigate investment, measured as the percentage change in total assets, and show that the related zero-cost-portfolio provides significant abnormal returns and that it has additional predictive ability in the cross-section of stock returns, taking both value and size into account.

While all other factor pairs exhibit correlations at or below zero, the value (HML) and investment (CMA) factors are highly positively correlated. \textcite{Zhang2005} predicts this positive relation in a model setting, and \textcite{AndersonGarciaFeijoo2006} confirm it on empirical data. More specifically, the empirical study shows that past investment has a significant negative relation with the book-to-market ratio. In other words, value firms with high book-to-market might be value firms precisely because they have invested little, and vice versa. \textcite{FF2015} consider it a fact that value firms invest less than growth firms.

\subsection{Factor premia -- anomalies, risk premia, or both?}
It is debated whether factor strategies constitute rational risk premia or whether they are the consequences of market imperfections and irrational behavior. There are some appealing rational stories for the return premium of HML, which also could explain the premium of CMA as there is overlap between the factors. \textcite{FamaFrench1993} show that the HML factor is related to systematic patterns of profitability and growth, and could proxy for a common risk source. This is supported by \textcite{LiewVassalou2000}, who show that the value factor can predict real GDP growth on data in several markets. \textcite{Zhang2005} uses a neoclassical model with rational expectations and competitive equilibrium to show that value firms have more tangible assets and are burdened by industry over-capacity in downturns, leading to higher down-market betas. \textcite{PetkovaZhang2005} find that the conditional betas of value stocks covary positively with the expected market risk premium. Despite there being a number of rational theories, they all predict effects that are fairly small and cannot fully motivate the value premium. So far, the most pervasive explanations of the value premium are based on market imperfections and irrational behavior.\footnote{See i.a. \textcite{Ilmanen2011}}

\textcite{LakonishokShleiferVishny1994} argue that the value factor is driven by investors' overreaction to changes in earnings. They show that value firms have often experieced a decline in earnings over the last three years, lowering their book to market ratio. When earnings have gone down, investors as a group extrapolate the trend into the future and push prices away from fundamentals, giving rise to higher average returns for value firms and vice versa. Similar to \textcite{LakonishokShleiferVishny1994}, \textcite{BarberisHuang2001} draw on the fact that value firms have experienced decreasing earnings, but suggest that the premium is driven by investors' loss aversion bias. Current value firms have had falling earnings, leading to lower share prices and negative returns. As many investors are deterred by the past performance of negative returns in itself, they are less willing to hold value stocks, and create a risk-reward upside for those who do. \textcite{LakonishokShleiferVishny1992} instead find an explanation in the structure of the money manager industry. They suggest that there money managers have career based incentives to avoid value stocks as such stocks are more likely to go bankrupt, and make the short-term performance look bad, than growth stocks, which are more widely held in the reference index of active funds.

For the profitability factor, RMW, risk based explanations are harder to come by \autocite{NovyMarx2013}. It is hard to pinpoint reasons for profitable firms to be more risky than unprofitable. \textcite{Wang2013} investigate the relationship between macro risks and the profitability factor and find risk based stories to be implausible. Instead they suggest that the RMW factor is driven mainly by systematic underreaction, causing a negative alpha in unprofitable stocks.

\subsection{Factor strategy investing}
Variations of the long-short strategy of the value factor has become a staple strategy of both quantitative and qualitative hedge funds, often under the "equity market neutral" or "fundamental quantitative" labels. Factor equity strategies have also become increasingly accessible for retail investors, especially with the advent of smart beta ETFs.\footnote{See i.a. \textcite{Pedersen2015}, \textcite{AQREMN} and \textcite{McKEMN}.} A number of large money managers including AQR, Blackrock and Robeco today provide factor investing based products, and MSCI provides indices on factor strategies. Generally, these managers advise against factor factor timing and use static strategies that are based on equal-weighting -- a simple heuristic that has proven hard to beat out-of-sample. The managers do, however, blend the equal-weights approach with optimization routines, including mean-variance and minimum-volatility, to arrive at the policy weights.\footnote{See i.a. \textcite{AQRSiren}, \textcite{BlackRock}, \textcite{MSCI} and \textcite{Robeco}.}

The use of leverage in hedge funds can exacerbate the flow patterns in factor strategies, as highlighted by the quant crash in July-August 2007. \textcite{KhandaniLo2011} and \textcite{KhandaniLo2007} revisit the sudden and large losses of factor strategies (including value and size) during this period, and provide evidence for the "Unwind hypothesis": The crash started with rapid sell-offs of large blocks of factor strategy portfolios, for which there was not enough liquidity to maintain prices. The price drops, in turn, led to further liquidations due to 1) margin calls in other leveraged and long-short funds and 2) risk management policies, even in traditional long-only funds. This liquidity and margin spiral is very similar to that proposed by \textcite{Brunnermeier2009} and \textcite{BrunnermeierPedersen2009}.

% Other papers have also highlighted the risk of "crowded trades" where leverage is applied. \textcite{Stein2009} shows that markets can become less pricing efficient and have increased chances of large fire-sale crashes when rational investors set their leverage level unknowing of how many others are engaging in a similar trade. \textcite{LouPolk2013} introduce measures of arbitrage activity and show that momentum strategies become destabilizing and prone to crash in times of high activity.

\subsection{Modeling factor strategy returns}

Recently, copula models have attracted much attention in the field of risk management, as they provide a flexible way to infer a multivariate probability distribution. Furthermore, copulas are flexible in the sense that they can capture tail dependence, i.e. when the dependence structure changes in extreme times. Copula models are most often estimated taking popular univariate models such as ARMA-GARCH models as a starting point, and use a copula function to explain the multivariate dependence structure.

There are only a handful of papers that study factor strategies using copula methods. A working paper by \textcite{CholleteNing2012} examines dynamic correlations between a four-factor model and aggregate US consumption, and find evidence for tail dependence across the five risk factors. \textcite{ChristoffersenLanglois2013} study the four-factor model alone on US data 1963-2010, and show significant and asymmetric tail dependence that cannot be captured by standard linear correlation measures. A skewed Student's \textit{t} copula model is found to be able to generate the data fairly well, and the authors proceed with 20 years of out-of-sample analysis on investing based on conditional expectations from the copula model, leading to significant improvements for investors with a CRRA utility function.
